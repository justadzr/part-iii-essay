\documentclass[11pt, a4paper]{article}
    %\usepackage[notref, notcite]{showkeys}
    \usepackage[symbol]{footmisc}
    \usepackage{indentfirst}
    \usepackage{amsmath}
    \usepackage{amssymb}
    \usepackage{amsthm}
    \usepackage{subcaption}
    \usepackage[utf8]{inputenc}
    \usepackage[margin=1in]{geometry}
    \usepackage{diagbox}
	\usepackage{enumerate}
    \usepackage{siunitx}
    \usepackage{graphicx}
    \usepackage{bbm}
    \usepackage{multirow}
    \usepackage{xcolor}
    \usepackage{tikz-cd}
    \usepackage{dynkin-diagrams}
    \usepackage{mathrsfs}
    \usepackage{bm}
    \usepackage{epic}
    \usepackage[colorlinks = true,
            linkcolor = black,
            urlcolor  = black,
            citecolor = blue,]{hyperref}
    \usepackage[capitalize, nameinlink]{cleveref}
    \usepackage[backend=bibtex, firstinits=true, sortcites=true]{biblatex}
    \usepackage{sectsty}
    \usepackage{titlesec}
    \addbibresource{reference.bib}

    %%%%%%%%%%%%%%%%%%
    % Tile and Authors
    %%%%%%%%%%%%%%%%%%
    
    %%%%%%%%%%%%%%%%%%%%%%%%%%%%%%%%%%%%%%%%%%%%%%%%%%%
    % Use \begin{theorem}/{lemma}/{corollary} to access
    % Use \begin{proof}...\end{proof}
    %%%%%%%%%%%%%%%%%%%%%%%%%%%%%%%%%%%%%%%%%%%%%%%%%%%
	\newtheorem{theorem}{Theorem}[subsection]
	\newtheorem{lemma}[theorem]{Lemma}
	\newtheorem{corollary}[theorem]{Corollary}
	\newtheorem{proposition}[theorem]{Proposition}
	\theoremstyle{definition}
	\newtheorem{remark}[theorem]{Remark}
    \newtheorem{fact}[theorem]{Fact}
	\newtheorem{example}[theorem]{Example}
	\newtheorem{definition}[theorem]{Definition}
	\sectionfont{\normalsize}
    \titleformat{\subsection}[runin]{\normalfont\bfseries}{\thesubsection}{1em}{}
    \subsubsectionfont{\normalsize}
    
	
    %%%%%%%%%%%%%%%%%%%%%%%%%%%%%%%%%%%%%%%%%%%%%%%%%%%%%%%%%%%%%%%
    %%% List of macros see macro_list.tex
    %%% Please update the list after adding 
    %%% new macros with the date and your name
    %%% Do NOT use \H for the upper half plane. Define \bbH instead
    %%% Do NOT change any used macros
    %%%%%%%%%%%%%%%%%%%%%%%%%%%%%%%%%%%%%%%%%%%%%%%%%%%%%%%%%%%%%%%
    \newcommand{\gr}{\operatorname{gr}}
    \newcommand{\res}[2]{\underset{#1}{\,\operatorname{Res}\,}#2}
    \newcommand{\ord}[0]{\operatorname{ord}}
    \newcommand{\ind}[0]{\operatorname{ind}}
    \newcommand{\w}[0]{\omega}
    \newcommand{\ve}[0]{\varepsilon}
    \newcommand{\s}[0]{\sigma}
    \newcommand{\D}[0]{\Delta}
    \newcommand{\Z}[0]{\mathbb{Z}}
    \newcommand{\R}[0]{\mathbb{R}}
    \newcommand{\F}[0]{\mathbb{F}}
    \newcommand{\ecO}[0]{\mathcal O}
    \newcommand{\N}[0]{\mathbb{N}}
    \newcommand{\Q}[0]{\mathbb{Q}}
    \newcommand{\C}[0]{\mathbb{C}}
    \newcommand{\A}[0]{\mathbb{A}}
    \newcommand{\G}[0]{\mathbb{G}}
    \renewcommand{\O}{\mathcal O}
    \newcommand{\Pc}{\mathbb{P}}   % Projective
    \newcommand{\Ac}{\mathbb{A}}   % Affine
    \newcommand{\Lam}[0]{\Lambda}
    \newcommand{\coker}[0]{\operatorname{coker}}
    \newcommand{\kp}[0]{\kappa}
    \newcommand{\doubp}[1]{\left(\left(#1\right)\right)}
    \newcommand{\lbd}[0]{\lambda}
    \newcommand{\Aut}[0]{\operatorname{Aut}}
    \renewcommand{\Re}[0]{\operatorname{Re}}
    \renewcommand{\Im}[0]{\operatorname{Im}}
    \newcommand{\nS}[0]{\mathcal{S}}
	\newcommand{\M}[0]{\mathcal{M}}
    \newcommand{\To}[0]{\mathbb{C}/\Lambda}
    \newcommand{\Too}[0]{\mathbb{C}/\Lambda'}
    \newcommand{\mtx}[4]{\begin{bmatrix}#1 & #2\\ #3 & #4\end{bmatrix}}
    \newcommand{\vp}[0]{\varphi}
    \newcommand{\norm}[1]{\left\lVert#1\right\rVert}
    \newcommand{\proj}[0]{\operatorname{proj}}
    \newcommand{\lcm}[0]{\operatorname{lcm}}
    \newcommand{\leg}[2]{\left(\frac{#1}{#2}\right)}
    \newcommand{\sgn}[0]{\operatorname{sgn}}
    \newcommand{\mult}[0]{\operatorname{mult}}
    \newcommand{\ft}[0]{\mathscr{F}}
    \newcommand{\rad}[0]{\operatorname{rad}}
    \newcommand{\Spec}[0]{\operatorname{Spec}}
	\newcommand{\Proj}[0]{\operatorname{Proj}}
    \newcommand{\Sym}[0]{\operatorname{Sym}}
    \newcommand{\MaxSpec}[0]{\operatorname{MaxSpec}}
    \newcommand{\Gal}[0]{\operatorname{Gal}}
    \newcommand{\im}[0]{\operatorname{im}}
    \newcommand{\Hom}[0]{\operatorname{Hom}}
    \newcommand{\End}[0]{\operatorname{End}}
    \newcommand{\height}[0]{\operatorname{height}}
    \newcommand{\id}[0]{\operatorname{id}}
    \newcommand{\comment}[1]{}
    \newcommand{\Top}[0]{\mathsf{Top}}
    \newcommand{\Aff}[0]{\mathsf{Aff}}
	\newcommand{\Sch}[0]{\mathsf{Sch}}
    \newcommand{\Set}[0]{\mathsf{Set}}
    \newcommand{\op}[0]{\mathsf{op}}
    \newcommand{\Rep}[0]{\mathsf{Rep}}
    \newcommand{\Div}[0]{\operatorname{Div}}
    \newcommand{\pdiv}[0]{\operatorname{div}}
    \newcommand{\Pic}[0]{\operatorname{Pic}}
    \newcommand{\trdeg}[0]{\operatorname{trdeg}}
    \newcommand{\Stab}[0]{\operatorname{Stab}}
    \newcommand{\Span}[0]{\operatorname{Span}}
    \newcommand{\Ind}{\operatorname{Ind}}
    \newcommand{\pt}{\operatorname{pt}}
    \newcommand{\Ext}{\operatorname{Ext}}
    \newcommand{\gl}{\operatorname{\mathfrak{gl}}}
    \newcommand{\ad}{\operatorname{ad}}
    \newcommand{\tr}{\operatorname{tr}}
    \newcommand{\Lie}{\operatorname{Lie}}
    \newcommand{\Mod}{\operatorname{Mod}}
    \newcommand{\cMod}{\operatorname{coMod}}
    \newcommand{\rmod}{\mathsf{R\text{-}mod}}
    \newcommand{\h}{\mathfrak h}
    \renewcommand{\b}{\mathfrak b}
    \newcommand{\g}{\mathfrak g}
    \newcommand{\Diff}{\operatorname{Diff}}
    \newcommand{\Der}{\operatorname{Der}}
    \newcommand{\Ann}{\operatorname{Ann}}


    %%%% For derived functors.
    \newcommand{\dL}{\mathbf{L}}
    \newcommand{\dR}{\mathbf{R}}

    %%%% Spencer complex
    \newcommand{\Sp}{\operatorname{Sp}}

    %%%% Support and characteristic variety
    \newcommand{\ch}{\operatorname{ch}}
    \newcommand{\supp}{\operatorname{supp}}

    %%%% Bibliography font size
    \renewcommand*{\bibfont}{\small}
    \setcounter{section}{-1}
\begin{document}
    \newpage
    \tableofcontents
    \section{Introduction}
    The theory of $D$-modules, where $D$ stands for the sheaf of differential operators, began with the extensive study of algebraic analysis in the Kyoto school led by Sato and Kashiwara. Most notably, in Kashiwara's thesis \cite{kashiwara-master}, the theory of $D$-modules was developed in-depth in an analytic setting. Around the same time, the French mathematician Bernstein proposed an algebraic theory of $D$-modules, focusing on methods used in algebras and representation theory. A part of the work is rooted in the Borel-Weil theorem, where representations of semisimple algebraic groups were realized as global sections of invertible sheaves on flag varieties. Later becoming the vast geometric representation theory, this part of Bernstein's work culminated in his joint paper with Beilinson \cite{bb-original}, which established an equivalence between the representation theory of semisimple algebraic groups and the theory of $D$-modules on flag variety. To be more precise, given a semisimple Lie algebra $\g$, a fixed Borel subalgebra $\b$, and the corresponding flag variety $X=G/B$, there is a correspondence between modules over the sheaf of differential operators twisted by a one-dimensional representation $\lambda$ of $\b$ and $U(\g)$-modules with actions of the center $\mathfrak Z$ of $U(\g)$ determined by $\lambda$. The theorem is outstanding because it classifies not only finite-dimensional representations but all infinite-dimensional ones which are generally difficult to understand in the algebraic setting. One major application of Beilinson-Bernstein is half of the Kazhdan-Lusztig conjecture, while the other half is given by the Riemann-Hilbert correspondence, another crown jewel of the theory of $D$-modules which will not be discussed in this paper.

    In the first section of this essay, we focus on the sheaf of differential operators. Starting with an algebraic introduction to differential operators on algebras over a field of characteristic zero, we study their filtrations and filtered modules over these rings. Later the algebraic data will be globalized and turned into a sheaf of differential operator $\mathcal D_X$ on smooth varieties $X$. We will study the sheaf of differential operators between locally free sheaves, and study the filtration of them. The core of the first section is the structure of $\mathcal D_X$, i.e., being generated by the tangent sheaf and structure sheaf, and the structure of its associated graded ring.

    The second section is a juicy part of the essay. We start by studying left and right $\mathcal D_X$-modules and demonstrating their relations. Then I will delve into the big question of pushforwards and pullbacks of $\mathcal D_X$-modules. After realizing the naive definitions of pushforwards and pullbacks do not actually work except for the special examples of immersions, we will investigate a derived version of the constructions. The crucial theorem of Kashiwara regarding the pushforward of $D$-modules by closed immersions will be the heart of this section. With it proved, we will develop the notion of twisted sheaves of differential operators and the affinity of smooth varieties over twisted sheaves of differential operators. To understand the difference between $\mathcal O_X$-coherence and $\mathcal D_X$-coherence, we employ the characteristic variety of $D$-modules and prove the celebrated inequality of Bernstein. At the end, we introduce special $D$-modules known as holonomic $D$-modules which are limiting examples in Bernstein's inequality. We will state but not prove several theorems relevant to the properties of holonomic $D$-modules as they deviate from the purpose of this paper. However, I will illustrate them by producing a great number of examples and demonstrating computational results that draw a connection between holonomic $D$-modules and systems of partial differential equations with finite-dimensional solution spaces.

    The theory of representations finally comes into the picture in the last section. We will first study equivariant sheaves, and understand the Borel-Weil-Bott theorem through explicit computations. Then we will collect the necessary tools such as the Harish-Chandra isomorphism, Kostant's theorem, and a map that computes the global section of twisted sheaves of differential operators of integral weights on flag varieties. Then we will state and prove the Beilinson-Bernstein localization theorem for integral weights. To see what objects would replace our constructions for general weights, I will introduce Lie-Rinehart algebras, which are the algebraic data of Lie algebroids. The Lie algebroids are the tools to construct the correct sheaves in general. 

    At the very end, I will briefly discuss two different analogous versions of Beilinson-Bernstein for quantum groups $U_q(\g)$ due to Lunts-Rosenberg-Tanisaki, and Backelin-Kremnizer. Even though the representation theory of $U_q(\g)$ at a generic $q$ is almost parallel to the classical $U(\g)$, we encounter several difficulties when trying to replicate the classical constructions. We will work with noncommutative and non-cocommutative bialgebras, representations that are not locally finite, and many interesting oddities that never happen for $U(\g)$. A good thing is that the guiding philosophy of the approach by Backelin and Kremnizer is in fact discussed extensively in previous sections.

    In this article, we use normal $M, N$ to denote abstract modules and algebraic objects, and curly $\mathcal M$ and $\mathcal N$ to denote sheaves of modules. We will be dealing with schemes over a field $k$ of characteristic zero. By algebraic varieties over $k$ I mean separated schemes over $k$ of finite type that are geometrically integral. Given a smooth variety $X$, I denote by $\Theta_X$, $\Omega_{X/k}$ and $\w_{X/k}=\det\Omega_{X/k}$ the tangent, cotangent and canonical sheaves on $X$. I will write $TX$ and $T^*X$ for the tangent and cotangent bundles on $X$. To prevent confusion, I shall specify all notations before using them and recall them regularly.

    The overall reference for the essay is \cite{htt-d-modules}. I will take many interesting arguments and constructions in \cite{ginzburg-notes, schnell-notes} to enrich the first two sections. In the classical part of the third section, I will mostly rely on \cite{htt-d-modules, kashiwara-d-modules}. For the quantum analogy, I will stick to the original papers \cite{MR1694897, tanisaki-quantum, backelin-quantum}.
    \section{Differential Operators}
    \subsection{Weyl algebras, filtrations and differential operators}\label{sec-1-1}
    Let $k$ be a field of characteristic zero. We start with the polynomial ring $R=k[x_1,\dots, x_n]$. Let the \textbf{$n$th Weyl algebra} be the associative $k$-algebra generated by elements in $R$ and derivations of $R$. Write $\partial_i\in\Der_k(R)$ to be the derivative with respect to $x_i$. Then the generators of $R$ have the following relations: $[\partial_i,\partial_j]=[x_i, x_j]=0$ (as they commute) and $[\partial_i, x_j]=\delta_{ij}$. When $n=1$, we omit the index. Since $A_n$ is noncommutative, left and right $A$-modules are not equivalent in general.
    \begin{example}
        The algebra $R$ is naturally a left $A_n$-module on which $A_n$ acts as regular differential operators.
    \end{example}

    Let $P\in A_n$ and consider the partial differential equation $Pu=0$. Then naturally the quotient $M=A_n/A_nP$ is an $A_n$-module. Let $N$ be another $A_n$-module. Then the space of $A_n$-linear homomorphisms from $M$ to $N$ is
    \[\Hom_{A_n}(M, N)=\{\varphi\in \Hom_{A_n}(A_n, N):\varphi(P)=0\}\]
    Yet $\Hom_{A_n}(A_n, N)$ is isomorphic to $N$ via $\varphi\mapsto \varphi(1)$ with inverse $n\mapsto (Q\mapsto Qn)$. Therefore,
    \[\Hom_{A_n}(M, N)=\{u\in N:Pu=P\varphi(1)=\varphi(P)=0\}\]
    meaning $\Hom_{A_n}(M, N)$ is the space of solutions to $Pu=0$ in $N$. In general, consider the system of linear equations 
    \[\sum_{j=1}^q P_{ij}u_j=0, \quad i=1,2,\dots, p\]
    To describe the space of solutions, we consider the module $M$ defined by the exact sequence
    \[A_n^p\xrightarrow{\psi} A_n^q\to M\to 0\]
    where $\psi$ sends
    \[(Q_1,\dots, Q_p)\mapsto \left(\sum_i Q_iP_{i1},\dots,\sum_i Q_iP_{iq}\right)\]
    That is, $M$ is the cokernel of $\psi$. Using the same argument, we see that
    \[\Hom_{A_n}(M, N)=\{\varphi\in\Hom_{A_n}(A_n^q, N):\varphi\circ\psi=0\}\]
    Under the map $\varphi\mapsto(\varphi(e_i))_{i=1}^q$, the space $\Hom_{A_n}(A_n^q, N)$ can be identified with $N^q$. Thus, 
    \[\Hom_{A_n}(M, N)=\left\{u\in N^q:\forall i\leqslant p, \sum_j P_{ij}u_j=\sum_j P_{ij}\varphi(e_j)=(\varphi\circ\psi)(e_j)=0\right\}\]
    which is the space of solutions to the system of differential equations.
    \begin{example}\label{exp-Mf}
        The $A_1$-orbit of $1$ in $k[x]$ is the module $A_1/A_1\partial$, which is simply $k[x]$. Any $A_1$-homomorphisms from $k[x]\to k[x]$ is a scalar multiplication. Thus, the Hom space corresponds to the space of solutions to $\partial u=0$, i.e., constants. In general, given any $A_n$-module $N$ and an element $u\in N$. Define the $A_n$-module $M(u)=A_n\cdot u$. Then we have a surjection $s:A_n\to M(u)$ given by $P\mapsto Pu$, meaning $M(u)\cong A_n/\ker s$ where $\ker s$ is the space of differential equations satisfied by $u$. For instance, let $N$ be the module of meromorphic functions on $k$. Then $M(1/x)=A_1/A_1(\partial x)$; $M(x^r)=A_1/A_1(x\partial-r)$; $M(\log x)=A_1/A_1(\partial x\partial)$; $M(e^x)=A_1/A_1(\partial -1)$.
    \end{example}
    \begin{example}\label{exp-general-ode}
        Consider the ordinary differential equation $Pu=0$ where $P=a_n(x)\partial^n+\cdots+a_1(x)\partial+a_0(x)$ for $a_0,\dots, a_n\in k[x]$. We can reduce the order of the equation as usual: let $u_{i+1}=\partial u_{i}$ for $i=0,\dots, n-2$. Then the equation is equivalent to the system of equations $\partial u_{i}-u_{i+1}=0$ for $i=0,\dots, n-2$, and $a_n(x)\partial u_{n-1}+a_{n-1}(x)u_{n-1}+\cdots+a_1(x)u_1+a_0(x)u_0=0$. We compute the linear map $\psi:A_1^n\to A_1^n$ describing the linear system to be the $n$ by $n$ matrix
        \[\begin{bmatrix}
            \partial & -1 &  \cdots & 0 & 0 \\
            0 & \partial & -1 & \cdots & 0 \\
            &  &\cdots & &\\
            0 & \cdots & 0 &\partial & -1\\
            a_0(x) & a_1(x) & \cdots & a_{n-2}(x) & a_n(x)\partial+a_{n-1}(x)
        \end{bmatrix}\]
        And the associated $A_1$-module is $M=\coker\psi$.
    \end{example}
    \begin{example}
        In the settings of \cref{exp-general-ode}, we can also construct the obvious $A_1$-module $M=A_1/A_1P$. By the virtue of \cref{exp-Mf}, given a solution $u$ to $P$, we have $M=A_1\cdot u$. But then on the set $U=\{x:a_n(x)\neq 0\}$, $M=\bigoplus_{i=0}^nk[x]u^{(i)}$ where $u^{(i)}=\partial^iu$. We therefore identified $M$ with the one in \cref{exp-general-ode}.
    \end{example}
    \begin{definition}
        Given a $k$-algebra $A$, a \textbf{filtration} of $A$ is a family $F_\bullet A$ of subspaces
        \[0=F_{-1}A\subseteq F_0A\subseteq F_1A\subseteq\cdots\]
        such that $F_iF_jA\subseteq F_{i+j}A$, $1\in F_0$ and $A=\cup_{l=0}^\infty F_l$. Sometimes we write $F_\bullet$ when the base algebra is clear. The \textbf{graded algebra associated to $F$} is the direct sum
        \[\gr^F A=\bigoplus_{l=0}^\infty\gr^F_lA=\bigoplus_{l=0}^\infty F_l/F_{l-1}\]
    \end{definition}
    Notice that $\gr^F A$ is naturally a graded $k$-algebra with well-defined multiplications given by $[x]\cdot[y]=[xy]$. Indeed, for $x, x'\in F_l$ and $y, y'\in F_m$, if $[x']=[x]$ and $[y']=[y]$, we have $x'y'=(x+p)(y+q)=xy+py+qx+pq$ for some $p\in F_{l-1}$, $q\in F_{m-1}$, so $py+qx\in F_{l+m-1}$ and $pq\in F_{l+m-2}\subseteq F_{l+m-1}$. On the $n$th Weyl algebra $A_n$, we have an \textbf{order filtration} given by
    \[F_l=\sum_{|\alpha|\leqslant l}R\cdot\partial^\alpha\]
    where $\alpha$ is an $n$-tuple, $|\alpha|=\alpha_1+\cdots+\alpha_n$, and $\partial^\alpha=\partial_1^{\alpha_1}\cdots\partial_n^{\alpha_n}$. It is easy to deduce the following
    \begin{proposition}
        Let $F_\bullet$ be the order filtration of $A_n$. Then
        \[\gr^F A_n\cong k[x_1,\dots, x_n,\xi_1\dots, \xi_n]\]
        which is a commutative $k$-algebra graded by orders of $\xi_i$.
    \end{proposition}
    \begin{proof}
        Let $x_1,\dots, x_n$ and $\xi_1,\dots,\xi_n$ be the images of $x_i$ and $\partial_i$ in $F_1/F_0$. Then by construction $F_l/F_{l-1}$ consists of monomials in terms of $x_i$ and $\xi_i$ such that the sum of degrees of $\xi_i$ is $l$. It remains to show commutativity. But this is easy, since for any $i, j$, we have $[\partial_i, x_j]=\delta_{ij}\in F_0$ so $[\xi_i, x_j]=0$ in $F_{1}/F_0$.
    \end{proof}
    \begin{definition}\label{def-almost-comm}
        We say an algebra $A$ admitting a filtration $F$ is \textbf{almost commutative} if $\gr^F A$ is commutative.
    \end{definition}
    The associated graded algebra is actually a simple but powerful tool to determine whether two filtered algebras are isomorphic.
    \begin{lemma}\label{lem-gr-iso-is-iso}
        Let $A$ and $B$ be two filtered $k$-algebras with filtrations $F_\bullet A$ and $F_\bullet B$. Given a map $f:A\to B$, if the induced map $\tilde f:\gr^FA\to\gr^FB$ is injective (resp. surjective), then $f$ is injective (resp. surjective).
    \end{lemma}
    \begin{proof}
        The core of this lemma is the exhaustive property of filtrations (assumed in our definition). Suppose $\tilde f$ is injective. For any nonzero $a\in A$, there is a minimal $k$ such that $a\in F_kA$. Then $a$ has a nonzero image in $\gr^FA$, i.e., $\tilde f(a)\neq 0$. But this is the class of $f(a)$, so $f(a)$ must be nonzero. Suppose $\tilde f$ is surjective. Take any $b\in B$. Let $k$ be the minimal number such that $b\in F_kB$. Denote still by $b$ its preimage in $\gr^F_kB$ and $a$ the preimage in $\gr^F_kA$ under $\tilde f$. Let $a\in F_kA$ be the element with class $a$ in $\gr^F_kA$. Then $b-f(a)$ vanish in $\gr^FB$, i.e., it's in $F_{k-1}B$. Then we can find another $a'\in F_{k-1}A$ with image $b-f(a)$, in which case $b-f(a)-f(a')$ is zero in $\gr^FB$. Repeat this process, we end with an element in $F_kA$ with image $b$ under $f$.
    \end{proof}
    \begin{example}
        The universal enveloping algebra $U(\g)$ of some Lie algebra $\g$ has a filtration $\mathit{PBW}_\bullet U(\g)$ by degrees of monomials given by Poincaré-Birkhoff-Witt. In this case $\gr^{\mathit{PBW}} U(\g)$ is commutative because for any $x, y\in\g\subseteq \mathit{PBW}_1 U(\g)$, we have $xy-yx=[x, y]\in\g\in \mathit{PBW}_1 U(\g)$ so commutators vanishes in $\gr^{\mathit{PBW}}U(\g)$.
    \end{example}
    Recall that a \textbf{Poisson structure} on an algebra $A$ is a Lie bracket that satisfies the Leibniz relation.
    \begin{lemma}
        If $A$ is an almost commutative algebra with filtration $F_\bullet A$, then $\gr^F A$ has a canonical Poisson structure.
    \end{lemma}
    \begin{proof}
        Let $x_i\in\gr^F_iA$ and $x_j\in\gr^F_jA$, and pick representatives $a_i\in F_iA$ and $a_j\in F_jA$. Define the Poisson bracket $\{x_i, x_j\}$ to be the class of $a_ia_j-a_ja_i$ in $\gr^F_{i+j-1}A$ ($[a_i, a_j]$ is an element in $F_{i+j-1}A$ because $\gr^F A$ is commutative). It is easy to check if $b_i\in F_{i-1}A$ then $b_ia_j-a_jb_i\in F_{i+j-2}A$ so the bracket is well-defined in $\gr^FA$. By construction the bracket is a Lie algebra and satisfies the Leibniz relation.
    \end{proof}
    \begin{example}
        The Poisson bracket on $\gr^F U(\g)$ is given by $\{x^i, y^j\}=ijx^{i-1}y^{j-1}[x, y]$ for any $x, y\in\g$. This can be easily seen using the construction of $\{\cdot, \cdot\}$ and Leibniz.
    \end{example}
    It is also tempting to study filtrations of $A$-modules when $A$ admits a filtration.
    \begin{definition}\label{def-filtered-module}
        Let $M$ be an $A$-module for a filtered $k$-algebra $A$ with filtration $F_\bullet A$. It has a \textbf{filtration} if there is an increasing family of additive subgroups $F_\bullet M$
        \[0=F_{-1}M\subseteq F_0M\subseteq F_1M\subseteq\cdots\subseteq M\]
        with $F_lA\cdot F_sM\subseteq F_{l+s}M$ and $M=\cup_s F_sM$. We say a filtration on $M$ is \textbf{good} if each $F_sM$ is finitely generated over $F_0A$, and for large enough $s$, $F_{l+s}M=F_lA\cdot F_sM$ for any $l$.
    \end{definition}
    \begin{remark}
        There are various ways to define a good filtration on $M$, depending on the context. All examples we encounter will be filtrations such that each $F_sM$ is finitely generated as an $F_0A$-module. We therefore include this condition in the definition of good filtrations.
    \end{remark}
    To determine whether a filtration on an $A$-module is good, we have the following convenient criterion
    \begin{proposition}\label{prop-good-gr-finite}
       If the filtration on $A$ is good (considered as a module over itself), then $F_\bullet M$ is good if and only if $\gr^F M$ is finitely generated over $\gr^F A$.
    \end{proposition}
    \begin{proof}
        If there exists an $s_0$ such that $F_{l+s}M=F_lA\cdot F_sM$ for all $s\geqslant s_0$, then $\gr^F M$ is generated by all components $\gr_s^FM=F_sM/F_{s-1}M$ for $s\leqslant s_0$. As each $F_sM$ is finitely generated over $F_0A$, $\gr_s^FM$ is finitely generated over $\gr_0^F A$. But then we get a finite number of generators of $\gr^FM$ over $\gr^FA$.

        Conversely, let $\{u_i\}$ be a finite set of generators of $\gr^F M$ over $\gr^F A$. We can assume they are homogeneous. Let $s_0$ be the maximum of their degrees. Then we have $\gr_s^F M=\sum_{t=0}^{s_0} \gr_{s-t}^FA\cdot\gr_t^F M$ for any $s\geqslant s_0$. This suggests
        \[F_sM=F_{s-1}M+\sum_{t=0}^{s_0} F_{s-t}A\cdot F_tM=F_{s-1}M+ F_{s-s_0}A\cdot\sum_{t=0}^{s_0} F_{s_0-t}A\cdot F_t M\]
        Here we used the fact that $A$ has a good filtration. After absorptions, the RHS becomes $F_{s-s_0}A\cdot F_{s_0}M$. Since $F_sM=F_{s-s_0}A\cdot F_{s_0}M$ for all $s\geqslant s_0$ and $F_\bullet A$ is good, we are done.
    \end{proof}
    \begin{lemma}\label{lem-good-finite}
        If $A$ has a good filtration, then so does $M$ if and only if $M$ is finitely generated over $A$.
    \end{lemma}
    \begin{proof}
        Suppose $M$ is finitely generated over $A$ by $m_1,\dots, m_r$. Define $F_sM=\sum_i F_sA\cdot m_i$. Observe that $F_sA$ is finitely generated over $F_0A$ from $F_\bullet A$ being good. Every $F_sM$ is therefore finitely generated over $F_0A$, and the above sum becomes $F_sM=F_sA\cdot F_0M$, that is, $F_\bullet M$ is good.

        On the other hand, if $M$ has a good filtration $F_\bullet M$, then there is some $s_0$ such that $F_sM=F_{s-s_0}A\cdot F_{s_0}M$ for all $s\geqslant s_0$.  Therefore $M$ is generated by elements in $F_{s_0}M$ over $A$, and by definition $F_{s_0}M$ is finitely generated over $F_0A\subseteq A$ so we are done.
    \end{proof}
    Next, we shall follow Grothendieck's approach to define differential operators. Let $k$ be a field of characteristic zero, $R$ a commutative $k$-algebra, and $M$ an $R$-bimodule such that elements in $k$ commute with all elements in $M$. We define a family of subspaces $M_l'$ by $M_{-1}'=0$ and $M_l'=\{x\in M: \forall r\in R, [r, x]\in M_{l-1}'\}$. Define the submodule $M'=\cup_{l=-1}^\infty M_l'$. We call $M'$ the \textbf{differential part} of $M$, and if $M=M'$ then we say $M$ is a \textbf{differential bimodule}. Now for any two left $R$-modules, $X, Y$, we can define
    \begin{definition}
        The module of \textbf{$k$-linear differential operators from $X$ to $Y$} is the $R$-bimodule $\Diff(X, Y)=\Hom_k(X, Y)'$. We denote by $D(R)$ the ring $\Diff(R, R)$ and $D_l(R)=\Diff(R, R)_l$.
    \end{definition}
    \begin{remark}
        By construction $\Diff(X, Y)=\cup_{l=-1}^\infty\Diff(X, Y)_l$ where $f\in\Diff(X, Y)_l$ if and only if 
        \[[r_{l+1},[r_l,\dots,[r_1, f]]]=0,\quad\forall r_i\in R\]
        Moreover we always have $D_l(R)D_m(R)\subseteq D_{l+m}(R)$ (easy to show if one writes $rx=xr+a$ for $x\in D_\bullet R$ and $a\in D_{\bullet-1}R$).
    \end{remark}
    One extremely important property of the constructed filtration is perhaps its compatibility with localizations. Namely, we have the following:
    \begin{proposition}\label{prop-loc}
        Let $M$ be an $R$-bimodule. Given any non-nilpotent $f$ in $R$, take $M_f$ to be the (two-sided) localization of $M$ at $f$, that is, $M_f=R_f\otimes_R M\otimes_R R_f$. Then the filtered differential part $M'$ localizes to
        \[(M_f)_l'=(M'_l)_f\cong R_f\otimes_R M'_l\cong M'_l\otimes_R R_f\]
        for any $l\geqslant 0$. In particular, we have $D_l(R_f)=R_f\otimes_R D_l(R)$ and thus $D(R_f)\cong R_f\otimes_R D(R)$.
    \end{proposition}
    \begin{proof}
        View $M$ as an $R\otimes R$-module. Let $J$ be the submodule in $R\otimes R$ generated by elements of the form $r\otimes 1-1\otimes r$. Then via an inductive argument, we obtain $M_l'=\{x\in M: J^{l+1}x=0\}$. In this case, we see that $J^{l+1}x=0$ if and only if for any $m$ we have $J_f^{l+1}(x/f^m)=0$ where $J_f$ is the kernel of the multiplication $R_f\otimes R_f\to R_f$. Indeed, $f$ is not nilpotent so we can freely remove the powers of $f$. Thus, $(M'_l)_f=(M_f)_l'$, and the latter two congruences are evident from the universal property of localizations.
    \end{proof}
    In the case of Weyl algebras, we observed that nice filtrations could carry a lot of information. For any commutative $k$-algebra $R$, consider the graded $k$-algebra $D(R)$. By definition, $D_0(R)=\{f\in \End_k(R):\forall r\in R, [r, f]=0\}$. Via the identification $f\mapsto f(1)$, $D_0(R)$ is isomorphic to $R$. Elements in $D_1(R)$ are then those $P\in\End_k(R)$ such that for any $r\in R$, $[r, P]$ is an element of $R$, say $f_r$. Then for any $r_1, r_2\in R$, $f_{r_1}r_2=r_1P(r_2)-P(r_1r_2)$. But if we consider $\xi_P=P(1)-P\in\End_k(R)$, it is immediate that
    \[\xi_P(r_1r_2)=P(1)r_1r_2-r_1P(r_2)+f_{r_1}r_2=2P(1)r_1r_2-r_1P(r_2)-r_2P(r_1)=r_1\xi_P(r_2)+r_2\xi_P(r_1)\]
    where we use $r_2P(r_1)=P(1)r_1r_2-f_{r_1}r_2$ in the second equality. Hence, the map $P\mapsto (P(1), \xi_P)$ is an isomorphism from $D_1(R)$ to $D_0(R)\oplus\Der_k(R)$ with inverse given by $(f, \xi)\mapsto f+\xi$. To extend the discussion, we have the proposition below
    \begin{proposition}\label{prop-d-filtration}
        Let $R$ be a commutative $k$-algebra, if $R$ has rank $n$ and $\Der_k(R)$ is free of rank $n$ over $k$, then $F_\bullet D(R)=D_\bullet(R)$ is a filtration of $R$ such that
        \begin{enumerate}[\normalfont(i)]
            \item We have $\gr^F_l D(R)=D_l(R)/D_{l-1}(R)\cong\Sym^l \Der_k(R)$. That is, as graded $R$-algebras,
            \[\gr^F D(R)\cong\Sym \Der_k(R)\]
            \item As an $R$-subalgebra of $\End_k(R)$, $D(R)$ is generated by elements $f\in R$, $\xi\in\Der_k(R)$ such that $[\xi, f]=\xi(f)$.
        \end{enumerate}
    \end{proposition}
    \begin{proof}
        (i) The isomorphism $D_0(R)\oplus \Der_k(R)\to D_1(R)$ induces an $R$-linear map $\Der_k(R)\to \gr^F_1 D(R)$. Now if $\gr^F D(R)$ is commutative, the above extends to a map $\Sym\Der_k(R)\to \gr^F D(R)$ respecting the grading since $D(R)$ is graded. Indeed, given $P\in D_l(R)$, $Q\in D_m(R)$, then for any $r\in R$, $[r,[P, Q]]=r[P, Q]-[P, Q]r=rPQ-rQP-PQr+QPr$. Write $rP=Pr+a$ and $rQ=Qr+b$ for some $a\in D_{l-1}(R)$, $b\in D_{m-1}(R)$. In this case,
        \begin{align*}
            [r, [P, Q]]&=PrQ+aQ-QrP-bP-PQr+QPr\\
            &=PQr-QPr-PQr+QPr-Qa+Pb+aQ-bP\\
            &=Pb-bP+aQ-Qa\in D_{l+m-1}(R)
        \end{align*}
        so $[P, Q]$ vanishes in $\gr^F_{l+m}D(R)$. We want to show this is an isomorphism that respects grading, so it suffices to construct an inverse on each graded component $\gr^F_l D(R)\to \Sym^i \Der_k(R)$. For $l=0$, we have the obvious map $D_0(R)\xrightarrow{\sim} R$. Now given any $P\in D_l(R)$ consider the map $\varphi_P:R\to D_{l-1}(R)\twoheadrightarrow\gr^F_{l-1}D(R)$ given by $f\mapsto [P, f]$. Clearly, $\varphi_P$ is a derivation in $\Der_k(R, \gr^F_{l-1}D(R))$. Observe that $\varphi_P$ vanishes for $P\in D_{l-1}(R)$, so we have a map $\gr_l^FD(R)\to\Der_k(R, \gr^F_{l-1}D(R))$ defined by $P\mapsto\varphi_P$ and
        \[\Der_k(R, \gr^F_{l-1}D(R))=\Der_k(R)\otimes\gr^F_{l-1}D(R)\cong\Der_k(R)\otimes\Sym^{l-1}\Der_k(R)\twoheadrightarrow\Sym^l\Der_k(R)\]
        where the isomorphism is given by induction. Explicitly, each $P\in D_{l}(R)$ is sent to $\xi Q$ for some $\xi\in\Der_k(R)$ and $Q\in \gr^{F}_{l-1}D(R)$ such that for each $f\in R$, $Q\xi(f)=[P, f]$. But then $(P-P(1))f=Pf-P(1)f=[P, f](1)=Q\xi(f)(1)$ which in the commutative algebra $\gr^FD(R)$ equals to $\xi(f)Q$. In conclusion, $P\mapsto\varphi_P$ is an inverse of the map $\Sym\Der_k(R)\to\gr^FD(R)$.

        (ii) For the proof of the second statement, I'd like to extend a bit by quoting the following lemma
        \begin{lemma}
            Let $A$ be an almost commutative filtered $k$-algebra such that $F_0A=k$ and $A$ is generated by $F_1A$, then $A\cong U(\g)/I$ for some Lie algebra $\g$ and an ideal $I$ of $U(\g)$.
        \end{lemma}
        \begin{proof}
            Since $A$ is almost commutative, the commutator $[a, b]\in F_0A\subseteq F_1A$ for any $a, b\in F_1A$. Thus, $[\cdot, \cdot]$ defines a Lie bracket on $\g=F_1A$. Therefore, there exists a map $U(\g)\to A$ by the universal property of $U(\g)$ applied to the map $\g\hookrightarrow A$. The map is surjective because $A$ is generated by $\g$.
        \end{proof}
        Back to our case of $D(R)$. The vector space $\g=D_1(R)=R\oplus\Der_k(R)$ clearly has a Lie bracket given by $[f,g]=0$, $[\xi, f]=\xi(f)$ and $[\xi,\zeta]=\xi\zeta-\zeta\xi$ for any $f, g\in R$ and derivations $\xi, \zeta$. Let $I$ be the ideal identifying $1\in U(\g)$ with $1\in R$, multiplications in $U(\g)$ with multiplications in $\g$. Then we have a map $U(\g)/I\to D(R)$ that respects grading, so we get $\gr^{\mathit{PBW}} U(\g)/I\to\gr^F D(R)$. With the relations defined by $I$, we get a surjection $\Sym\Der_k(R)\to\gr^{\mathit{PBW}} U(\g)/I$. Their composition is precisely the map constructed in (i), so $\gr^{\mathit{PBW}} U(\g)/I\cong\gr^F D(R)$ and by \cref{lem-gr-iso-is-iso}, $D(R)\cong U(\g)/I$ completing the proof.
    \end{proof}
    \begin{remark}
        It is immediately apparent to the reader that differential operators are somehow associated to Lie algebras. We will investigate a much deeper connection later.
    \end{remark}

    \subsection{Sheaf of differential operators}
    In this section we investigate differential operators on smooth algebraic varieties. We have so far studied affine/local data of differential operators. Indeed, we should be able to define, globally, differential operators on smooth algebraic varieties over some field $k$ by pasting the data. From now on, fix a field $k$ of characteristic zero. Again, by algebraic varieties over $k$ I mean separated schemes over $k$ of finite type that are geometrically integral. Of course, when $k$ is algebraically closed, algebraic varieties are integral. We make the following definitions
    \begin{definition}\label{def-ega-d}
        Let $X$ be a scheme over $k$. Let $\mathcal M$ and $\mathcal N$ be two $\mathcal O_X$-modules on $X$. Let $U$ be an open of $X$. We define differential operators inductively. We define elements $P\in\Diff(\mathcal M|_U, \mathcal N|_U)_l$ to be $k$-morphisms \[P:\mathcal M|_U\to \mathcal N|_U\] 
        such that for all local sections $f\in\mathcal O_X$, the morphism $s\mapsto fP(s)-P(fs)\in\Diff(\mathcal M|_U, \mathcal N|_U)_{l-1}$ (whenever they are defined). By convention for all $l<0$, set $\Diff(\mathcal M|_U, \mathcal N|_U)_l=0$. Let $\Diff(\mathcal M|_U, \mathcal N|_U)$ be the union of all such modules. Define the sheaf of differential operators of order $\leqslant l$, $F_l\mathcal D_X(\mathcal M,\mathcal N)$, to be the sheaf $U\mapsto\Diff(\mathcal M|_U, \mathcal N|_U)_l$ and the \textbf{sheaf of differential operators} to be $\mathcal D_X(\mathcal M,\mathcal N)=\cup_{l=-1}^\infty F_l\mathcal D_X(\mathcal M,\mathcal N)$.
    \end{definition}
    But to make $\mathcal D_X(\mathcal M, \mathcal N)$ quasicoherent, we need extra structure on $\mathcal M$ and $\mathcal N$.
    \begin{lemma}\label{lem-diff-bij}
        If $X\to \Spec k$ is locally of finite presentation, $\mathcal M$ is coherent and $\mathcal N$ is quasicoherent, then each $F_l\mathcal D_X(\mathcal M,\mathcal N)$ is quasicoherent.
    \end{lemma}
    \begin{proof}
        We will not prove this, as the general definition is rarely used in this paper. But we \textit{will} prove the quasicoherence of $\mathcal D_X$ later (cf. \cref{cor-qcoh-dx}) as well as a special case of the statement (cf. \cref{cor-twist-end}). The reader might consult IV.16.8.6 in \cite{ega} for a proof of this general statement.
    \end{proof}
    From now on, we also assume $X$ to be a smooth algebraic variety over $k$. We have defined a (both left and right) $\mathcal O_X$-module $\mathcal D_X$ such that on each affine $U$ it's just the algebra of differential operators (apply \cref{lem-diff-bij} to the coherent $\mathcal O_X$-modules $\mathcal M=\mathcal N=\mathcal O_X$). In \cref{prop-loc}, we computed sections of $\mathcal D_X$ at distinguished opens in an affine open. It is almost immediate that
    \begin{corollary}\label{cor-qcoh-dx}
        Each (both left and right) $\mathcal O_X$-module $F_l\mathcal D_X$ is quasicoherent over $\mathcal O_X$. In particular, $\mathcal D_X$ is quasicoherent.
    \end{corollary}
    \begin{proof}
        On each affine $U=\Spec R\subseteq X$ and any non-nilpotent $f\in R$, by \cref{prop-loc} we have
        \[\Gamma(D(f), F_l\mathcal D_X)=D_l(R_f)=R_f\otimes_R D_l(R)=D_l(R)\otimes_R R_f\]
        which suggests $F_l\mathcal D_X$ is quasicoherent as a left and right $\mathcal O_X$-module.
    \end{proof}
    \begin{remark}
        Although we did not show the quasicoherence of general $\mathcal D_X(\mathcal M,\mathcal N)$, for $\mathcal M$ and $\mathcal N$ locally free, we can prove a similar result.
    \end{remark}
    \begin{corollary}\label{cor-twist-end}
        Given locally free $\mathcal M$ and $\mathcal N$, as $\mathcal O_X$-modules, we have
        \[\mathcal D_X(\mathcal M,\mathcal N)\cong \mathcal N\otimes_{\mathcal O_X}\mathcal D_X\otimes_{\mathcal O_X}\mathcal M^\vee\]
        where $\mathcal M^\vee$ denotes the locally free dual $\mathcal Hom_{\mathcal O_X}(\mathcal N,\mathcal O_X)$. In particular, $\mathcal D_X(\mathcal M,\mathcal N)$ is quasicoherent when $\mathcal M$ and $\mathcal N$ are locally free.
    \end{corollary}
    \begin{proof}
        First consider the isomorphism
        \[\mathcal N\otimes_{\mathcal O_X}\mathcal End_k(\mathcal O_X)\otimes_{\mathcal O_X}\mathcal M^\vee\xrightarrow{\sim}\mathcal Hom_k(\mathcal M,\mathcal N)\]
        given by $n\otimes F\otimes \mu\mapsto (x\mapsto F(\mu(x))n)\in \mathcal Hom_k(\mathcal M,\mathcal N)$. Since $\mathcal M$ and $\mathcal N$ are locally free, they have local $A$-bases given by $\{e_i\}_{i\leqslant r}$ and $\{\ve_j\}_{j\leqslant s}$ where $r$ and $s$ are the rank of $\mathcal M$ and $\mathcal N$. Denote by $\{e_i^\vee\}$ and $\{\ve_j^\vee\}$ the dual bases (exist by local freeness) over $A$. Then the map above has an inverse
        \[(\Phi:\mathcal M\to \mathcal N)\mapsto\sum_{i, j}\ve_j\otimes\Phi_{ij}\otimes e_i^\vee\in \mathcal N\otimes_{\mathcal O_X}\mathcal End_k(\mathcal O_X)\otimes_{\mathcal O_X}\mathcal M^\vee\]
        where $\Phi_{ij}:\mathcal O_X\to\mathcal O_X$ is the $k$-linear morphism defined by
        \[\Phi_{ij}(x)=\ve_j^\vee(\Phi(x\cdot e_i)),\quad x\in\mathcal O_X.\]
        The differential part of $\mathcal Hom_k(\mathcal M,\mathcal N)$ is precisely $\mathcal D_X(\mathcal M,\mathcal N)$, so it remains show the differential part of the tensor product is $\mathcal N\otimes\mathcal D_X\otimes \mathcal M^\vee$. But this is easy, since the tensor product is taken over $\mathcal O_X$ so the action of $\mathcal O_X$ can be passed to $\mathcal End_k(\mathcal O_X)$, giving us $\mathcal D_X$ in the middle factor.
    \end{proof}
    Suppose $\dim X=n$. Notice that since $X$ is smooth, on each affine open $U\subseteq X$, $R=\Gamma(U,\mathcal O_X)$ is generated by some $x_1,\dots, x_n$ and $\Gamma(U, \Theta_X)$ is generated by $\partial_1,\dots,\partial_n$ with $\Theta_X$ being the tangent sheaf of $X$, meaning $\{x_i,\partial_i\}_{i\leqslant n}$ is an étale coordinate system. By \cref{prop-d-filtration}, $\Gamma(U, \mathcal D_X)$ is the $R$-subalgebra generated by elements $f\in R$, $\xi\in\Gamma(U, \Theta_X)$ such that $[\xi, f]=\xi(f)$. Since all $F_l\mathcal D_X$ are quasicoherent, both statements in \cref{prop-d-filtration} can be globalized, obtaining the following result.
    \begin{proposition}\label{prop-dx-filtration}
        Let $X$ be a smooth algebraic variety over $k$. Then $F_l\mathcal D_X$ is a quasicoherent filtration of $\mathcal D_X$ such that
        \begin{enumerate}[\normalfont(i)]
            \item We have $\gr_l^F\mathcal D_X=F_l\mathcal D_X/F_{l-1}\mathcal D_X\cong\Sym^l\Theta_X$. That is, as graded $\mathcal O_X$-algebras,
            \[\gr^F\mathcal D_X=\bigoplus_{l=0}^\infty \gr_l^F\mathcal D_X\cong\Sym\Theta_X\]
            \item An a subsheaf of $\mathcal End_k(\mathcal O_X)$, $\mathcal D_X$ is generated by elements (i.e., local sections whenever they are defined) $f\in\mathcal O_X$ and $\xi\in\Theta_X$ such that $\xi\zeta-\zeta\xi=[\xi,\zeta]$ with the commutator in $\Theta_X$, and $[\xi, f]=\xi(f)$. 
        \end{enumerate}
    \end{proposition}
    Therefore, on each affine open $U$ we can write
    \[\mathcal D_U=\bigoplus_{\alpha\in\N^n}\mathcal O_U\partial_1^{\alpha_1}\cdots\partial_n^{\alpha_n}, \quad \gr^F \mathcal D_X=\mathcal O_U[\xi_1,\xi_2,\dots,\xi_n]\]
    where $\xi_i$ is the image of $\partial_i$ in $\gr^F_l\mathcal D_X$.
    \section{Algebraic D-Modules}
    \subsection{General theory}
    We start with a general theory of $D$-modules.
    \begin{definition}
        A \textbf{left $D$-module} on $X$ is an $\mathcal O_X$-module endowed with a left $\mathcal D_X$-module structure (i.e., on each open subset of $X$ compatible with restrictions). Similarly a \textbf{right $D$-module} is an $\mathcal O_X$ module with a right $\mathcal D_X$-module structure.
    \end{definition}
    Unless otherwise stated for the purpose of generality, we assume all $D$-modules in this paper are \textit{quasicoherent}. By (ii) in \cref{prop-dx-filtration}, we can check easily if a left action of $\Theta_X$ on a quasicoherent $\mathcal O_X$-module $\mathcal M$ determines a left $D$-module structure. The only concern is the action of $\xi\in\Theta_X$ on $\mathcal M$. We need for any such $f\in\mathcal O_X$ and $\xi\in\Theta_X$,
    \[f\cdot(\xi\cdot m)=(f\xi)\cdot m,\quad \xi(f)\cdot m=\xi\cdot(f\cdot m)-f\cdot (\xi\cdot m),\quad [\xi,\zeta]\cdot m=\xi\cdot(\zeta\cdot m)-\zeta\cdot(\xi\cdot m)\]
    If all relations above are satisfied, $\mathcal M$ is a left $D$-module. Similarly one can write down the criterion for right $D$-modules:
    \[(m\cdot \xi)\cdot f=m\cdot(\xi f),\quad m\cdot\xi(f)=(m\cdot\xi)\cdot f-(m\cdot f)\cdot \xi,\quad [\xi, \zeta]\cdot m=(m\cdot\xi)\cdot\zeta-(m\cdot\zeta)\cdot\xi\] 
    I must remind the reader that the associativity axiom for right modules is given by $(x\cdot r)\cdot s=x\cdot (rs)$, which is why the two sides of the minus sign are somehow interchanged. Since $\mathcal D_X$ is not commutative, the left and right $D$-modules cannot be thought as the same for the moment. We will discuss some techniques that will eventually connect the two categories together. Before that, here are some examples:
    \begin{example}
        The $\mathcal O_X$-modules $\mathcal D_X$ and $\mathcal O_X$ are examples of $D$-modules on $X$. On $\mathcal D_X$, $\mathcal D_X$ acts by multiplications (both left and right). On $\mathcal O_X$, $\mathcal D_X$ acts on the \textit{left} by differentiations.
    \end{example}
    \begin{example}
        All $A_n$-modules in \cref{sec-1-1} are (global sections of) left $D$-modules on the affine space $\mathbb A^n_k$.
    \end{example}
    \begin{example}\label{exp-canonical-bundle}
        The canonical bundle $\w_{X/k}=\det(\Omega_{X/k})=\wedge^n_{i=1}\Omega_{X/k}$ where $\Omega_{X/k}$ is the cotangent sheaf on $X$ is a \textit{right} $D$-module. For any $f\in\mathcal O_X$, $\xi\in\Theta_X$, the actions on $\w\in\w_{X/k}$ are given by
        \[\w\cdot f=f\w,\quad w\cdot\xi=-\Lie_\xi\w\]
        where the Lie derivative of the top form $\w$ is defined as usual: for any algebraic vector fields $\xi_1,\dots,\xi_n$,
        \[\Lie_\xi\w(\xi_1,\dots,\xi_n)=\xi(\w(\xi_1,\dots,\xi_n))-\sum_{i=1}^n\w(\xi_1,\dots,[\xi,\xi_i],\dots,\xi_n)\]
        Using a combinatorial argument (or simply assuming properties of Lie derivatives), the reader may check that
        \[\Lie_\xi(f\w)=f\Lie_\xi\w+\xi(f)\w=\Lie_{f\xi}\w,\quad \Lie_{[\xi,\zeta]}\w=\Lie_{\xi}\Lie_{\zeta}\w-\Lie_{\zeta}\Lie_\xi\w\]
        and thus
        \[\w\cdot\xi(f)=(\w\cdot\xi)\cdot f-(\w\cdot f)\cdot\xi,\quad \w\cdot[\xi,\zeta]=(\w\cdot\xi)\cdot\zeta-(\w\cdot\zeta)\cdot\xi\]
        meaning $\w_{X/k}$ is a right $D$-module. On each local affine open $U$ with an étale coordinate system $\{x_i,\partial_i\}_{i\leqslant n}$, the action is given by
        \[((fdx_1\wedge\cdots\wedge dx_n)\cdot P)(x,\partial)=({}^tP(x,\partial)f)dx_1\wedge\cdots\wedge dx_n\]
        where, given that $P(x,\partial)=\sum_{\alpha\in\N^n} a_\alpha(x)\partial^\alpha\in\mathcal D_U$, we define the adjoint of $P$ as
        \[{}^tP(x,\partial)=\sum_{\alpha\in\N^n}(-\partial)^\alpha a_\alpha(x)\in \mathcal D_U\]
    \end{example}
    We denote by $\mathcal D_X^\op$ the opposite sheaf of algebras of $\mathcal D_X$. The \textit{right} $\mathcal D_X$-modules can thus be identified with \textit{left} $\mathcal D_X^\op$-modules. Indeed, as $\w_{X/k}$ is a right $\mathcal D_X$-module, $\mathcal D_X^\op$ acts on $\w_{X/k}$ on the left. Moreover, from the relation between $P$ and ${}^tP$ in \cref{exp-canonical-bundle} and apply the isomorphism in \cref{cor-twist-end}, we see that
    \begin{lemma}
        There is an isomorphism 
        \[\mathcal D_X^\op\xrightarrow{\sim}\w_{X/k}\otimes_{\mathcal O_X}\mathcal D_X\otimes_{\mathcal O_X}\w_{X/k}^\vee\cong\mathcal D_X(\w_{X/k}, \w_{X/k})\]
        of $k$-algebras.
    \end{lemma}
    To continue our discussion of right $D$-modules, we must investigate how tensor product and Hom spaces interact with left and right $D$-modules, as they constitute the most essential tools one uses to pass from left to right $D$-modules. We denote by $\Mod(\mathcal D_X)$ and $\Mod(\mathcal D_X^\op)$ the category of left and right $D$-modules (assumed to be quasicoherent).
    \begin{proposition}
        For $\mathcal M$, $\mathcal N\in\Mod(\mathcal D_X)$ and $\mathcal M', \mathcal N'\in \Mod(\mathcal D_X^\op)$, and any $\xi\in\Theta_X$,
        \begin{enumerate}[\normalfont(i)]
            \item $\mathcal M\otimes\mathcal N\in\Mod(\mathcal D_X)$: $\xi(x\otimes y)=\xi(x)\otimes y+x\otimes\xi(y)$.
            \item $\mathcal M'\otimes\mathcal N\in\Mod(\mathcal D_X^\op)$: $(x'\otimes y)\xi=(x')\xi\otimes y-x'\otimes \xi(y)$.
            \item $\mathcal Hom(\mathcal M,\mathcal N)\in\Mod(\mathcal D_X)$: $(\xi\varphi)(x)=\xi(\varphi(x))-\varphi(\xi(x))$.
            \item $\mathcal Hom(\mathcal M',\mathcal N')\in\Mod(\mathcal D_X)$: $(\xi\varphi)(x)=-(\varphi(x))\xi+\varphi((x)\xi)$.
            \item $\mathcal Hom(\mathcal M,\mathcal N')\in\Mod(\mathcal D_X^\op)$: $(\varphi\xi)(x)=(\varphi(x))\xi+\varphi(\xi(x))$.
        \end{enumerate}
    \end{proposition}
    \begin{proof}
        I think the only case worth checking is probably (ii). Clearly it's compatible with the tensor product over $\mathcal O_X$ since $(x'f)\xi\otimes y=(x')(\xi\cdot f)\otimes y-x'\xi(f)\otimes y$. Passing the $\mathcal O_X$ coefficients to the second term and apply $[\xi, f]=\xi(f)$ again, we see that $(x'f\otimes y)\xi=(x'\otimes fy)\xi$. It remains to verify the relation $[\xi, f]=\xi(f)$ and $[\xi,\zeta]=\xi\zeta-\zeta\xi$ as actions on the tensor product. This is easy:
        \begin{align*}
            (x'\otimes y)(\xi\cdot f)&=((x')\xi\otimes y-x'\otimes\xi(y))f\\
            &=(x'f)\xi\otimes y+(x'\otimes y)\cdot\xi(f)-x'\otimes (f\cdot\xi)(y)\\
            &=(x'\otimes y)(f\cdot \xi)+(x'\otimes y)\xi(f)
        \end{align*}
        and similarly for the commutator $[\xi,\zeta]$.
    \end{proof}
    Now we know that $\w_{X/k}$ is a right $D$-module from \cref{exp-canonical-bundle}. So tensoring a left $D$-module on the left by $\w_{X/k}$ gives us a right $D$-module. Similarly, for any right $D$-module $\mathcal M$, $\w_{X/k}^\vee\otimes\mathcal M=\mathcal Hom(\w_{X/k},\mathcal M)$ is a left $D$-module! One thus obtains an equivalence of categories
    \begin{proposition}
        The functor $\w_{X/k}\otimes-:\Mod(\mathcal D_X)\to\Mod(\mathcal D_X^\op)$ is an equivalence of categories with a quasi-inverse given by $\w_{X/k}^\vee\otimes-$.
    \end{proposition}

    \subsection{Pushforwards and pullbacks}
    Like $\mathcal O_X$-modules, the pushforwards and pullbacks of $D$-modules do not automatically gain $D$-module structures. Thus it is an essential task to define the correct objects that carry the desired structures. Unlike $\mathcal O_X$-modules, where pullbacks require a bit more effort, pullbacks of $D$-modules can be easily defined. 
    
    Let $f:X\to Y$ be a morphism of smooth algebraic varieties over $k$. Suppose $\mathcal M$ is a left $D$-module on $Y$. We consider the pullback of $\mathcal M$ as an $\mathcal O_X$-module: $f^*\mathcal M=\mathcal O_X\otimes_{f^{-1}\mathcal O_Y}f^{-1}\mathcal M$. Recall that we have a homomorphism of $\mathcal O_X$-modules $df:\Theta_X\mapsto f^*\Theta_Y$: compose $f^\#:\mathcal O_X\to f_*\mathcal O_Y$ with $f_*d_{Y/k}:f_*\mathcal O_Y\to f_*\Omega_{Y/k}$ to get a derivation $\mathcal O_X\to f_*\Omega_{Y/k}$. By universal property of the cotangent sheaf, we get a map $\Omega_{X/k}\to f_*\Omega_{Y/k}$. By adjointness, there is a map $f^*\Omega_{X/k}\to\Omega_{Y/k}$ and taking its dual we get $df:\Theta_X\mapsto f^*\Theta_Y$. Now we may define a left $D$-module structure on $f^*\mathcal M$. For any $\xi\in\Theta_X$, write $df(\xi)=\sum g_j\otimes\eta_j$ for $g_j\in\mathcal O_X$ and $\eta_j\in\Theta_Y$. Then define, for each $x\in\mathcal O_X$ and $m\in\mathcal M$,
    \[\xi(x\otimes m)=\xi(x)\otimes m+x\sum_{j}g_j\otimes \eta_j(m)\]
    Now since $\mathcal D_Y$ is a left $D$-module on $Y$, the previous procedure gives us a left $D$-module on $X$.
    \begin{definition}
        The \textbf{transfer bimodule $X$ to $Y$} is $\mathcal D_{X\to Y}=f^*\mathcal D_Y$.
    \end{definition}
    An essential property of this module is that, as its name suggested, it's a $(\mathcal D_X, f^{-1}\mathcal D_Y)$-bimodule, where the left $\mathcal D_X$-module structure is defined in the previous paragraph and the right $f^{-1}\mathcal D_Y$-module structure is the obvious one. By the absorptions of tensor products, we get $f^*\mathcal M\cong \mathcal D_{X\to Y}\otimes_{f^{-1}\mathcal D_Y}f^{-1}\mathcal M$ as left $\mathcal D_X$-modules. 
    \begin{remark}
        I want to clarify that $\mathcal D_{X\to Y}$ is a \textit{right} $f^{-1}\mathcal D_Y$ but $f^{-1}\mathcal M$ is a \textit{left} $f^{-1}\mathcal D_Y$-module. The difference here made the tensor product possible --- since $\mathcal D_Y$ is not commutative, tensor products of two left modules make no sense, but tensoring one right module and one left module works.
    \end{remark}
    \begin{definition}
        Denote the pullback functor by $f^+=\mathcal D_{X\to Y}\otimes_{f^{-1}\mathcal D_Y}f^{-1}(\cdot):\Mod(\mathcal D_X)\to\Mod(\mathcal D_Y)$ as the restriction of $f^*$ to $\Mod(\mathcal D_X)$.
    \end{definition}
    Clearly, $f^+$ is right exact, so it's acceptable compared to standard pushforwards, which we will try to construct now. Given a right $D$-module $M$ on $X$, there is an obvious right $f^{-1}\mathcal D_Y$-module to define: $\mathcal M\otimes_{\mathcal D_X}\mathcal D_{X\to Y}$ (still, a \textit{right} $\mathcal D_X$-module tensoring a \textit{left} $\mathcal D_X$-module). Pushing it to $Y$, we get a right $\mathcal D_Y$-module $f_*(\mathcal M\otimes_{\mathcal D_X}\mathcal D_{X\to Y})$. But the functor $f_*(-\otimes_{\mathcal D_X}\mathcal D_{X\to Y})$ is not cool. In general, $f_*$ is left exact and $-\otimes_{\mathcal D_X}\mathcal D_{X\to Y}$ is right exact so a problem occurs whenever we want to do some homological algebra on $D$-modules.

    Given a left $D$-module $\mathcal M$ on $X$, we can first turn it into a right $D$-module on $X$ using the functor $\w_{X/k}\otimes-$. But wait: the right $D$-module on $X$ can be pushed to $Y$ via $f_*(-\otimes_{\mathcal D_X}\mathcal D_{X\to Y})$ defined above. So we get a right $D$-module on $Y$. But finally apply $\w_{Y/k}^\vee\otimes-$ to the right $D$-module, we get a left $D$-module on $Y$. Summarizing, we associated a left $\mathcal D_Y$-module
    \[f_+\mathcal M=\w_{Y/k}^\vee\otimes_{\mathcal O_Y}f_*(\w_{X/k}\otimes_{\mathcal O_X} \mathcal M\otimes_{\mathcal D_X}\mathcal D_{X\to Y})\]
    To simplify the notation
    \begin{align*}
        \w_{Y/k}^\vee&\otimes_{\mathcal O_Y}f_*((\w_{X/k}\otimes_{\mathcal O_X} \mathcal M)\otimes_{\mathcal D_X}\mathcal D_{X\to Y})\\
        &=\w_{Y/k}^\vee\otimes_{\mathcal O_Y}f_*((\w_{X/k}\otimes_{\mathcal O_X}\mathcal D_{X\to Y})\otimes_{\mathcal D_X} \mathcal M)\\
        &=f_*((\w_{X/k}\otimes_{\mathcal O_X}\mathcal D_{X\to Y}\otimes_{f^{-1}\mathcal O_Y}f^{-1}\w_{Y/k}^\vee)\otimes_{\mathcal D_X} \mathcal M)
    \end{align*}
    where the last equality follows from the projection formula. There is also a subtlety in the first equality where we used the fact that
    \[(\w_{X/k}\otimes_{\mathcal O_X} \mathcal M)\otimes_{\mathcal D_X}\mathcal D_{X\to Y}\cong (\w_{X/k}\otimes_{\mathcal O_X}\mathcal D_{X\to Y})\otimes_{\mathcal D_X} \mathcal M\]
    as right $f^{-1}\mathcal D_Y$-modules. The right $\mathcal D_X$-module structure of the bracket is given by the right module $\w_{X/k}$. Note here the action of $f^{-1}\mathcal D_Y$ on the right hand side is $((\w\otimes P)\otimes m)Q=(\w\otimes PQ)\otimes m$ for any $Q\in\mathcal D_Y$. So the map $(\w\otimes m)\otimes P\mapsto (\w\otimes P)\otimes m$ is an $f^{-1}\mathcal D_Y$-module isomorphism, as the $f^{-1}\mathcal D_Y$-module structures are given component-wise.
    \begin{definition}
        The \textbf{transfer bimodule $Y$ from $X$} is the $(f^{-1}\mathcal D_Y, \mathcal D_X)$-bimodule $\mathcal D_{Y\leftarrow X}=\w_{X/k}\otimes_{\mathcal O_X}\mathcal D_{X\to Y}\otimes_{f^{-1}\mathcal O_Y}f^{-1}\w_{Y/k}^\vee$.
    \end{definition}
    \begin{definition}
        For a left $D$-module $\mathcal M$ on $X$, we define its pushforward $f_+\mathcal M$ to be the left $\mathcal D_Y$-module $f_*(\mathcal D_{Y\leftarrow X}\otimes_{\mathcal D_X}\mathcal M)$.
    \end{definition}
    \begin{remark}\label{rmk-push-closed-imm}
        As we mentioned before, $f^+$ is cool, yet $f_+$ is not for being neither left nor right exact. But if $f:X\hookrightarrow Y$ is a \textit{closed immersion}, it is affine, which means $R^if_*=0$ for all $i>0$ (since the higher direct images are precisely cohomology groups on preimages of affines, which are affine by our assumption). Thus, $f_*$ is exact so $f_+$ is exact in this case, as $\mathcal D_{Y\leftarrow X}$ is locally free over $\mathcal D_X$.
    \end{remark}




    \subsection{Derived pushforwards and pullbacks}
    To resolve the problem of nonexactness, we introduce pullbacks and pushforwards of morphisms in derived categories so that no information is lost.
    \begin{lemma}
        Let $X$ be a topological space and $\mathcal R$ be a sheaf of rings on $X$. Then the category of left $\mathcal R$-modules has enough injectives, and for any \textit{right} $\mathcal R$-module $\mathcal M$ enough projectives with respect to $\mathcal M\otimes_{\mathcal R}-$ given by flat $\mathcal R$-modules.
    \end{lemma}
    \begin{proof}
        For enough injectives, see 2.4.3 in \cite{kashiwara-sheaves}. For enough $(\mathcal M\otimes_{\mathcal R}-)$-projectives, combine 2.4.12 and 2.4.13 in \cite{kashiwara-sheaves}.
    \end{proof}
    The lemma above allows us to define derived functors for certain functors in the category of $D$-modules.
    \begin{definition}
        For a sheaf of rings on a locally ringed space $X$, denote by $D^b(\mathcal R)$, $D^-(\mathcal R)$ and $D^+(\mathcal R)$ the bounded, right bounded and left bounded derived category of the category of \textit{not necessarily quasicoherent} $\mathcal R$-modules. On a smooth variety $X$, denote by $\smash{D_{qc}^\#(\mathcal D_X)}$ the full subcategory of $D^\#(\mathcal D_X)$ consisting of complexes with \textit{quasicoherent cohomology sheaves over $\mathcal O_X$}.
    \end{definition}
    Since we have enough $f^+=(\mathcal D_{X\to Y}\otimes-)$-projectives in the category of left $f^{-1}\mathcal D_Y$-modules, we can define the left derived functor of $f^+$ to be $\dL f^+:D^b(\mathcal D_Y)\to D^b(\mathcal D_X)$ given by
    \[\dL f^+\mathcal M^\bullet=\mathcal D_{X\to Y}\otimes_{f^{-1}\mathcal D_Y}^{\dL}\mathcal M^\bullet\]
    Just to clarify the notations in what follows: we may consider an object in some abelian category as the complex with the object concentrated in degree 0, which is an object in the derived category. 
    \begin{remark}
        I also need to address some subtlety here. Our definition of $D^b_{qc}(\mathcal D_X)$ is from \cite{htt-d-modules}, in which $\mathcal D_X$-modules are not forced to be quasicoherent over $\mathcal O_X$, although we did require $D$-modules to be quasicoherent in their definition. If we denote by $\Mod(\mathcal D_X)$ the category of (quasicoherent) left $D$-modules on $X$, $\Mod^*(\mathcal D_X)$ the category of all $\mathcal D_X$-modules, then there is an obvious inclusion $\Mod(\mathcal D_X)\to \Mod^*(\mathcal D_X)$. A complex $\mathcal M^\bullet$ of quasicoherent $\mathcal D_X$-modules clearly has quasicoherent cohomology sheaves. Therefore, the inclusion induces a map $D^b(\Mod(\mathcal D_X))\to D^b_{qc}(\mathcal D_X)$. Indeed, Bernstein showed it's an equivalence of category:
        \begin{theorem}[Bernstein]\label{thm-bernstein}
            Let $\mathcal R$ be a sheaf of algebras on a quasicompact separated scheme $X$. Denote by $\Mod(\mathcal R)$ the category of quasicoherent $\mathcal A$-modules. Then there is an equivalence of categories
            \[D^b(\Mod(\mathcal R))\xrightarrow{\sim}D^b_{qc}(\mathcal R)\]
        \end{theorem}
        \begin{proof}
            For a proof see IV.2.10 in \cite{borel-d-modules}.
        \end{proof}
        Furthermore, the equivalence holds for smaller categories of $D$-modules: coherent $D$-modules and holonomic $D$-modules (to be introduced). We can of course formulate the derived category with complexes of quasicoherent modules (c.f. \cite{ginzburg-notes} or \cite{bernstein-notes}). There is no essential difference between the two: we will always construct a quasicoherent resolution. I follow the latter because I find it more interesting. Besides, such a property of cohomologies could be very important, especially in the case of holonomic $D$-modules.
    \end{remark}
    \begin{definition}
        We define a shifted derived pullback functor $f^!$ by
        \[f^\dagger\mathcal M^\bullet=\dL f^+\mathcal M^\bullet[\dim X-\dim Y]\]
        This will be of interest in the next section.
    \end{definition}
    \begin{lemma}\label{lem-pullback-natural}
        Given $f:X\to Y$ and $g:Y\to Z$, then
        \[\dL (g\circ f)^+\simeq \dL f^+\circ\dL g^+,\quad (g\circ f)^\dagger\simeq f^\dagger\circ g^\dagger\]
    \end{lemma}
    \begin{proof}
        It suffices to show that $\mathcal D_{X\to Z}$ is isomorphic to $\mathcal D_{X\to Y}\otimes^{\dL}_{f^{-1}\mathcal D_Y}f^{-1}\mathcal D_{Y\to Z}$, in which case we have
        \begin{align*}
            \dL (g\circ f)^+\mathcal M^\bullet
            &=\mathcal D_{X\to Z}\otimes_{(g\circ f)^{-1}\mathcal D_Y}^\dL (g\circ f)^{-1}\mathcal M^\bullet\\
            &\cong \mathcal D_{X\to Y}\otimes^{\dL}_{f^{-1}\mathcal D_Y}f^{-1}\mathcal D_{Y\to Z}\otimes_{(g\circ f)^{-1}\mathcal D_Y}^\dL (g\circ f)^{-1}\mathcal M^\bullet\\
            &= \mathcal D_{X\to Y}\otimes^{\dL}_{f^{-1}\mathcal D_Y}f^{-1}(\mathcal D_{Y\to Z}\otimes_{g^{-1}\mathcal D_Y}^\dL g^{-1}\mathcal M^\bullet)\\
            &=(\dL f^+\circ\dL g^+)(\mathcal M^\bullet)
        \end{align*}
        and the second equality follows from the fact $[\dim Y-\dim Z]\circ [\dim X-\dim Y]=[\dim X-\dim Z]$. The claim at the beginning is merely a derived version of the simple fact $\mathcal D_{X\to Z}\cong\mathcal D_{X\to Y}\otimes_{f^{-1}\mathcal D_Y}f^{-1}\mathcal D_{Y\to Z}$ by functoriality of $\mathcal O_X$-module pullbacks. Since $\mathcal D_Z$ and $\mathcal D_Y$ are locally free $\mathcal O_Z$- and $\mathcal O_Y$-modules resp. (though of infinite rank), the tensor product is exact so the derived version is trivially true.
    \end{proof}
    \begin{lemma}
        The functor $\dL f^+$ sends $D^b_{qc}(\mathcal D_Y)$ to $D^b_{qc}(\mathcal D_X)$.
    \end{lemma}
    \begin{proof}
        An affine local argument works since $\dL f^+$ commutes with pullbacks by inclusions of affines. But the proof would require a special case of \cref{thm-bernstein}, which we did not prove. So I will refer to (1) in 08DW of \cite{stacks-project}.
    \end{proof}
    \begin{example}
        When a morphism $f:X\to Y$ is smooth, its pullback $f^*$ is exact for quasicoherent $\mathcal O_Y$-modules since $f$ is flat (IV.2.1.3 in \cite{ega}). Therefore we have $H^i\dL f^+\mathcal M^\bullet=0$ for all $i>0$ where $\mathcal M$ is a $D$-module on $Y$. We can drop the derived functor notation in this case.
    \end{example}
    \begin{example}
        Given an open immersion $j:X\hookrightarrow Y$, $j^*$ is exact. By definition $\mathcal O_X=j^{-1}\mathcal O_Y$ and thus $j^{-1}\mathcal D_Y=\mathcal D_X$. Therefore,
        \[\mathcal D_{X\to Y}\cong\mathcal O_X\otimes_{\mathcal O_X}\mathcal D_X=\mathcal D_X\]
        in which case $j^\dagger=\dL j^+=j^{-1}$ as $\dim X=\dim Y$.
    \end{example}

    Next we deal with the task of defining pushforwards. We do this in two ways. Given a general morphism $f:X\to Y$ of smooth algebraic varieties over $k$, we define
    \begin{definition}
        The derived pushforward $f_+$ of $f$ is $\dR f_*(\mathcal D_{Y\leftarrow X}\otimes^{\dL}_{\mathcal D_X}-):D^b(\mathcal D_X)\to D^b(\mathcal D_Y)$ where we first use a flat resolution to derive the tensor product and then an injective resolution for the pushforward.
    \end{definition}
    \begin{example}\label{exp-open-immersion-pushforward}
        Given an open immersion $j:X\hookrightarrow Y$, we've seen $\mathcal D_{Y\leftarrow X}=\mathcal D_X$ in a previous example, so $j_+=\dR j_*(\mathcal D_{Y\leftarrow X}\otimes^{\dL}_{\mathcal D_X}-)=\dR j_*(\mathcal D_{X}\otimes^{\dL}_{\mathcal D_X}-)=\dR j_*$.
    \end{example}
    \begin{lemma}
        Let $f:X\to Y$ and $g:Y\to Z$ be morphisms of smooth varieties. Then we have
        \[(g\circ f)_+\simeq g_+\circ f_+:D^b(\mathcal D_X)\to D^b(\mathcal D_Z)\]
    \end{lemma}
    \begin{proof}
        We have
        \[g_+(f_+\mathcal M^\bullet)=\dR g_*(\mathcal D_{Z\leftarrow Y}\otimes^{\dL}_{\mathcal D_Y}\dR f_*(\mathcal D_{Y\leftarrow X}\otimes^{\dL}_{\mathcal D_X}\mathcal M^\bullet))\]
        and
        \[(g\circ f)_+\mathcal M^\bullet=\dR(g\circ f)_*(\mathcal D_{Z\leftarrow X}\otimes^{\dL}_{\mathcal D_X}\mathcal M^\bullet)=\dR g_*(\dR f_*(\mathcal D_{Z\leftarrow X}\otimes^{\dL}_{\mathcal D_X}\mathcal M^\bullet))\]
        Similar to \cref{lem-pullback-natural}, it is easy to verify
        \[\mathcal D_{Z\leftarrow X}\cong f^{-1}\mathcal D_{Z\leftarrow Y}\otimes^{\dL}_{f^{-1}\mathcal D_Y}\mathcal D_{Y\leftarrow X}\]
        Therefore, it remains to show
        \[\mathcal D_{Z\leftarrow Y}\otimes^{\dL}_{\mathcal D_Y}\dR f_*(\mathcal D_{Y\leftarrow X}\otimes^{\dL}_{\mathcal D_X}\mathcal M^\bullet)\cong \dR f_*(f^{-1}\mathcal D_{Z\leftarrow Y}\otimes^{\dL}_{f^{-1}\mathcal D_Y}\mathcal D_{Y\leftarrow X}\otimes_{\mathcal D_X}^\dL \mathcal M^\bullet)\]
        which follows from the projection formula as $\mathcal D_{Z\leftarrow Y}$ is a locally free $\mathcal D_Y$-module.
    \end{proof}
    For any morphism $f:X\to Y$, we should expect the image of $D^b_{qc}(\mathcal D_X)$ under $f_+$ to sit inside $D^b_{qc}(\mathcal D_Y)$. But this is no longer easy to show. Instead we will have to reformulate the definition of $f_+$. We can factor it using a projection and a closed immersion, i.e., we take $\iota:X\hookrightarrow Z=X\times Y$ to be the map $\id\times f$ and $p:Z\to Y$ the projection onto $Y$, meaning $f=p\circ \iota$. We know the \textit{standard} pushforward via $\iota$ can be defined easily as $\iota_*(\mathcal D_{Y\leftarrow X}\otimes_{\mathcal D_X}\mathcal M)$, which is quasicoherent. So it remains to compute and study the pushforward via $p$.
    \begin{definition}
        For any smooth algebraic variety $X$ of dimension $n$, the \textbf{Spencer complex} $\Sp(\mathcal O_X)$ of $\mathcal O_X$ is the complex
        \[0\to \mathcal D_X\otimes_{\mathcal O_X}\bigwedge^n\Theta_X\to\cdots\to\mathcal D_X\otimes_{\mathcal O_X}\Theta_X\to \mathcal D_X\to\mathcal O_X\to 0\]
        The differentials $d:\mathcal D_X\otimes_{\mathcal O_X}\bigwedge^k\Theta_X\to \mathcal D_X\otimes_{\mathcal O_X}\bigwedge^{k-1}\Theta_X$ are given by
        \begin{align*}
            d(P\otimes \xi_1\wedge\cdots\wedge\xi_k)=&\sum_i(-1)^{i+1}P\xi_i\otimes\xi_1\wedge\cdots\wedge\hat{\xi_i}\wedge\cdots\wedge\xi_k\\
            &+\sum_{i<j}(-1)^{i+j}P\otimes[\xi_i,\xi_j]\wedge\xi_1\wedge\cdots\wedge\hat{\xi_i}\wedge\cdots\wedge\hat{\xi_j}\wedge\cdots\wedge\xi_k
        \end{align*}
        and the last map $\mathcal D_X\to\mathcal O_X$ is given by $P\mapsto P(1)$.
    \end{definition}
    \begin{lemma}
        The Spencer complex of $\mathcal O_X$ is a locally free resolution of the left $\mathcal D_X$-module $\mathcal O_X$.
    \end{lemma}
    \begin{proof}
        The Spencer complex $\Sp(\mathcal O_X)$ is filtered by subcomplexes
        \[F_p\Sp(\mathcal O_X):F_{p-n}\mathcal D_X\otimes_{\mathcal O_X}\bigwedge^n\Theta_X\to\cdots\to F_{p-1}\mathcal D_X\otimes_{\mathcal O_X}\Theta_X\to F_{p}\mathcal D_X\to\mathcal O_X\]
        As the filtration is bounded below and exhaustive, the bounded homology spectral sequence 
        \[E_{p,q}^1=H_{p+q}(F_p\Sp(\mathcal O_X)/F_{p-1}\Sp(\mathcal O_X))\implies H_{p+q}(\Sp(\mathcal O_X))\] where the convergence is given by the classical convergence theorem. Thus, to show that $\Sp(\mathcal O_X)$ is a resolution which in our case is equivalent to being acyclic, it suffices to prove that each $F_p\Sp(\mathcal O_X)/F_{p-1}\Sp(\mathcal O_X)$ is acyclic. But observe we have $\gr^F\Sp(\mathcal O_X)$ being the direct sum of these components, we can instead try to prove that $\gr^F\Sp(\mathcal O_X)$ is acyclic since direct sums are exact. Since $\Theta_X$ is locally free, we have
        \[\gr^F\Sp(\mathcal O_X):\gr^F\mathcal D_X\otimes_{\mathcal O_X}\bigwedge^n\Theta_X\to\cdots\to \gr^F D_X\otimes_{\mathcal O_X}\Theta_X\to \gr^F\mathcal D_X\to\mathcal O_X\]
        This is a Koszul resolution of the regular sequence $\partial_1,\dots,\partial_n$. By the vanishing theorem, the complex is acyclic and $\Sp(\mathcal O_X)$ is a resolution of $\mathcal O_X$ with each term being locally free over $\mathcal D_X$.
    \end{proof}
    Tensoring $\Sp(\mathcal O_X)$ by the locally free canonical sheaf $\w_{X/k}$, we get a locally free resolution
    \[0\to\mathcal D_X\to\cdots\to\Omega_{X/k}^{n-1}\otimes_{\mathcal O_X}\mathcal D_X\to \Omega_{X/k}^n\otimes_{\mathcal O_X}\mathcal D_X\to\w_{X/k}\to 0\]
    of the right $\mathcal D_X$-module $\w_{X/k}$. Back to our projections $p:Z\to Y$ and $q:Z\to X$. One sees that
    \[\mathcal D_{Y\leftarrow Z}=\mathcal D_Y\boxtimes \w_{X/k}=p^*\mathcal D_Y\otimes_{\mathcal O_Z}q^*\w_{X/k}\]
    Indeed, algebraically we have $\Omega_{B\otimes_A C/A}=(\Omega_{B/A}\otimes_A C)\oplus (\Omega_{C/A}\otimes_A B)$. Therefore there is an exact sequence of locally free sheaves of finite rank on $Z$
    \[0\to p^*\Omega_{Y/k}\to \Omega_{Z/k}\to q^*\Omega_{X/k}\to 0\]
    Then taking the determinants, we get $\w_{Z/k}=p^*\w_{Y/k}\otimes_{\mathcal O_Z}q^*\w_{X/k}$. The decomposition of $\mathcal D_{Y\leftarrow Z}$ then follows from its definition. 
    \begin{definition}
        Let $n=\dim X=\dim Z-\dim Y$. For any left $D$-module $\mathcal M$ on $Z$, the \textbf{relative de Rham complex} $\mathcal A_{Z/Y}^\bullet$ associated to $\mathcal M$ is given by
        \[\mathcal A_{Z/Y}^k(\mathcal M)=\Omega_{Z/Y}^{n+k}\otimes_{\mathcal O_Z}\mathcal M\]
        for $k=-n,\dots,0$ where $\Omega_{Z/Y}^{n+k}=\mathcal O_Y\boxtimes\Omega^{n+k}_{X}$. The differentials are given by $d(\w\otimes s)=d\w\otimes s+\sum_{j=1}^n (dx_j\wedge \w)\otimes\partial_j s$ for a coordinate system $\{x_j,\partial_j\}$ of $X$ pulled back to $Z$.
    \end{definition}
    \begin{remark}
        Observe that $\mathcal A_{Z/Y}^k(q^*\mathcal D_X)$ is simply the pullback of the Spencer resolution of $\w_{X/k}$ to $Z$, i.e., a locally free resolution of $q^*\w_{X/k}$. One can therefore see that the relative de Rham complex is induced by the Spencer resolution, i.e., a free resolution of $\mathcal D_{Y\leftarrow Z}\otimes_{\mathcal D_X}\mathcal M$, meaning $\mathcal D_{Y\leftarrow Z}\otimes_{\mathcal D_Z}^\dL \mathcal M\cong\mathcal A_{Z/Y}^\bullet(\mathcal M)$ as complexes of $p^{-1}\mathcal D_Y$-modules, where the right hand side is a locally free complex.
    \end{remark}
    Summarizing our discussion:
    \begin{proposition}\label{prop-dpush-qc}
        Let $X, Y$ be smooth varieties, $Z=X\times Y$, $q, p$ projections to the first, second factor resp. Then for any left $\mathcal D_Z$-module $\mathcal M$, we have $p_+\mathcal M\cong \dR p_*(\mathcal A_{Z/Y}^\bullet(\mathcal M))$. Moreover, the image of $D^b_{qc}(\mathcal D_Z)$ under $p_+$ sits inside $D^b_{qc}(\mathcal D_Y)$.
    \end{proposition}
    \begin{proof}
        We know $\mathcal A_{Z/Y}^\bullet(\mathcal M)$ is quasicoherent, so the claim follows from the general fact (1) in 08D5 of \cite{stacks-project}.
    \end{proof}
    Given $f_+=p_+\circ\iota_+$ where $\iota=\id\times f:X\hookrightarrow Z=X\times Y$ and $p:Z=X\times Y\to Y$, the final task is to check if $\iota_+$ preserves quasicoherent cohomologies. But this is always the case, even for more interesting $D$-modules, due to the following lemma
    \begin{lemma}
        If $\iota:X\hookrightarrow Y$ is a closed immersion, we have
        \[H^k(\iota_+\mathcal M^\bullet)\cong\iota_+(H^k\mathcal M^\bullet)\]
        for all $k$ and complexes $\mathcal M^\bullet$ of $\mathcal D_X$-modules on $X$.
    \end{lemma}
    \begin{proof}
        Recall in \cref{rmk-push-closed-imm}, we observed the standard pushforward via $\iota$ is exact (both $\iota_*$ and $(\mathcal D_{Y\leftarrow X}\otimes -)$ are exact). Therefore, $\iota_+\mathcal M^\bullet =\iota_*(\mathcal D_{Y\leftarrow X}\otimes_{\mathcal D_X}\mathcal M^\bullet)$. The claim is then apparent from exactness.
    \end{proof}
    The image of $D^b_{qc}(\mathcal D_X)$ under $\iota_+$ therefore lies inside $D^b_{qc}(\mathcal D_Z)$. Since $f_+=p_+\circ \iota_+$, combining \cref{prop-dpush-qc}, we finally obtain
    \begin{proposition}\label{prop-pushforward-quasicoherent}
        Let $f:X\to Y$ be an arbitrary morphism of smooth algebraic varieties. Then $f_+$ sends $D^b_{qc}(\mathcal D_X)$ to $D^b_{qc}(\mathcal D_Y)$.
    \end{proposition}
    %%% hht: X = Y x Z
    %%% mine:   Z = X x Y
    %%% ginzburg: Y = Z x X --> X

    \subsection{Closed immersions and Kashiwara's equivalence}
    Suppose $\iota:X\hookrightarrow Y$ is a closed immersion of smooth algebraic varieties over $k$. As we mentioned, their pullbacks and pushforwards have nice properties. In this section, I will state and prove the first interesting result in this essay: Kashiwara's equivalence on closed immersions. We have defined the (standard) pullback of $D$-modules on $Y$ in a canonical way, but it is also possible to define another pullback $\iota^!:\Mod(\mathcal D_Y)\to\Mod(\mathcal D_X)$ given by
    \[\iota^!\mathcal M=\mathcal Hom_{\iota^{-1}\mathcal D_Y}(\mathcal D_{Y\leftarrow X}, \iota^{-1}\mathcal M)\]
    We use the left $\iota^{-1}\mathcal D_Y$ structure of the transfer bimodule and $\iota^{-1}\mathcal M$ to get the sheaf hom, and the left $\mathcal D_X$-module structure of the sheaf hom is given by the right $\mathcal D_X$-module structure of the transfer bimodule via precomposing morphisms with actions of $\mathcal D_X$ on the right. This pullback is not completely unreasonable. In fact, it is closely related to the shifted derived pullback $\iota^\dagger\mathcal M^\bullet=\dL\iota^+\mathcal M^\bullet[\dim X-\dim Y]$:
    \begin{lemma}
        The derived pullback $\iota^\dagger$ is the derived functor of $\iota^!$, that is, $\iota^\dagger\mathcal M^\bullet\cong \dR\iota^!\mathcal M^\bullet$.
    \end{lemma}
    \begin{proof}
        This lemma is never used in this essay. For a proof see 1.5.16 in \cite{htt-d-modules}.
    \end{proof}
    To understand the nature of $\iota_+$ and $\iota^+$, we need to write down their local descriptions. Since $\iota:X\hookrightarrow Y$ is a closed immersion of smooth varieties, for any $x\in X$, there is an affine open $U$ of $Y$ with local coordinates $\{y_i,\partial_{y_i}\}_{i=1,\dots, n}$ such that $x\in U$ and $\iota(X)\cap U$ is defined by equations $y_{r+1}=\cdots=y_n=0$ for some $r$. Write $x_i=y_i\circ\iota$ for $i=1,\dots, r$. The preimage $V=\iota^{-1}(U)$ has a local coordinate $\{x_i,\partial_{i}\}$ (see 0H1G in \cite{stacks-project}). The canonical morphism $\Theta_X\to\iota^*\Theta_Y=\mathcal O_X\otimes_{\iota^{-1}\mathcal O_Y}\Theta_Y$ is given by $\partial_i\mapsto1\otimes\partial_{y_i}$ in the exact sequence 
    \[0\to\Theta_X\to\iota^*\Theta_Y\to\mathcal N_{X/Y}=(\mathcal I_X/\mathcal I_X^2)^\vee\to 0\] 
    where $\mathcal I_X$ is the ideal sheaf of $\iota$.
    \begin{example}
        Consider the subring $\mathcal D_Y^X$ of $\mathcal D_Y$ given by $\bigoplus_{\alpha}\mathcal O_Y\partial_{y_1}^{\alpha_1}\cdots\partial_{y_r}^{\alpha_r}$. It is clear that $\mathcal D_Y^X\otimes_k k[\partial_{y_{r+1}},\dots,\partial_{y_n}]\cong \mathcal D_Y$ via $Q\otimes P\mapsto QP$ from their local descriptions. But then $\mathcal D_{X\to Y}=\iota^*\mathcal D_Y$ which is just $\iota^*\mathcal D_Y^X\otimes_k k[\partial_{y_{r+1}},\dots,\partial_{y_n}]$. Here $\iota^*\mathcal D_Y^X$ is precisely the image of $\mathcal D_X$ under the injective map $\Theta_X\to\iota^*\Theta_Y$ in $\mathcal D_{X\to Y}$, so $\mathcal D_X\cong \iota^*\mathcal D_Y^X$. Hence, affine locally, $\mathcal D_{X\to Y}\cong\mathcal D_X\otimes_k k[\partial_{y_{r+1}},\dots,\partial_{y_n}]$, a locally free $\mathcal D_X$-module of infinite rank. In other words, $\mathcal D_{X\to Y}=\mathcal D_Y/\mathcal I_X\mathcal D_Y$.
    \end{example}
    \begin{example}\label{exp-push-local}
        Similarly we compute $\mathcal D_{Y\leftarrow X}$ affine locally. By definition,
        \[\mathcal D_{Y\leftarrow X}=\iota^{-1}\mathcal D_Y\otimes_{\iota^{-1}\mathcal O_Y}\mathcal O_X\otimes_{\mathcal O_X}(\iota^{-1}\w_{Y/k}^\vee\otimes_{\iota^{-1}\mathcal O_Y}\w_{X/k})\]
        We've shown that $\mathcal D_Y^X\otimes_k k[\partial_{y_{r+1}},\dots,\partial_{y_n}]\cong \mathcal D_Y$, so
        \[\mathcal D_{Y\leftarrow X}\cong k[\partial_{y_{r+1}},\dots,\partial_{y_n}]\otimes_k(\iota^{-1}\mathcal D_Y^X\otimes_{\iota^{-1}\mathcal O_Y}\mathcal O_X)\cong k[\partial_{y_{r+1}},\dots,\partial_{y_n}]\otimes_k\mathcal D_X\]
        where we identified $\smash{\iota^{-1}\w_{Y/k}^\vee\otimes_{\iota^{-1}\mathcal O_Y}\w_{X/k}}$ with $\mathcal O_X$ in the first congruence (we are working closed immersions on affine opens so this is true). The $\iota^{-1}\mathcal D_Y$-module structure on this new expression is not obvious. Observe that $\iota^{-1}\mathcal D_Y=k[\{\partial_{y_i}\}]\otimes_k\iota^{-1}\mathcal D_Y^X$. It suffices to determine the action of each component on the module. The first term $k[\{\partial_{y_i}\}]$ clearly acts on $\mathcal D_{Y\leftarrow X}$ by multiplications on the first component. Now, for any $Q\in\iota^{-1}\mathcal D_Y^X$, write $QS=\sum S_kQ_k$ for $Q_k\in\iota^{-1}\mathcal D_Y^X$ and $S, S_k\in k[\{\partial_{y_i}\}]$. Then for any $P$ of $\mathcal D_X$, we have $Q(S\otimes P)=\sum S_k\otimes Q_k P$, where each $Q_k\in\iota^{-1}\mathcal D_Y^X$ acts on $\mathcal D_X\cong\iota^{-1}\mathcal D_Y^X\otimes_{\iota^{-1}\mathcal O_Y}\mathcal O_X$ in the obvious way. We see that since $\iota^{-1}\mathcal D_Y^X$ is the image of $\mathcal D_X$ in $\iota^{-1}\mathcal D_Y$, the transfer bimodule $\mathcal D_{Y\leftarrow X}$ is generated as an $\iota^{-1}\mathcal D_Y$-module by $1\otimes 1$.
    \end{example}
    \begin{example}\label{exp-push-module-local}
        It is immediate that $\iota_+\mathcal M$ is locally isomorphic to $k[\partial_{y_{r+1}},\dots,\partial_{y_n}]\otimes_k\iota_*\mathcal M$ for any left $\mathcal D_X$-module $\mathcal M$. The pushforward and inverse image preserve the polynomial ring via an easy argument on sections. Intuitively, $\iota_+$ provides actions of $\mathcal D_Y$ on $\iota_*\mathcal M$ by adding copies of $\partial_{y_{r+1}},\dots,\partial_{y_n}$ besides the already existing $\partial_i$ on $X$.
    \end{example}
    More importantly, we can directly relate $\iota^!$ to $\iota_+$.
    \begin{proposition}\label{prop-adjunction}
        The functors $\iota_+$ and $\iota^!$ are adjoints between the category of all $\mathcal D_X$-modules and the category of $\mathcal D_Y$-modules, i.e., for any left $\mathcal D_X$-module $\mathcal M$ and left $\mathcal D_Y$-module $\mathcal N$,
        \[\Hom_{\mathcal D_Y}(\iota_+\mathcal M,\mathcal N)\cong\Hom_{\mathcal D_X}(\mathcal M, \iota^!\mathcal N)\]
    \end{proposition}
    \begin{proof}
        The idea is the adjunction between $\mathcal Hom$ and tensor products, i.e., we have an isomorphism of sheaves of abelian groups
        \begin{equation}\label{eq-hom-hom-tensor}
            \mathcal Hom_{\mathcal D_X}(\mathcal M,\mathcal Hom_{\iota^{-1}\mathcal D_Y}(\mathcal D_{Y\leftarrow X},\iota^{-1}N))\xrightarrow{\sim}\mathcal Hom_{\iota^{-1}\mathcal D_Y}(\mathcal D_{Y\leftarrow X}\otimes_{\mathcal D_X}\mathcal M, \iota^{-1}\mathcal N)
        \end{equation}
        given by $\varphi\mapsto (P\otimes s\mapsto \varphi(s)(P))$ with inverse $\psi\mapsto(s\mapsto (P\mapsto \psi(P\otimes s)))$. Since $\iota$ is an immersion, the canonical map $\iota^{-1}\iota_*$ is an isomorphism of sheaves. Taking the global sections of both sides of \cref{eq-hom-hom-tensor}, one arrives at
        \begin{align}\label{eq-hom-hom-tensor-hom-tensor}
            \Hom_{\mathcal D_X}(\mathcal M,\iota^!\mathcal N)&\cong\Hom_{\iota^{-1}\mathcal D_Y}(\iota^{-1}\iota_*(\mathcal D_{Y\leftarrow X}\otimes_{\mathcal D_X}\mathcal M), \iota^{-1}\mathcal N)\\
            &\cong\Hom_{\mathcal D_Y}(\iota_*(\mathcal D_{Y\leftarrow X}\otimes_{\mathcal D_X}\mathcal M),\iota_*\iota^{-1}\mathcal N)
        \end{align}
        where the second congruence follows from general adjunction results such as 008Y in \cite{stacks-project}.
        
        It remains to show the last Hom space is isomorphic to the desired result. But this is not immediate, since in general $\iota_*\iota^{-1}\mathcal N$ is not isomorphic to $\mathcal N$. But a stalk-wise argument shows that if $\mathcal N$ is supported on $X$, then $\iota_*\iota^{-1}\mathcal N\cong\mathcal N$. So our aim is to replace $\mathcal N$ with a sheaf supported on $X$. Let $\mathcal N^X$ be the subsheaf $V\mapsto \mathcal N^X(V)$ with the latter being sections in $\Gamma(\mathcal N, V)$ supported on $X\subseteq Y$. To achieve the goal, we claim that for any $\mathcal D_X$-module $\mathcal M'$ and $\mathcal D_Y$-module $\mathcal N'$
        \begin{align*}
            \mathcal Hom_{\mathcal D_Y}(\iota_*\mathcal M',\mathcal N')&\cong\mathcal Hom_{\mathcal D_Y}(\iota_*\mathcal M',\mathcal N'^X)\\
            \mathcal Hom_{\iota^{-1}\mathcal D_Y}(\mathcal D_{Y\leftarrow X},\iota^{-1}\mathcal N')&\cong \mathcal Hom_{\iota^{-1}\mathcal D_Y}(\mathcal D_{Y\leftarrow X},\iota^{-1}(\mathcal N')^{X})
        \end{align*}
        so that every $\mathcal N$ in \cref{eq-hom-hom-tensor} and \cref{eq-hom-hom-tensor-hom-tensor} can be replaced by $\mathcal N^X$, in which case we are done. The first congruence is immediate: all sections of $\iota_*\mathcal M'$ vanish on small enough opens around points outside $X$. To wit the second, it suffices to show $\psi(s)\in\iota^{-1}(\mathcal N')^{X}$ for any $\psi\in\mathcal Hom_{\iota^{-1}\mathcal D_Y}(\mathcal D_{Y\leftarrow X},\iota^{-1}\mathcal N')$ and $s\in\mathcal D_{Y\leftarrow X}$. The question is of a local nature, so we can write $\mathcal D_{Y\leftarrow X}\cong k[\partial_{y_{r+1}},\dots,\partial_{y_n}]\otimes_k\mathcal D_X$ by \cref{exp-push-local}. As an $\iota^{-1}\mathcal D_Y$-module, the right hand side is generated by $1\otimes 1$. Let $\mathcal I_X\subseteq\mathcal O_Y$ be the ideal sheaf of $X$. Then a section of $\iota^{-1}\mathcal N'$ killed by $\iota^{-1}\mathcal I_X$ is a section of $\iota^{-1}\mathcal (N')^X$ since a section of $\mathcal N$ killed by $\mathcal I_X$ is supported on $X$. Now $\iota^{-1}\mathcal I_X(1\otimes 1)$ is zero by construction (use the expression in \cref{exp-push-local} and the fact that $\mathcal I_X$ annihilates $\mathcal D_X$), and since $\psi$ is $\iota^{-1}\mathcal D_Y$ linear, $\psi(s)$ sits inside $\iota^{-1}(\mathcal N')^{X}$.

        In conclusion, we have
        \begin{align*}
            \Hom_{\mathcal D_X}(\mathcal M,\iota^!\mathcal N)&\cong\Hom_{\mathcal D_Y}(\iota_*(\mathcal D_{Y\leftarrow X}\otimes_{\mathcal D_X}\mathcal M),\iota_*\iota^{-1}\mathcal N^X)\\
            &\cong\Hom_{\mathcal D_Y}(\iota_*(\mathcal D_{Y\leftarrow X}\otimes_{\mathcal D_X}\mathcal M),\mathcal N^X)\\
            &=\Hom_{\mathcal D_Y}(\iota_*(\mathcal D_{Y\leftarrow X}\otimes_{\mathcal D_X}\mathcal M),\mathcal N)=\Hom_{\mathcal D_Y}(\iota_+\mathcal M,\mathcal N)
        \end{align*}
        completing the proof.
    \end{proof}
    \begin{remark}\label{rmk-support-d-module}
        I have to make the annihilator argument clear here because I personally struggled with the proof above for a while. First, define $\mathcal M_X$ to be the submodules of section $m$ such that for any $f\in\mathcal I_X$, there is some power of $f$ that kills $m$. We claim that $\mathcal M$ is supported on $X$ if and only if $\mathcal M_X=\mathcal M$. Note that this claim is local, and as $X$ is Noetherian $\mathcal I_X$ is coherent. Suppose $y\notin X$, it can be checked affine locally that there is some $f\in\mathcal I_X$ invertible in $\mathcal O_{Y, y}$, meaning all sections of $\mathcal M_X$ vanish in the stalk at $y$. Thus, $\mathcal M_X$ is supported on $X$. Conversely, we show that every $\mathcal D_Y$-submodule of $\mathcal M$ supported on $X$ is contained in $\mathcal M_X$, i.e., $\mathcal M_X$ is the maximal submodule of $\mathcal M$ supported on $X$. Say $m$ is a section of one such submodule. Let $\mathcal N$ be the $\mathcal O_X$-submodule generated by $m$. Then since $\mathcal N$ is finite, by the annihilator-support theorem, the support of $\mathcal N$ is cut out by its annihilator $\mathcal J\subseteq\mathcal O_X$. Then Hilbert's Nullstellensatz gives us $\mathcal I_X\subseteq\sqrt{\mathcal J}$, completing the proof. In the proof of the previous proposition, we used the fact that if $\mathcal I_X$ annihilates a section then the section has support in $X$.
    \end{remark}
    \begin{remark}\label{exp-ideal-kills-pullback}
        In the proof of \cref{prop-adjunction}, we actually showed more. We proved that for any $\psi\in\mathcal Hom_{\iota^{-1}\mathcal D_Y}(\mathcal D_{Y\leftarrow X},\iota^{-1}\mathcal N')$ and $s\in\mathcal D_{Y\leftarrow X}$, $\varphi(s)$ is annihilated by $\iota^{-1}\mathcal I_X$. Thus it's easy to identify $\iota^!\mathcal N$ with the subsheaf of $\iota^{-1}\mathcal N$ annihilated by $\iota^{-1}\mathcal I_X$. Recall in \cref{exp-push-module-local} we commented that $\iota_+$ ``added copies of $\partial_{y_i}$'' to $\iota_*\mathcal M$. Note that $\iota_*\mathcal M$ as usual is annihilated by $\mathcal I_X$. Therefore intuitively, the pullback $\iota^!$ recovers the part annihilated by $\mathcal I_X$ in a $\mathcal D_Y$-module.
    \end{remark}
    As usual, the category of quasicoherent $\mathcal D_X$-modules I denote by $\Mod(\mathcal D_X)$. In addition, denote by $\Mod^X(\mathcal D_Y)$ the full subcategory of quasicoherent $\mathcal D_Y$-modules supported on $X\subseteq Y$ in $\Mod(\mathcal D_Y)$. Then
    \begin{theorem}[Kashiwara's equivalence]
        The standard pushforward functor $\iota_+$ induces an equivalence of categories $\Mod(\mathcal D_X)\xrightarrow{\sim}\Mod^X(\mathcal D_Y)$ with a quasi-inverse given by $\iota^!$.
    \end{theorem}
    \begin{proof}
        Since $\iota_+$ and $\iota^!$ are adjoints by \cref{prop-adjunction}, we have to show the canonical unit $\id\to\iota^!\iota_+$ and counit $\iota_+\iota^!\to\id$ are isomorphisms. The problem can be checked affine locally so we may assume $Y=\Spec R$, $X=\Spec R/I$ where $R$ is a $k$-algebra with generators $y_1,\dots, y_n$ and $I=(y_{r+1},\dots, y_n)$ for some $r$. Let $X_i$ be the closed subscheme cut out by the last $i$ elements in the ideal. By an induction on $i$, it suffices to prove the theorem for $i=1$, i.e., $X=\Spec R/(y)$ for some $y$. The local coordinates in \cref{exp-push-local} is then $\{y, \partial\}$.

        The isomorphism $\mathcal M\to\iota^!\iota_+\mathcal M$ is not hard to see. Note that $\iota_*\mathcal M$ is killed by $y$. We first claim that $\iota_+\mathcal M=k[\partial]\otimes_k\iota_*\mathcal M$ is supported on $X$. Indeed, for any $m\in\iota_*\mathcal M$, we have
        \[y(\partial\otimes m)=-1\otimes m+\partial\otimes ym=-1\otimes m\]
        using the expression given in \cref{exp-push-local} and $[\partial, y]=1$. Thus, $y^{p+1}(\partial^p\otimes m)=0$ for any $m\in\iota_*\mathcal M$. Therefore, by \cref{rmk-support-d-module}, $\iota_+\mathcal M$ is supported on $X$. Notice that $\iota^!\iota_+\mathcal M$ is isomorphic to the subsheaf of $\iota^{-1}\iota_+\mathcal M$ annihilated by $y$. The reader has in fact seen this in the proof of \cref{prop-adjunction} and \cref{exp-ideal-kills-pullback}! Therefore it remains to compute the subsheaf of $\iota^{-1}\iota_+\mathcal M$ killed by $\mathcal I_X$. Still we write $\iota^{-1}\iota_+\mathcal M=k[\partial]\otimes_k\iota^{-1}\iota_*\mathcal M\cong k[\partial]\otimes_k\mathcal M$. Then by induction 
        \[y(\partial^p\otimes m)=-p\partial^{p-1}\otimes m\]
        which is zero if and only if $p=0$ or $m=0$. Thus, the subsheaf of $\iota^{-1}\iota_+\mathcal M$ killed by $\iota^{-1}\mathcal I_X$ is precisely isomorphic to $\mathcal M$, i.e., $\iota^!\iota_+\mathcal M\cong\mathcal M$.

        Conversely, fix some $\mathcal N\in\Mod^X(\mathcal D_Y)$. Then $\iota^!\mathcal N$ is (isomorphic to) the subsheaf of $\iota^{-1}\mathcal N$ killed by $y$, i.e., it is the kernel of the map $y:\iota^{-1}\mathcal N\to\iota^{-1}\mathcal N$. Denote by $\theta$ the operator $y\partial:\mathcal N\to\mathcal N$. For any $j\in\Z$, let $\mathcal N^j$ be the eigenspace of $\theta$ with an eigenvalue $j$. We claim that
        \[\mathcal N=\bigoplus_{j=1}^\infty\mathcal N^{-j}\]
        Since $\mathcal N$ is supported on $X$, as before we have $\mathcal N=\cup_{k=1}^\infty \ker y^k$. So it suffices to show
        \[\ker y^k\subseteq \bigoplus_{j=1}^k\mathcal N^{-j}\]
        For $k=1$ this is obvious: $yn=0$ means
        \[\theta n=y\partial n=(\partial y-1)n=-n\]
        Now assuming the assertion holds for $k$. Then for any section $n$ killed by $y^{k+1}$ we have
        \[yn\in\ker y^k\subseteq\bigoplus_{j=1}^k\mathcal N^{-j}\]
        Compute that if $n'\in\mathcal N^j$,
        \[\theta (yn')=y(\theta+1)n'=(j+1)yn', \quad \theta(\partial n')=\partial \theta n'-\partial n'=(j-1)\partial n'\]
        From this we see that $\partial y=\theta+1$ is an isomorphism $\mathcal N^j\to\mathcal N^j$ for $j\neq -1$, and for $j<-1$, the maps $y:\mathcal N^{j}\to\mathcal N^{j+1}$ and $\partial:\mathcal N^{j+1}\to\mathcal N^{j}$ are isomorphisms. Applying $\partial$ to $yn\in\oplus_{j=1}^{k}\mathcal N^{-j}$, the direct sum becomes $\oplus_{j=2}^{k+1}\mathcal N^{-j}$. Similarly, using the relation $[\partial, y]=1$ we deduce
        \[y^k(\theta n+(k+1)n)=y^{k+1}\partial n+(k+1)y^kn=\partial y^{k+1}n=0\]
        meaning $(\theta+k+1)n\in\ker y^k\subseteq\oplus_{j=1}^k\mathcal N^{-j}$. The final touch is the difference
        \[kn=(\theta+k+1)n-(\theta+1)n=(\theta+k+1)n-\partial yn\in\bigoplus_{j=1}^k\mathcal N^{-j}+\bigoplus_{j=2}^{k+1}\mathcal N^{-j}\subseteq\bigoplus_{j=1}^{k+1}\mathcal N^{-j}\]
        and as the base field has characteristic zero, $n$ sits inside the direct sum. Since $\partial \mathcal N^{j}\cong\mathcal N^{j-1}$, we have $\mathcal N=k[\partial]\otimes_{k}\mathcal N^{-1}$. Yet
        \[\theta n=-n\iff \partial yn-n=-n\iff yn=0\]
        as $\partial$ is an isomorphism, meaning $\iota^!\mathcal N=\ker y=\iota^{-1}\mathcal N^{-1}$. Again, as $\mathcal N^{-1}$ is supported on $X$, $\iota_*\iota^{-1}\mathcal N^{-1}\cong\mathcal N^{-1}$, i.e., 
        \[\mathcal N\cong k[\partial]\otimes_k\iota_*\iota^{-1}\mathcal N^{-1}\cong k[\partial]\otimes_k\iota_*\iota^!\mathcal N=\iota_+\iota^!\mathcal N\]
        completing the proof.
    \end{proof}
    Kashiwara's equivalence is not trivial and demonstrates that $\mathcal D_X$-module structures are very rigid. An immediate contrast arises from the annihilators of $\iota_+\mathcal M$ and $\iota_*\mathcal F$ (here $\mathcal M$ any $D$-module and $\mathcal F$ is an arbitrary $\mathcal O_X$-module). We see that $\mathcal I_X$ annihilates $\iota_*\mathcal F$ but not $\iota_+\mathcal M$, even though sections in the latter are killed by powers of $\mathcal I_X$.
    \begin{example}\label{exp-construct-by-push}
        Given a closed immersion $\iota:X\to Y$, we can construct an obvious $D$-module on $Y$ by $\mathcal B_{X|Y}=\iota_+\mathcal O_X$ (here I'm using the notation from \cite{htt-d-modules, ginzburg-notes}, although I'm not sure what $\mathcal B$ stands for). Using the same coordinate system in \cref{exp-push-local}, the module can locally be written as $k[\partial_{y_{r+1}},\dots,\partial_{y_n}]\otimes_k\iota_*\mathcal O_X\cong k[\partial_{y_{r+1}},\dots,\partial_{y_n}]\otimes_k\mathcal O_Y/\mathcal I_X$. Let $\mathcal J$ be the $\mathcal D_Y$-submodule generated by $\{\partial_{y_i}\}_{i=1,\dots, r}$ and $\{y_i\}_{i=r+1,\dots, n}$. Then $\mathcal B_{X|Y}=\mathcal D_Y/\mathcal J$.

        Suppose $X$ is a point $y\in Y$, cut out by a maximal ideal $\mathfrak m_y=(y_1,\dots, y_n)$. In this case $r=0$ so $\mathcal B_{y|Y}=\mathcal D_Y/\mathcal D_Y\mathfrak m_y$. Take some $\delta_y=1\bmod{\mathfrak m_y}$, we can rewrite $\mathcal B_{y|Y}$ as $\mathcal D_Y\cdot\delta_y$. In the sense of \cref{exp-Mf}, $\mathcal B_{y|Y}$ corresponds to the system of differential equations given by $y_iu=0$ for all $i$, which has the Dirac delta function at $y$ as a solution. Since $D$-modules on $y$ are simply $k$-vector spaces, $\Mod^y(\mathcal D_Y)$ is equivalent to the category of $k$-vector spaces, where $\mathcal B_{y|Y}$ corresponds to $k$. Thus, $D$-modules on $Y$ supported by $y$ are direct sums of $\mathcal B_{y|Y}$. On the other hand, $\mathcal O_X$-modules supported at a point are much more flexible. For instance, take $Y=\Spec k[t]$, $y$ the origin and $\mathcal N=\mathcal O_Y/(t^2)$. Then clearly $\mathcal N$ is supported at $y$ and $t$ acts nontrivially on $\mathcal N$. Thus, $\mathcal N$ is not the pushforward of any $\mathcal O_{\{y\}}$-modules. In general, $D$-modules cannot have nontrivial nilpotent sections, contrary to $\mathcal O_X$-modules.
    \end{example}
    To discuss other important properties of $D$-modules and results of Kashiwara's equivalence, we will introduce more types of $D$-modules that fit into the theory.

    \subsection{D-affinity}
    Suppose a scheme $X$ is Noetherian. A well-known result due to Serre states that $X$ is affine if and only if $H^i(X,\mathcal F)=0$ for all $i>0$ and quasicoherent $\mathcal O_X$-modules $\mathcal F$, and $\mathcal F$ is zero if it has no global sections. In other words, the global section functor $\Gamma(X, -)$ is exact. We will study a similar property of varieties $X$ by considering quasicoherent $\mathcal D_X$-modules.
    \begin{definition}
        A smooth algebraic variety $X$ is \textbf{$\mathcal D_X$-affine} if the global section functor $\Gamma(X,-):\Mod(\mathcal D_X)\to\Mod(\Gamma(X, \mathcal D_X))$ is exact, and if a $D$-module $\mathcal M$ has no global sections, $\mathcal M=0$.
    \end{definition}
    \begin{lemma}
        Smooth affine varieties are $\mathcal D_X$-affine.
    \end{lemma}
    \begin{proof}
        The global section functor of a smooth affine variety $X$ is exact in the category of $\mathcal O_X$-modules, and so is its restriction to $\Mod(\mathcal D_X)$. The second condition is clear for quasicoherent modules on affine schemes.
    \end{proof}
    \begin{remark}
        There are very few known examples of $\mathcal D_X$-affine spaces. It is conjectured that the only $\mathcal D_X$-affine varieties are affine varieties and partial flag varieties (to be introduced).
    \end{remark}
    One particularly important (and so far the first concrete) consequence of Kashiwara's equivalence is the $\mathcal D_X$-affinity of projective spaces:
    \begin{theorem}
        The fiber product of a projective space and a smooth affine variety is $\mathcal D_X$-affine.
    \end{theorem}
    \begin{proof}
        Let $X=\mathbb P^n_k\times_k Y$ where $Y=\Spec A$ is smooth affine. Consider the open subscheme $\widetilde V=\A^{n+1}_k\backslash \{0\}$ in $V=\A^{n+1}_k$ and the variety $\widetilde X=\widetilde V\times Y$. Denote by $\pi:\widetilde X\to X$ the quotient map and $j:\widetilde{V}\hookrightarrow V$ the open immersion. The multiplicative group $\mathbb G_m$ acts on $\widetilde V$ by scaling and thus we have an action $\s:\mathbb G_m\times \widetilde V\times Y\to \widetilde V\times Y$. The coaction $\s^\#$ of $\s$ defines a map $\Gamma(\widetilde{X}, \mathcal O_{\widetilde{X}})\to k[t, t^{-1}]\otimes_k \Gamma(\widetilde{X}, \mathcal O_{\widetilde{X}})$ given by $x_i\otimes a\mapsto t\otimes x_i\otimes a$ (here $a\in \Gamma(Y, \mathcal O_Y)$). By the properties of group scheme actions, we obtain a $\Z$-grading of $\Gamma(\widetilde{X}, \mathcal O_{\widetilde{X}})$ given by eigenspaces of $t$ with eigenvalues $t^n$, consisting of $f\otimes a$ where $f$ is a homogeneous polynomial in $k[x_1,\dots, x_{n+1}]$ of degree $n$. Now for any quasicoherent left $\mathcal D_X$-module $\mathcal M$, we consider the module $\pi^*\mathcal M=\mathcal O_{\widetilde{X}}\otimes_{\pi^{-1}\mathcal O_X}\pi^{-1}\mathcal M$. From the grading of $\Gamma(\widetilde{X}, \mathcal O_{\widetilde{X}})$, the global section of this module has a decomposition
        \[\Gamma(\widetilde{X}, \pi^*\mathcal M)=\bigoplus_{n\in\Z}\Gamma(\widetilde{X}, \pi^*\mathcal M)^n\]
        where $\Gamma(\widetilde{X}, \pi^*\mathcal M)^n$ is eigenspace of $t$ with eigenvalue $t^n$. Multiplying by $x_i$ (resp. $\partial_i$) increases (resp. decreases) the homogenous degree by $1$. Now, the \textbf{Euler vector field} $\theta=\sum_i x_i\partial_i$ acts on $\Gamma(\widetilde{X}, \mathcal O_{\widetilde{X}})$ and it computes the degree of homogeneous polynomials in $\Gamma(Y, \mathcal O_Y)[x_1,\dots, x_{n+1}]$. If we let $\theta$ act on $\pi^*\mathcal M$ by $\theta\otimes\id$, then it is immediate that
        \[\Gamma(\widetilde{X}, \pi^*\mathcal M)^n=\{u\in\Gamma(\widetilde{X}, \pi^*\mathcal M) :\theta \cdot u=nu\}\]
        Moreover, homogeneous sections of degree $0$ correspond precisely to the global sections of $\mathcal M$ as $\pi$ is a surjective open map, meaning $\Gamma(\widetilde{X}, \pi^*\mathcal M)^0=\Gamma(X, \mathcal M)$.

        Let $0\to \mathcal M_1\to\mathcal M_2\to\mathcal M_3\to 0$ be an exact sequence of $D$-modules on $X$. The quotient map is smooth so $\pi^*$ is exact. The pushforward $j_*$ is left exact. Thus we need to understand $R^1j_*\pi^*\mathcal M_1$. Note that \cref{exp-open-immersion-pushforward} says $R^1j_*=H^1(\dR j_*)=H^1(j_+)$, so by \cref{prop-pushforward-quasicoherent}, $R^1j_*\pi^*\mathcal M_1$ is a quasicoherent $\mathcal D_{V\times Y}$-module. For an open affine $U\subseteq V\times Y$ not intersecting $\{0\}\times Y$, $j:j^{-1}(U)\to U$ is an isomorphism so $R^1j_*\pi^*\mathcal M_1$ is supported on the closed subscheme $\{0\}\times Y$. By Kashiwara's equivalence, we see that there exists some $D$-module $\mathcal N$ on $\{0\}\times Y$ such that
        \[R^1j_*\pi^*\mathcal M_1=\iota_+\mathcal N\cong k[\partial_1,\dots,\partial_{n+1}]\otimes_k\mathcal \iota_*N\]
        where $\iota:\{0\}\times Y\hookrightarrow V\times Y$ with a kernel generated by all $\{x_i\}$. Since
        \[x_i(\partial_j^{k}\otimes s)=-k\delta_{ij}(\partial_j^{k-1}\otimes s)\]
        the action of $\theta$ on global sections of $R^1j_*\pi^*\mathcal M_1$ has negative eigenvalues. Now, $\Gamma(V\times Y, -)$ is exact so we get a long exact sequence
        \[0\to\Gamma(V\times Y,j_*\pi^*\mathcal M_1)\to \Gamma(V\times Y,j_*\pi^*\mathcal M_2)\to\Gamma(V\times Y,j_*\pi^*\mathcal M_3)\to\Gamma(V\times Y,R^1j_*\pi^*\mathcal M_1)\to\cdots\]
        Notice that $\Gamma(V\times Y,j_*\pi^*\mathcal M_i)=\Gamma(\tilde X,\pi^*\mathcal M_i)$, and maps in the exact sequences are $A_{n+1}\otimes_k\Gamma(Y,\mathcal O_Y)$-module homomorphisms, meaning they commute with $\theta$. Thus, taking the eigenspace with eigenvalue $0$ of $\theta$ preserves exactness. Together with the fact $\Gamma(\tilde X,\pi^*\mathcal M_i)^0=\Gamma(X, \mathcal M)$, one obtains the exact sequence
        \[0\to \Gamma(X, \mathcal M_1)\to \Gamma(X, \mathcal M_2)\to \Gamma(X, \mathcal M_3)\to 0\]
        So $\Gamma(X, -)$ is exact.

        It remains to show that if $\Gamma(X, \mathcal M)=0$ then $\mathcal M=0$. Assume otherwise, then as $\pi$ is surjective and flat, $\pi^*\mathcal M$ is nonzero. Take some nonzero $s\in\Gamma(\widetilde X, \pi^*\mathcal M)$ with eigenvalue $n$. If $n=0$, we are done. If $n>0$, then $\theta s=ns\neq 0$, meaning some $\partial_is\neq 0\in\Gamma(\widetilde X, \pi^*\mathcal M)^{n-1}$. Repeating the same thing, we get a nonzero element in $\Gamma(\widetilde X, \pi^*\mathcal M)^0=\Gamma(X,\mathcal M)$, a contradiction. If $n<0$, then there must some $x_is\neq 0$ because otherwise, $s\in\Gamma(\widetilde X, \pi^*\mathcal M)=\Gamma(V\times Y, j_*\pi^*\mathcal M)$ is annihilated by the ideal sheaf of $\{0\}\times Y$, meaning it's on $\{0\}\times Y$ and as $\tilde X$ is open, $s$ must be zero. Then we get a nonzero element $x_is\in \Gamma(\widetilde X, \pi^*\mathcal M)^{n+1}$. Again this eventually results in a nonzero global section of $\mathcal M$, a contradiction, completing the proof.
    \end{proof}
    \begin{example}
        The flag variety of $SL_2$ is $X=\mathbb P^1_k$. The above actually shows that there is an equivalence of categories
        \[\Mod(\mathcal D_X)\xrightarrow{\sim}\Mod(\Gamma(X,\mathcal D_X))\]
        Once we connect $\Mod(\Gamma(X,\mathcal D_X))$ with representations of $\mathfrak{sl}_2$, the above would become the celebrated Beilinson-Bernstein localization.
    \end{example}
    In the near future, the sheaf of algebras of interest would no longer be $\mathcal D_X$. It will be replaced by twisted versions of $\mathcal D_X$. Recall that the cotangent bundle $T^*X$ of $X$ is naturally a symplectic algebraic variety, with a symplectic form $\w$. The closed nondegenerate $2$-form induces a Poisson bracket on $\mathcal O_{T^*X}$ given by $\{f,g\}=\w(H_f,H_g)$ where $H_f$ and $H_g$ are the Hamiltonian vector fields. As $\pi_*\mathcal O_{T^*X}\cong\Sym\Theta_X$, we have a Poisson structure on $\Sym\Theta_X$. Indeed, the Poisson structure given by $\gr^F\mathcal D_X\cong\Sym\Theta_X$ is the same as the symplectic one.
    \begin{definition}
        A \textbf{sheaf of twisted differential operators} (abbreviation t.d.o.) $\mathcal D$ on $X$ is a quasicoherent sheaf of almost commutative algebras (cf. \cref{def-almost-comm}) with a positive filtration $F_\bullet\mathcal D$ such that Poisson algebras $\gr^F\mathcal D$ and $\Sym\Theta_X$ are isomorphic.
    \end{definition}
    \begin{remark}
        By definition, if $\mathcal D$ is a t.d.o., there is an isomorphism $F_0\mathcal D\cong\mathcal O_X$. The commutator in $\mathcal D$ gives us a derivation $[\xi, \cdot]$ on $F_0\mathcal D$ for any $\xi\in F_1\mathcal D$, i.e., every section in $F_1\mathcal D$ induces a section of $\Theta_X$. In fact, we have $\Theta_X\cong F_1\mathcal D/F_0\mathcal D$. Thus, an isomorphism $\Sym\Theta_X\cong \gr^F\mathcal D$ of $\mathcal O_X$-modules automatically respects the Poisson structures. We obtained an equivalent definition of t.d.o.: (i) $F_0\mathcal D\cong\mathcal O_X$, (ii) $F_1\mathcal D/F_0\mathcal D\cong\Theta_X$ and (iii) $\gr^F\mathcal D\cong\Sym\Theta_X$.
    \end{remark}
    \begin{example}
        Clearly $\mathcal D_X$ is a t.d.o. Suppose $\mathcal L$ is an invertible sheaf on $X$. Then we've seen the sheaf of differential operators $\Diff(\mathcal L,\mathcal L)$ which is isomorphic to $\mathcal L\otimes_{\mathcal O_X}\mathcal D_X\otimes_{\mathcal O_X}\mathcal L^\vee$ by \cref{cor-twist-end}. From now we denote this sheaf by $\mathcal D_X^{\mathcal L}$. Note that $F_0\mathcal D_X^{\mathcal L}=\mathcal Hom_{\mathcal O_X}(\mathcal L,\mathcal L)=\mathcal O_X$. Indeed, it's not difficult to check (using the same argument as $\mathcal D_X$ affine locally) that $\mathcal D_X^{\mathcal L}$ is a t.d.o. Note that $\mathcal D_X^\op=\mathcal D_X^{\w_{X/k}}$ we previously studied is an example.
    \end{example}
    \begin{example}
        If $\mathcal E$ is a locally free sheaf of rank $>1$, $\Diff(\mathcal E,\mathcal E)$ is not a t.d.o. per our definition since $\mathcal End_{\mathcal O_X}(\mathcal E)$ is not $\mathcal O_X$.
    \end{example}
    \begin{definition}
        Given a t.d.o $\mathcal D$ on $X$, we say $X$ is \textbf{$\mathcal D$-affine} if $\Gamma(X, -)$ is an exact functor on $\Mod(\mathcal D)$ and every quasicoherent $\mathcal D$-module $\mathcal M$ is generated by global sections (equivalently $\Gamma(X, \mathcal M)=0$ implies $\mathcal M=0$).
    \end{definition}
    \begin{example}
        We are not defining something redundant/unnecessary here. The projective line $X=\mathbb P^1_k$ is $\mathcal D_X$-affine, but $\Gamma(X,\w_{X/k})=0$ for the left $\mathcal D_X^\op$-module $\w_{X/k}=\mathcal O_{X}(-2)$. Therefore, $\mathbb P^1_k$ is not $\mathcal D_X^\op$-affine.
    \end{example}
    Quasicoherent $\mathcal D$-modules on $\mathcal D$-affine varieties play the role of quasicoherent $\mathcal O_X$-modules on affine schemes:
    \begin{proposition}\label{prop-d-affinity}
        Given a $\mathcal D$-affine variety $X$, the global section functor gives an equivalence of categories 
        \[\Mod(\mathcal D)\to\Mod(\Gamma(X, \mathcal D))\]
        with quasi-inverse $\mathcal D\otimes_{\Gamma(X, \mathcal D)}-$.
    \end{proposition}
    \begin{remark}
        I need to clarify the notation here. The tensor product $\mathcal D\otimes_{\Gamma(X, \mathcal D)}M$ is the sheaf associated to the presheaf $U\mapsto \Gamma(U, \mathcal D)\otimes_{\Gamma(X,\mathcal D)}M$ where $\Gamma(X,\mathcal D)$ acts on the first component by restricting to $U$ and multiplying on the left. Given an $R$-module $M$, under the same notation $\widetilde{\mathcal M}=\mathcal O_X\otimes_RM$ on $X=\Spec R$. On each distinguished open $D(f)$, we have $\Gamma(D(f), \widetilde{\mathcal M})=M_f$ which is a localization. Therefore, the functor $\mathcal D\otimes_{\Gamma(X, \mathcal D)}-$, being a $D$-module analogy of $\mathcal O_X\otimes_{\Gamma(X, \mathcal O_X)}-$, gives us a sense of localization.
    \end{remark}
    \begin{proof}
        First notice the two functors are adjoint by the adjunction of Hom and tensor products, and the fact that $\Hom_{\mathcal D}(\mathcal D, -)=\Gamma(X,-)$. Write $D=\Gamma(X,\mathcal D)$ for convenience. We need to show for any $\mathcal D$-module $\mathcal M$, $\mathcal D\otimes_{D}\Gamma(X, \mathcal M)\cong\mathcal M$, and for any $D$-module $N$, $N\cong\Gamma(X, \mathcal D\otimes_DN)$. Let $N$ be presented by
        \[D^{\oplus I}\to D^{\oplus J}\to N\to 0\]
        for suitable $I$ and $J$. Applying first the right exact functor $\mathcal D\otimes_{D}-$ and then the exact functor $\Gamma(X, -)$, we get a commutative diagram with exact rows 
        \[\begin{tikzcd}
            D^I \arrow[rr] \arrow[dd] &  & D^J \arrow[rr] \arrow[dd] &  & N \arrow[rr] \arrow[dd, "\eta"]                &  & 0 \arrow[dd] \\
                                      &  &                           &  &                                                &  &              \\
            D^I \arrow[rr]            &  & D^J \arrow[rr]            &  & {\Gamma(X, \mathcal D\otimes_DN)} \arrow[rr] &  & 0           
            \end{tikzcd}\]
        By the four lemma, $\eta$ is an isomorphism. Since $\mathcal M$ is generated by global sections,  the map $\mathcal D\otimes_{D}\Gamma(X, \mathcal M)\to \mathcal M$ is surjective. Let $\mathcal K$ be its kernel. Then since $\Gamma(X, -)$ is exact, we get an exact sequence
        \[0\to\Gamma(X, \mathcal K)\to\Gamma(X, \mathcal D\otimes_{D}\Gamma(X, \mathcal M))\to\Gamma(X, \mathcal M)\to 0\]
        We've shown that $\Gamma(X, \mathcal D\otimes_{D}\Gamma(X, \mathcal M))\cong \Gamma(X, \mathcal M)$ so $\Gamma(X,\mathcal K)=0$. Again by $\mathcal D$-affinity, the kernel $\mathcal K$ is zero, completing the proof.
    \end{proof}





    \subsection{Coherent D-modules}
    A property of $D$-modules we will treat is coherence. Recall we say a quasicoherent $\mathcal D$-module is \textbf{coherent} if it is locally finitely generated over $\mathcal D$ (we are working with locally Noetherian schemes). We denote by $\Mod_c(\mathcal D)$ the category of coherent $\mathcal D$-modules on $X$. Coherence of $\mathcal D_X$-modules is different from coherence over $\mathcal O_X$.
    \begin{proposition}\label{prop-ox-coh-loc-free}
        A $\mathcal D_X$-module is coherent over $\mathcal O_X$ if and only if it's locally free of finite rank over $\mathcal O_X$.
    \end{proposition}
    \begin{proof}
        I will only sketch the proof here since it's nonessential. We want to show that $\mathcal M_x$ is locally free of finite rank over $\mathcal O_{X, x}$ for each $x\in X$. Let $\{x_i,\partial_i\}$ be a local coordinate system of $X$ such that the maximal ideal $\mathfrak m_x$ is generated by the $x_i$. Then by Nakayama's lemma, there exists a $\kappa(x)$-basis of $\mathcal M_x/\mathfrak m_x\mathcal M_x$ which lifts to a spanning set $s_1,\dots, s_n$ of $\mathcal M_x$ over $\mathcal O_{X, x}$. Suppose there is an equality
        \[\sum_{i=1}^mf_is_i=0\]
        Define the order of $f\in\mathcal O_{X, x}$ by $\ord(f)=\max\{p:f\in\mathfrak m_x^p\}$. Applying $\partial_j$ to the equality we get
        \[0=\sum_{i=1}^m[(\partial_jf_i)s_i+f_i(\partial_js_i)]=\sum_{i=1}^m\left(\partial_j f_i+\sum_{k=1}^m a_{ikj}f_k\right)s_i\]
        where we write $\partial_j s_i=\sum_k a_{ijk}s_k$. Let $f_i$ be the term with minimal order $l$. Let $\partial_j$ be the operator that reduces the order of $f_i$. Then $\partial_j f_i+\sum_{k=1}^m a_{ikj}f_k$ sits inside $\mathfrak m_x^{l-1}$, i.e., the minimal order of coefficients is reduced by $1$. Repeating the reduction, we arrive at a sum with coefficients of minimal order $0$. Yet since the $s_i$ are linearly independent over $\kappa(x)$ in $\mathcal M_x/\mathfrak m_x\mathcal M_x$, the minimal order must be larger than zero, or otherwise a nontrivial relation exists.
    \end{proof}
    \begin{remark}
        A proof using Kashiwara's equivalence can be found in 2.1 in \cite{bernstein-notes}. I do not know a lot about curves so I did not include this proof.
    \end{remark}
    \begin{example}
        The result in Kashiwara's equivalence can be strengthened for coherent $D$-modules. Indeed, if $\iota:X\hookrightarrow Y$ is a closed immersion where $X$ is a hypersurface in $Y$, then $\iota_+\mathcal M=k[\partial]\otimes_k\iota_*\mathcal M$ is coherent over $\mathcal D_Y$ if $\mathcal M$ is $\mathcal D_X$-coherent. Conversely, $\iota^!\mathcal N=\iota^{-1}\mathcal N^{-1}$ where $\mathcal N=k[\partial]\otimes_k\mathcal N^{-1}$ is a coherent $D$-module on $Y$ supported on $X$. It is clear that $\mathcal N^{-1}$ is locally finitely generated and thus so is $\iota^!\mathcal N$. Therefore we have a coherent Kashiwara's equivalence: there is an equivalence of categories
        \[\iota_+:\Mod_c(\mathcal D_X)\xrightarrow{\sim}\Mod_c^X(\mathcal D_Y)\]
        with its quasi-inverse $\iota^!$.
    \end{example}
    The equivalence of categories provided by $\mathcal D$-affinity can also be refined to coherent $\mathcal D$-modules.
    \begin{lemma}\label{lem-coh-fin-equiv}
        If $X$ is $\mathcal D$-affine, the global section functor and localization functor give an equivalence of categories
        \[\Mod_c(\mathcal D)\simeq\Mod_f(\Gamma(X, \mathcal D))\]
        where $\Mod_f$ is the category of finite modules.
    \end{lemma}
    \begin{proof}
        Clearly $\mathcal D_X\otimes M$ for a finite $\Gamma(X, \mathcal D_X)$-module $M$ is coherent. Conversely, any coherent $\mathcal D_X$-module $\mathcal M$ is generated by finitely many global sections as $X$ is quasicompact.
    \end{proof}
    Recall how we defined good filtrations on modules over filtered rings in \cref{def-filtered-module}. We can say the same thing about $\mathcal D_X$-modules.
    \begin{definition}
        A \textbf{good filtration} on a $\mathcal D_X$-module $\mathcal M$ is a filtration $F_\bullet\mathcal M$ such that each $F_s\mathcal M$ is coherent over $\mathcal O_X$ and for large enough $s$, $F_{l+s}\mathcal M=F_l\mathcal D_X\cdot F_s\mathcal M$ for any $l$.
    \end{definition}
    \begin{example}
        The filtration of  $\mathcal D_X$ is good.
    \end{example}
    \cref{prop-good-gr-finite} and \cref{lem-good-finite} can thus be translated into global statements:
    \begin{proposition}
        A filtration $F_\bullet\mathcal M$ on a $\mathcal D_X$-module $\mathcal M$ is good if and only if $\gr^F\mathcal M$ is coherent over $\gr^F\mathcal D_X\cong\Sym\Theta_X\cong\pi_*\mathcal O_{T^*X}$.
    \end{proposition}
    \begin{lemma}\label{lem-coherent-iff-good}
        A $\mathcal D_X$-module $\mathcal M$ has a good filtration if and only if it's a coherent $\mathcal D_X$-module.
    \end{lemma}
    \begin{proof}
        \cref{lem-good-finite} only says a $D$-module $\mathcal M$ is coherent if and only if it admits a locally good filtration. If $\mathcal M$ has a globally good filtration, then it's locally good and thus coherent. We need to show the converse, i.e., if $\mathcal M$ is coherent then it has a globally good filtration.

        Our algebraic variety $X$ can be covered by finitely many affine opens $\{U_i\}$. Then each $\Gamma(U_i,\mathcal M)$ is finitely generated over $\Gamma(U_i,\mathcal D_X)$, say with a set of generators $\{s_i\}$. Let $\mathcal N_{i}$ be the $\mathcal O_X$-submodule generated by this set on $U_i$. We obtain coherent $\mathcal O_X$-modules $\mathcal M_i$ such that $\mathcal M_i|_{U_i}=\mathcal N_i$. Then their direct sum maps to a coherent $\mathcal O_X$-submodule $\mathcal M'$ of $\mathcal M$ that generates $\mathcal M$ over $\mathcal D_X$. Hence we can define a filtration $F_s\mathcal M=F_s\mathcal D_X\cdot\mathcal M'$, each coherent over $\mathcal O_X$ and the multiplication relation holds trivially, completing the proof.  
    \end{proof}
    We would like to understand to what extent a coherent $D$-module fails to be $\mathcal O_X$-coherent. This is one of the various purposes of the following geometric object.
    \begin{definition}
        Let $\mathcal M$ be a $D$-module on $X$ with a good filtration $F_\bullet\mathcal M$. The \textbf{characteristic variety} (or singular support) of $\mathcal M$ is
        \[\ch(\mathcal M)=\supp\pi^*\gr^F\mathcal M\]
    \end{definition}
    \begin{remark}
        $F_\bullet \mathcal M$ is good so $\gr^F\mathcal M$ is coherent over $\pi_*\mathcal O_{T^*X}$. Varieties $X$ and $T^*X$ are locally Noetherian so $\pi^*\gr^F\mathcal M$ is coherent over $\mathcal O_{T^*X}$, meaning $\ch(\mathcal M)$ can be given a reduced closed subscheme structure in $T^*X$. Note that $\ch(\mathcal M)$ is independent of the good filtration. This is a nontrivial but uninspiring fact. For a proof see \cite{bernstein-notes}. 
    \end{remark}
    \begin{example}
        I will compute some characteristic varieties here. Let $X=\Spec k[x]$. As we've shown, $\gr^F\mathcal D_X$ is the sheaf associated to $k[x, \xi]$ where $\xi$ is the image of $\partial_x$, and $T^*X=\mathbb A_k^2$. Then $\Gamma(T^*X, \pi^*\gr^F\mathcal O_X)=k[x,\xi]/(\xi)$ so the sheaf's support is the curve $\xi=0$. Similarly, the module $\mathcal D_X/\mathcal D_X(\partial x)$ has characteristic variety $\{x\xi=0\}$, the same as $\ch(\mathcal D_X/\mathcal D_X(x\partial-\lambda))$.
    \end{example}
    Results in algebraic geometry suggest the characteristic variety $\ch(\mathcal M)$ is the subvariety given by the ideal
    \[\mathcal I_{\mathcal M}=\sqrt{\Ann_{\gr^F\mathcal D_X}(\gr^F\mathcal M)}\]
    since $T^*X=\underline{\Spec}\gr^F\mathcal D_X$. Relevant results can be found in \cite{stacks-project}. The following result demonstrates the interpretation of characteristic varieties mentioned before, that is, a measurement of lack of $\mathcal O_X$-coherence.
    \begin{proposition}\label{prop-ox-coh-zero-sec}
        A nonzero coherent $\mathcal D_X$-module $\mathcal M$ is coherent over $\mathcal O_X$ if and only if $\ch(\mathcal M)$ is the zero section of $T^*X$, isomorphic to $X$.
    \end{proposition}
    \begin{proof}
        Suppose $\mathcal M$ is coherent over $\mathcal O_X$. Then by \cref{prop-ox-coh-loc-free}, it is locally free of rank $n>0$ over $\mathcal O_X$. Then a good filtration on $\mathcal M$ can be obtained by setting $F_i\mathcal M=\mathcal M$ for all $i\geqslant 0$ and $\gr^F\mathcal M=\mathcal M$ which is locally $\mathcal O_X^n$. Then clearly $\Theta_X$ annihilates $\gr^F\mathcal M$ and thus $\ch(\mathcal M)$ sits inside the base space $X\subseteq T^*X$. Since $\gr^F\mathcal M$ has support equal to $X$, $\ch(\mathcal M)\cong X$. Conversely, we can assume $X$ is affine of dimension $n$ so $T^*X=X\times\mathbb A_k^n$. If $\ch(\mathcal M)$ is the zero section of $T^*X$, then
        \[\mathcal I=\sqrt{\Ann_{\gr^F\mathcal D_X}(\gr^F\mathcal M)}=\sum_{i=1}^n\mathcal O_X[\xi_1,\dots,\xi_n]\xi_i\]
        where $F_\bullet\mathcal M$ is a good filtration. So for a large enough $r$, $\mathcal I^r$ annihilates $\gr^F\mathcal M$ since $\mathcal I$ is finitely generated. But then there is some large enough $s$ such that
        \[F_{r+s}\mathcal M=F_r\mathcal D_X\cdot F_{s}\mathcal M=\sum_{|\alpha|\leqslant r}\mathcal O_X\partial^\alpha\cdot F_s\mathcal M\subseteq F_{r+s-1}\mathcal M\]
        where the inclusion follows from the fact that $\xi^\alpha$ with $|\alpha|=r$ generates $I^r$ which annihilates $\gr^F_{r+s}\mathcal M=F_{r+s}\mathcal M/F_{r+s-1}\mathcal M$. Therefore, we must have $F_{r+s}\mathcal M=\mathcal M$ and by goodness the LHS is coherent over $\mathcal O_X$.
    \end{proof}
    \begin{theorem}[Bernstein's inequality]
        Given a coherent $\mathcal D_X$-module $\mathcal M$, then $\ch(\mathcal M)$ has dimension $\geqslant \dim X$.
        \label{thm-bernstein-ineq}
    \end{theorem}
    \begin{remark}
        The theorem actually holds for every irreducible component of $\ch(\mathcal M)$. But we will not go into details of that here.
    \end{remark}
    \begin{proof}
        Let $\iota:S\hookrightarrow X$ be a closed immersion where $S$ is a smooth closed subvariety. Let $\mathcal M$ be any coherent $\mathcal D_S$-module.
        \begin{lemma}
            We have $\dim\ch(\iota_+\mathcal M)=\dim\ch(\mathcal M)-(\dim S-\dim X)$.
        \end{lemma}
        \begin{proof}
            As in the proof of Kashiwara's equivalence, we can reduce to the case of $S$ being a hypersurface of $X$. Let $\{x_i,\partial_i\}$ be a local coordinate of $X$ such that $x_1$ defines $S$. Then $\iota_+\mathcal M=k[\partial_1]\otimes_k\iota_*\mathcal M$. Since $\mathcal M$ is $\mathcal D_S$-coherent, by \cref{lem-coherent-iff-good}, we can choose a good filtration $F_\bullet\mathcal M$. Then we can define
            \[F_j\iota_+\mathcal M=\sum_{l=0}^j\sum_{s\leqslant l}k\cdot\partial_1^s\otimes\iota_*F_{j-l}\mathcal M\]
            which is a good filtration on $\iota_+M$ since each term is coherent and $F_\bullet\mathcal M$ is good. An easy computation shows that
            \[F_j\iota_+\mathcal M/F_{j-1}\iota_+\mathcal M=\bigoplus_{l=0}^jk\cdot \partial_1^l\otimes\iota_*(F_{j-l}\mathcal M/F_{j-l-1}\mathcal M)\]
            and thus
            \[\gr^F\iota_+\mathcal M\cong k[\xi]\otimes_k\gr^F\mathcal M\cong k[x, \xi]/(x)\otimes_k\gr^F\mathcal M\]
            The annihilator of $\gr^F\iota_+\mathcal M$ in $\gr^F\mathcal D_X$ is thus generated by $x$ and the annihilator of $\gr^F\mathcal M$, and the same relation holds for their radicals. Then an exercise on Krull dimensions shows that the characteristic varieties, defined by the radical of annihilators, have dimensions given by the statement of this lemma.
        \end{proof}
        Now if $X$ has dimension $0$ the theorem is clearly true. We can induct on $\dim X$. If $\supp \mathcal M$ is the entire $X$, $\ch(\mathcal M)$ contains the zero section of $T^*X$ and thus has dimension $\geqslant \dim X$. Otherwise, say $\supp\mathcal M$ is a closed proper subset of $X$, i.e., of dimension strictly less than $X$. Then we can choose a closed hypersurface $S$ of $X$ containing $\supp\mathcal M$. By Kashiwara's equivalence, there is a $\mathcal D_S$-module $\mathcal N$ such that $\iota_+\mathcal N=\mathcal M$. Therefore, $\dim\ch(\mathcal M)=\dim\ch(\mathcal N)+1$ by the previous lemma. The induction hypothesis says $\dim\ch(\mathcal N)\geqslant \dim S=\dim X-1$, so we are done.
    \end{proof}
    \begin{remark}
        Indeed, the theorem is a consequence of a much stronger fact: the characteristic variety of $\mathcal M$ is \textbf{involutive} with respect to the symplectic structure of $T^*X$, meaning the tangent spaces of $\ch(\mathcal M)$ at $(x,\xi)\in T^*X$ has dimension $\geqslant \frac{1}{2}\dim T_{(x,\xi)}(T^*X)=\dim X$. The fact is equivalent to $\{\mathcal I_{\mathcal M},\mathcal I_{\mathcal M}\}\subseteq\mathcal I_{\mathcal M}$. The theorem of involutivity is due to Gabber, and a proof of Bernstein's inequality via this route can be found in \cite{ginzburg-notes}.
    \end{remark}




    \subsection{Holonomic D-modules}
    The characteristic variety of a coherent $\mathcal D_X$-module reveals much more than a lack of $\mathcal O_X$-coherence.
    \begin{definition}
        We say a left coherent $\mathcal D_X$-module $\mathcal M$ is \textbf{holonomic} if $\dim\ch(\mathcal M)=\dim X$. Denote by $\Mod_h(\mathcal D_X)$ the full subcategory of $\Mod_c(\mathcal D_X)$ of holonomic $D$-modules. Let $D^b_h(\mathcal D_X)$ be the derived categories of complexes of $\mathcal D_X$-modules with holonomic cohomologies.
    \end{definition}
    Indeed, $\Mod_h(\mathcal D_X)$ is an abelian category. If $0\to\mathcal M_1\to\mathcal M_2\to\mathcal M_3\to 0$ is exact in $\Mod_c(\mathcal D_X)$, then as $\gr$ and localizations preserve exactness, $\ch(\mathcal M_2)=\ch(\mathcal M_1)\cup\ch(\mathcal M_3)$. Therefore, $\mathcal M_2$ is holonomic if and only if $\mathcal M_1$ and $\mathcal M_3$ are. Although the characteristic variety is not isomorphic to $X$, their equal dimensions still imply a strong sense of $\mathcal O_X$-coherence.
    \begin{proposition}
        Let $\mathcal M$ be a holonomic $\mathcal D_X$-module. Then there exists an open dense subset $U$ of $X$ such that $\mathcal M|_U$ is coherent over $\mathcal O_U$.
    \end{proposition}
    \begin{proof}
        Regard $X$ as the zero section of $T^*X$. Let $S$ be the complement of $X$ in $\ch(\mathcal M)$. If $S$ is empty, then we are done by \cref{prop-ox-coh-zero-sec}. Suppose $S\neq\emptyset$. The group $\mathbb G_m$ acts on fibers of $S$, so the dimension of each fiber of $\pi|_S$ is positive, meaning $\dim\pi(S)<\dim S\leqslant \dim\ch(\mathcal M)=\dim X$. Therefore, there is some nonempty open $U$ in $X$ containing the complement of $\pi(S)$. But then $\ch(\mathcal M|_U)$ is completely contained in the zero section of $T^*U$, so again by \cref{prop-ox-coh-zero-sec}, $\mathcal M|_U$ is coherent over $\mathcal O_X$, completing the proof.
    \end{proof}
    Now I will state some theorems without proof.\footnote[2]{Holonomic $D$-modules form an essential part of the theory. They are related to important objects such as perverse sheaves. However, proofs of results stated here deviate significantly from the purpose of this essay. So to make some room for later in-depth discussions, I will omit their proofs. But to compensate, we will go over some examples and illustrate the main ideas.} For proofs, see 3.2.3, 2.6.5 and 2.6.8 in \cite{htt-d-modules}. Like the results on pullbacks and pushforwards of $\mathcal D_X$-modules, the same holds for holonomic $\mathcal D_X$-modules.
    \begin{proposition}
        Given a morphism $f:X\to Y$, then $f_+$ sends $D_h^b(\mathcal D_X)$ to $D_h^b(\mathcal D_Y)$, and $f^!$ sends $D_h^b(\mathcal D_Y)$ to $D_h^b(\mathcal D_X)$
    \end{proposition}
    We also want to have a dual notion for $\mathcal D_X$-modules. The obvious way would be to define $\mathcal Hom_{\mathcal D_X}(\mathcal M,\mathcal D_X)\otimes_{\mathcal O_X}\w_{X/k}^{\vee}$. But the functor $\mathcal Hom_{\mathcal D_X}(-,\mathcal D_X)\otimes_{\mathcal O_X}\w_{X/k}^{\vee}$ is not exact, so we consider 
    \begin{definition}
        The \textbf{Grothendieck-Serre duality functor} $\mathbb D:D^b(\mathcal D_X)\to D^b(\mathcal D_X)$ of $\mathcal D_X$-modules is given by
        \[\mathbb D\mathcal M^\bullet=\dR\mathcal Hom_{\mathcal D_X}(\mathcal M^\bullet,\mathcal D_X\otimes_{\mathcal O_X}\w_{X/k}^{\vee}[\dim X])\]
        We shift the complex by $\dim X$ because we want the index in the theorem below to be zero.
    \end{definition}
    \begin{proposition}
        The duality functor sends $D_c^b(\mathcal D_X)$ to $D_c^b(\mathcal D_X)^{\op}$. the double dual is isomorphic to identity.
    \end{proposition}
    \begin{theorem}\label{thm-holonomic-criterion}
        A coherent $D$-module $\mathcal M$ is holonomic if and only if $H^i(\mathbb D\mathcal M)=0$ for all $i\neq 0$. Moreover, if $\mathcal M$ is holonomic then $H^0(\mathbb D\mathcal M)$ is holonomic. In particular, $H^0(\mathbb D(-))$ gives a duality on $\Mod_h(\mathcal D_X)$.
    \end{theorem}
    \begin{example}
        On the affine line $X=\Spec k[x]$, we've seen that $\ch(\mathcal O_X)=\{\xi=0\}$ which has dimension 1 so $\mathcal O_X$ is holonomic. Similarly, modules $\mathcal D_X/\mathcal D_X(\partial x)$ and $\mathcal D_X/\mathcal D_X(x\partial-\lambda)$ have characteristic variety $\{x\xi=0\}$ with two irreducible components of dimension 1, so still holonomic.
    \end{example}
    \begin{example}
        Let $\iota:X=\{0\}\hookrightarrow Y=\mathbb A_k^1$. In \cref{exp-construct-by-push}, we showed that $\mathcal B_{X|Y}=\iota_+\mathcal O_X\cong\mathcal D_Y/\mathcal D_Y\cdot y$. We have an exact sequence
        \[0\to\mathcal D_Y\to\mathcal D_Y\to\mathcal B_{X|Y}\to 0\]
        Applying $\mathcal Hom_{\mathcal D_Y}(-,\mathcal D_Y)$ to obtain an exact sequence
        \[0\to\mathcal Hom_{\mathcal D_Y}(\mathcal B_{X|Y},\mathcal D_Y)\to\mathcal D_Y\to\mathcal D_Y\]
        so $\mathcal Ext^0_{\mathcal D_Y}(\mathcal B_{X|Y},\mathcal D_Y)=\mathcal Hom_{\mathcal D_Y}(\mathcal B_{X|Y},\mathcal D_Y)=\ker y=0$ and $\mathcal Ext^1_{\mathcal D_Y}(\mathcal B_{X|Y},\mathcal D_Y)=\mathcal D_Y/\mathcal D_Y\cdot y$ as $Y$ has dimension one so all other cohomologies vanish. Then $H^0(\mathbb D\mathcal B_{X|Y})=\mathcal Ext^1_{\mathcal D_Y}(\mathcal B_{X|Y},\mathcal D_Y)\otimes\w_{Y/k}^\vee\cong\mathcal B_{X|Y}$, and all other $H^i(\mathbb D\mathcal B_{X|Y})$ vanish. Therefore $\mathcal B_{X|Y}$ is holonomic by \cref{thm-holonomic-criterion}.
    \end{example}
    \begin{example}\label{exp-not-holo-1}
        A trivial counter-example: on $\mathbb A_k^2$, the $D$-module $\mathcal D_X$ is not holonomic because very easily we can compute, by definition, that $H^{-2}(\mathbb D\mathcal D_X)=\mathcal D_X\otimes\w_{X/k}^\vee$, which is nonzero.
    \end{example}
    \begin{example}
        Still take $X=\mathbb A_k^2$. For $P\in\mathcal D_X$, write $\s(P)$ to be the symbol of $P$ in $\gr^F\mathcal D_X$ (replacing $\partial_i$ with $\xi_i$). Bernstein and Lunts showed that if $\s(P)$ satisfies certain conditions, then $\mathcal D_X/\mathcal D_X\cdot P$ is a simple, non-holonomic $\mathcal D_X$-module.
    \end{example}
    \begin{example}
        To extend the above example, suppose $X=\A^n_k$ and fix a left $A_n$-ideal $I$. By Proposition 2 in \cite{oaku-computational}, the characteristic variety of the $\mathcal D_X$-module $A_n/I$ is the zero locus of the ideal $\operatorname{int}(I)$ in $\A^{2n}_k$, where $\operatorname{int}(I)$ is the vector space (which turns out to be an ideal) spanned by $\{\operatorname{int}(P):P\in I\}$ where $\operatorname{int}(P)$ is obtained by taking the sum of monomials of maximal order in $\xi$ of $\s(P)$ (e.g., $\operatorname{int}(\partial_1^2+\partial_2\partial_1+x_1^{10}\partial_2)=\xi_2^2+\xi_1\xi_2$). Therefore, $A_n/I$ is holonomic if and only if the Krull dimension of $\operatorname{int}(I)$ is $n$.
    \end{example}
    \begin{remark}
        In \cref{exp-not-holo-1}, $\mathcal D_X=\mathcal D_X/\mathcal D_X\cdot 0$ is the $D$-module corresponding to the system of equations $0\cdot u=0$, whose solution space is infinite-dimensional. Indeed, holonomic $D$-modules are those corresponding to systems of partial equations with a finite-dimensional solution space. In general, given an ideal of $A_n$, this notion is captured by the \textbf{holonomic rank} of $I$, defined as $\dim_{k(\underline{x})}k(\underline{x})[\underline{\xi}]/k(\underline{x})[\underline{\xi}]\cdot\operatorname{int}(I)$. In some cases, this is the dimension of the solution space (c.f. 1.4.19 in \cite{sst-computational} which is a special case of 2.2.1 in one of the fundamental pieces in the theory of $D$-modules: Kashiwara's Master's thesis \cite{kashiwara-master}). By 1.4.9 in \cite{sst-computational}, if $A_n/I$ is holonomic then the holonomic rank of $I$ is finite, though the converse is false.
    \end{remark}
    \begin{example}
        Consider the heat equation $\partial_1u=\alpha\partial_2^2u$ where $\alpha\neq 0$ in $\A^2_k$. The $D$-module associated to it is $A_2/I$ where $I$ is $(\partial_1-\alpha\partial_2^2)$ and $\operatorname{int}(I)=(\alpha\xi_2^2)$. Therefore, $\ch(A_2/I)$ is a hypersurface in $\A^4_k$, which has dimension $3$, meaning $A_2/I$ is not holonomic. Indeed, using methods \mbox{\texttt{inw(I, \string{0, 0, 1, 1\string})}} and \mbox{\texttt{holonomicRank}} in the \mbox{\texttt{Dmodules}} package of \mbox{\texttt{macaulay2}}, we can compute $\operatorname{int}(I)$ and the holonomic rank of $I$ which is infinite, agreeing with the standard results in analysis. Similarly, let $J=(x_1\partial_1, x_2\partial_2+x_2^n)$ for any $n\neq 0$. Then $\operatorname{int}(J)=(x_1\xi_1, x_2\xi_2)$ and thus $A_2/J$ is holonomic. Again we can compute its holonomic rank which is $1$, the dimension of its solution space $\{c\exp(-x^{n-1}/(n-1))\}$.
    \end{example}


    \section{D-Modules on Flag Varieties}
    \subsection{Flag varieties and equivariant sheaves}
    We assume a basic understanding of algebraic groups and flag varieties. In this section let $k$ be an algebraically closed field of characteristic zero. Let $G$ be a connected semisimple linear algebraic group over $k$. Let $\g=T_eG$ be the Lie algebra of $G$. Denote by $\D$ and $\D^+$ the root system and the set of positive roots of $G$, by $\Pi$ the set of simple roots, by $Q$ and $Q^+$ the root and nonnegative root lattices, and by $P$ and $P^+$ the weight and nonnegative weight lattices. For a root $\alpha\in \D$, denote by $\alpha^\vee$ its coroot. Let $H$ be a fixed maximal torus in $G$ and $B$ a Borel subgroup containing $H$. The Lie algebras of $H$ and $B$ I denote by $\h$ and $\b$, which are Cartan and Borel subalgebras of $\g$. As a consequence of Chevalley's theorem, given a closed subgroup variety $P$ of $G$, the homogeneous space $G/P$ exists and is a smooth algebraic variety since $G$ is smooth. Moreover, $G/P$ is complete if and only if $P$ is parabolic (i.e., containing a Borel subgroup). Since $G$ is linear, $G/P$ is projective. In particular, $G/B$ is a smooth projective variety. For details see \cite{milne-groups}.
    \begin{definition}
        The \textbf{flag variety} of $G$ is the smooth projective variety $X=G/B$.
    \end{definition}
    Of course, standard results on algebraic groups show that $X$ can be identified with the set $\mathcal B$ of all Borel subgroups of $G$, on which $G$ acts by conjugation. Sheaves on $X$ are closely related to the representation theory of $G$.
    \newcommand{\pr}{\operatorname{pr}}
    \begin{definition}
        Let $G$ be any group scheme and $X$ a scheme over $X$. Suppose $\s:G\times X\to X$ is an action of $G$ on $X$. A \textbf{$G$-equivariant sheaf} on $X$ is an $\mathcal O_X$-module $\mathcal M$ with an isomorphism
        \[\varphi:\s^*\mathcal M\xrightarrow{\sim}\pr_2^*\mathcal M\]
        such that $\pr_{23}^*\varphi\circ(\id\times\s)^*\varphi=(m\times\id)^*\varphi$ as morphisms of quasicoherent modules on $G\times G\times X$ where $m:G\times G\to G$ the multiplication.
    \end{definition}
    If $X$ is the flag variety of the algebraic group $G$, we have an obvious action of $B$ on $X$ so this allows us to talk about $B$-equivariant sheaves on $X$. Let $G$ act on itself in the obvious way. The quotient $\pi:G\to X$ is a \textbf{$B$-torsor} in the category of schemes with the fppf topology. Indeed, given an open affine cover $\{U_i\}$ of $X$, we obtain an fppf covering $\{U_i\to X\}$. Now by 9.19 in \cite{milne-groups}, $G\times B\cong G\times_X G$ via $(g, b)\mapsto (g, gb)$ and by 4.43 in \cite{fantechi-fundamental}, $\pi:G\to X$ is a $B$-torsor. Again, apply 4.46 in \cite{fantechi-fundamental}, we see that
    \begin{proposition}\label{prop-B-equivariant}
        The category of quasicoherent sheaves on $X$ is equivalent to the category of $B$-equivariant quasicoherent sheaves on $G$.
    \end{proposition}
    Denote by $\Mod_{\mathcal O_B}(\mathcal O_G)$ the category of $B$-equivariant quasicoherent $\mathcal O_G$-modules. Notice that $B$ and $G$ are both affine, so we probably can write out a version of \cref{prop-B-equivariant} in terms of plain modules. I will follow some excellent introductions in \cite{ganev-notes-on-equivariant} to illustrate the results. Let's first translate the group scheme action into algebras. Write $G=\Spec R$. Then the group scheme structure on $G$ is equivalent to a \textbf{Hopf algebra} structure on $R$, which we recall is a $k$-algebra with a comultiplication map $\D:R\to R\otimes R$, a counit map $\ve:R\to k$ and an antipode $S:R\to R$ compatible with the $k$-algebra structure on $R$.
    \begin{definition}
        A \textbf{right $R$-comodule} is a vector space $M$ over $k$ with a linear map $\rho_M:M\to M\otimes R$ such that $(\id\otimes\;\D)\circ\rho_M=(\rho_M\otimes\id)\circ\rho_M$ and $(\id\otimes\;\ve)\circ\rho_M=\id$. Write $\cMod(R)$ as the category of $R$-comodules.
    \end{definition}
    It's immediate that the category is monoidal. Given a $k$-algebra $A$ that is also an $R$-comodule such that its algebra structures are comodule maps, denote by $\Mod_R(A)$ the full subcategory of $A$-modules in $\cMod(R)$ such that the module map $A\otimes_k M\to M$ is a map of $R$-comodules. Denote by $\rho_A:A\to A\otimes_k R$ the comodule structure on $A$. Let $p:A\to A\otimes_k R$ be $a\mapsto a\otimes 1$. We have functors $\rho_A^*:\Mod(A)\to \Mod(A\otimes R)$ and $p^*:\Mod(A)\to\Mod(A\otimes B)$ both sending an $A$-module $M$ to the vector space $M\otimes R$, but the action of $a\otimes r\in A\otimes R$ on $\rho_A^*M$ is given by Sweedler notations
    \[(a\otimes r)(m\otimes r')=(\rho_A(a)_{(1)}\cdot m)\otimes(\rho_A(a)_{(2)}rr')\]
    while the action on $p^*M$ is simply component-wise. These functors have left adjoints $\rho_{A,*}$ and $p_*$ resp., given by $N\mapsto N$ such that $a$ acts on $\rho_{A,*}N$ by $a\cdot n=\rho_A(a)n$, and on $p_*N$ as $a\cdot n=(a\otimes 1)n$.
    \begin{proposition}
        An $A$-module $M$ sits inside $\cMod(R)$, i.e., in $\Mod_R(A)$ if and only if we have an isomorphism $\psi:\rho_A^*M\xrightarrow{\sim}p^*M$ satisfying cocycle conditions similar to equivariant sheaves.
    \end{proposition}
    \begin{proof}
        Let $M\in\Mod_R(A)$. Then by definition, $A\otimes M\to M$ is a map of comodules, i.e., $\rho_M(a\cdot m)=\rho_A(a)\cdot\rho_M(m)$ for all $a\in A$ and $m\in M$. Notice that $M\otimes R=\rho_{A,*}p^*M$, so under this $A$-module structure $\rho_M$ is an $A$-module map $M\to \rho_{A,*}p^*M$. By adjunction, we get an $A\otimes R$-linear map $\psi:\rho_A^*M\to p^*M$ which is the identity on vector space levels. Thus, $\psi$ is an isomorphism. It's easy to check its cocycle conditions.
    \end{proof}
    Write $B=\Spec L$. Then the action of $B$ on $G$ is the same as a right $L$-comodule structure $R\to R\otimes L$ on $R$ such that the $k$-algebra map $R\otimes R\to R$ is an $L$-comodule homomorphism. We therefore have the following
    \begin{proposition}\label{prop-b-equiv-module-comodule}
        There is an equivalence of categories
        \[\Mod_{\mathcal O_B}(\mathcal O_G)\xrightarrow{\sim}\Mod_L(R)\]
    \end{proposition}
    \begin{example}
        Given a $G$-equivariant $\mathcal O_X$-module $\mathcal M$, there is a canonical $G$-module structure on $\Gamma(X, \mathcal M)$. Consider the coaction $\varphi^\#:\Gamma(X, \mathcal M)\to\Gamma(G, \mathcal O_G)\otimes \Gamma(X, \mathcal M)$ given by the $G$-equivariant structure $\varphi$. If we write $\varphi^\#(s)=\sum f_i\otimes s_i$ where $f_i\in\Gamma(G, \mathcal O_G)$ then the $G$-action on $s$ is given by $g\cdot s=\sum f_i(g)s_i$.
    \end{example}
    \begin{example}\label{exp-act-on-hom}
        At last, given two locally free $G$-equivariant sheaves $\mathcal M$ and $\mathcal N$ on $X$, we can define a $G$-action on $\Hom_k(\mathcal M,\mathcal N)$ as follows. Let $\s(g)$ be the composition of $g\times X\hookrightarrow G\times X$ and $\s$. Then there are isomorphism $\varphi_{\mathcal M}(g):\mathcal M\xrightarrow{\sim}\s(g)^*\mathcal M$ and $\varphi_{\mathcal N}(g):\mathcal N\xrightarrow{\sim}\s(g)^*\mathcal N$ induced by the $G$-equivariant structure. For a morphism $\psi:\mathcal M\to\mathcal N$, define 
        \[g\cdot\psi=\varphi_{\mathcal N}(g)^{-1}\circ\s(g)^*\psi\circ\varphi_{\mathcal M}(g)\]
        which is clearly a $G$-action since the $\varphi(g)$ are compatible with the group law.
    \end{example}
    


    \subsection{Equivariant vector bundles}\label{sec-equiv}
    I will quickly establish an equivariant version of the correspondence between locally free sheaves of finite rank on $X=G/B$ and equivariant vector bundles on $X$.
    \begin{definition}
        We say a vector bundle $V$ on $X$ is \textbf{$G$-equivariant} if it has a $G$-action such that for all $g\in G$, $x\in X$, the action $g:V_{x}\to V_{gx}$ is a linear isomorphism where $V_x=V\times_X\Spec \kappa(x)$ is the fiber at $x$.
    \end{definition}
    Let $\mathcal V$ be the sheaf of sections of $V$. Then $\mathcal V$ is $G$-equivariant if and only if $V$ is $G$-equivariant. To see this, the fiber of $\pr_2^*\mathcal V$ at $(g, x)\in G\times X$, i.e. its stalk at $(g, x)$ tensoring the residue field, is precisely $V_x$, and the fiber of $\s^*\mathcal V$ is $V_{gx}$ (simply evaluate sections of $V$ at $(g, x)$). Then the equivalence is evident. Note that we can construct induced representations of $B$-modules on $G$. The following theorem identifies $G$-equivariant vector bundles on flag varieties with $B$-modules via induced representations of the latter.
    \begin{proposition}\label{prop-vec-bundle}
        Given a $B$-module $M$, there is a $G$-equivariant vector bundle $V$ on $X$ such that $V_x=M$ where $x$ is the point in $X=\mathcal B$ corresponding to the Borel subgroup $B$.
    \end{proposition}
    \begin{proof}
        Let $G\times M$ be the trivial vector bundle on $G$. Then $B$ acts on $G\times M$ by $b\cdot (g, m)=(gb^{-1},b\cdot m)$. Let $V_M$ be the quotient $(G\times M)/B$ (everything is affine in our case so the quotient exists), which is a vector bundle $\pi_M:V_M\to X$ sending $[(g, m)]$ to $gB\in X$. The canonical action of $G$ on $G\times M$ induces a $G$-action on $V_M$ given by
        \[g\cdot [(g', m)]=[(gg', m)]\]
        which is clearly well-defined since the action of $b$ is on the right in the first coordinate. Locally, we see that $g:V_{M, g'B}\to V_{M, gg'B}$ is simply the identity map on $M$ so a linear isomorphism. Thus $V$ is $G$-equivariant on $G/B$ and clearly $V_{M,x}=M$ for $x=1B$.
    \end{proof}
    \begin{remark}\label{rmk-tensor-sheaf}
        Let $\mathcal V_M$ be the sheaf of sections on $V_M$. Indeed, given two $B$-modules $M_1, M_2$ and any open $U\subseteq X$, we have a map $\Gamma(U,\mathcal V_{M_1})\otimes\Gamma(U, \mathcal V_{M_2})\to \Gamma(U,\mathcal V_{M_1\otimes M_2})$ given by $f_1\otimes f_2\mapsto (g\mapsto f_1(g)\otimes f_2(g))$. This gives us an isomorphism $\mathcal V_{M_1}\otimes\mathcal V_{M_2}\cong \mathcal V_{M_1\otimes M_2}$. Similar relations hold for symmetric and exterior products.
    \end{remark}
    From now on, we assume $G$ is simply connected and write $B=HN$ where $N$ is the unipotent radical of $B$ and $H$ the maximal torus fixed at the beginning. Every integral weight $\lambda$ of $\g$ gives rise to a character $e^\lambda$ of $H$. A character of $B$ maps $N$ to the unipotent radical of $\mathbb G_m$, which is trivial. Therefore, one-dimensional $B$-modules are equivalent to characters of $H$. For any integral weight $\lambda$ defining a character of $H$, by the previous constructions we obtain a vector bundle $V_\lambda$ and a $G$-equivariant sheaf $\mathcal V_\lambda$. Write $\mathcal L(\lambda)=\mathcal V_\lambda$.
    \begin{example}\label{exp-sl2-L}
        Let us compute all constructions we mentioned for $G=SL_2$. Let $B$ be the Borel subgroup of all upper triangular matrices and $H$ the maximal torus of diagonal matrices. Then it's easy to see that $G/B$ is $\mathbb P^1$. Let $\rho$ be the half sum of all positive roots, so the character $e^\rho$ sends $\operatorname{diag}(a, a^{-1})$ to $a$. It is immediate from the definition that the character group $\widehat{H}=P=\Z\rho$. Note that $\pi:G\to X$ is sends
        $\begin{bsmallmatrix}
          a & b\\
          c & d  
        \end{bsmallmatrix}$ 
        to $[a:c]$. Cover $\mathbb P^1$ by open affines $U_0$ and $U_\infty$ containing $[0:1]$ and $[1:0]$ respectively. Let $z=a/c$ and $w=c/a$ be the local coordinates on $U_0$ and $U_\infty$. Then points in $U_0$ and $U_\infty$ are classes with representatives
        \[\begin{bmatrix}
            z & -1\\
            1 & 0
        \end{bmatrix}, \quad \begin{bmatrix}
            1 & 0\\
            w & 1
        \end{bmatrix}\]
        The preimage of $U_0$ and $U_\infty$ in the vector bundle $V_{n\rho}$ are trivial bundles $U_0\times k$ and $U_1\times k$, and they glue by
        \[[(z, t)]=\left[\left(\begin{bmatrix}
            z & -1\\
            1 & 0
        \end{bmatrix}, t\right)\right]=\left[\left(\begin{bmatrix}
            z & -1\\
            1 & 0
        \end{bmatrix}\begin{bmatrix}
            w & 1\\
            0 & z
        \end{bmatrix}, \begin{bmatrix}
            w & 1\\
            0 & z
        \end{bmatrix}^{-1}t\right)\right]=[(w, z^{n}t)]\]
        Thus, we see that $\mathcal L(n\rho)=\mathcal O_X(-n)$. Now the $G$-action on $V_{n\rho}$ is given by
        \[\gamma\cdot [(z, t)]=\gamma\cdot \left[\left(\begin{bmatrix}
            z & -1\\
            1 & 0
        \end{bmatrix}, t\right)\right]=\left[\left(\begin{bmatrix}
            \gamma\cdot z & -1\\
            1 & 0
        \end{bmatrix},\begin{bmatrix}
            j(\gamma, z) & -c\\
            1 & j(\gamma, z)^{-1}
        \end{bmatrix} t\right)\right]=\left[\left(\gamma\cdot z, j(\gamma, z)^nt\right)\right]\]
        where $\gamma=\begin{bsmallmatrix}
            a & b\\
            c & d
        \end{bsmallmatrix}$, $\gamma\cdot z=(az+b)/(cz+d)$ and $j(\gamma, z)=cz+d$.
    \end{example}
    \begin{theorem}\label{thm-invertible-sheaf-weight}
        \begin{enumerate}[\normalfont(i)]
            \item If $-\lambda\in P^+$, then $H^i(X, \mathcal L(\lambda))=0$ for all $i>0$.
            \item If for all $\alpha\in\D^+$, we have $\alpha^\vee(\lambda)\leqslant 0$, then $\mathcal L(\lambda)$ is generated by global sections.
            \item The invertible sheaf $\mathcal L(\lambda)$ is ample if and only if for all $\alpha\in\D^+$, we have $\alpha^\vee(\lambda)<0$.
        \end{enumerate} 
    \end{theorem}
    \begin{proof}
        See 4.5 and 4.4 in \cite{janzten-groups} for proofs. The first statement is called the Kempf vanishing theorem, which is true even if $k$ has a positive characteristic.
    \end{proof}
    And without proof we state the Borel-Weil-Bott theorem, the origin of geometric representations. Let $\rho$ be the half sum of all positive roots and $W=N_G(H)/H$ be the Weyl group of $G$. Recall for any $w\in W$, it has a length $l(w)$ equal to the minimal number of reflections $\s_\alpha:\beta\mapsto 2(\alpha, \beta)/(\alpha,\alpha)\alpha$ required to express $w$. We say $\lambda$ is \textbf{singular} if there is some root $\alpha$ such that $\alpha^\vee(\lambda-\rho)=0$.
    \begin{definition}
        We define $w\star \lambda=w(\lambda-\rho)+\rho$.
    \end{definition}
    \begin{theorem}[Borel-Weil-Bott]\label{thm-borel-weil-bott}
        If $\lambda$ is singular, all cohomology groups of $\mathcal L(\lambda)$ vanish. Otherwise, there is a \textup{unique} $w\in W$ such that $-w\star \lambda\in P^+$. In this case,
        \[H^i(X, \mathcal L(\lambda))=\begin{cases}
            L^-(w\star\lambda), & i=l(w)\\
            0, &\text{otherwise}
        \end{cases}\]
        where $L^-(\mu)$ denotes the lowest weight module of $G$ with lowest weight $\mu$.
    \end{theorem}
    \begin{remark}
        Some authors choose to define $\mathcal L(\lambda)$ using $-\lambda$, and statements here must be adjusted accordingly.
    \end{remark}
    \begin{example}
        We can verify Borel-Weil-Bott for $G=SL_2$, $X=\mathbb P^1$. In \cref{exp-sl2-L}, we showed that $\mathcal L(n\rho)=\mathcal O_X(-n)$. If $n\leqslant 0$, $\mathcal O_X(-n)$ has global sections spanned by homogeneous polynomials $x_0^kx_1^{-n-k}$. Restricted to $U_0$ they become $1,z,\dots, z^{-n}$. Then as $V_{n\rho}|_{U_0}$ is trivial, sections are given by $s:x\mapsto [(x, f)]$ where $f$ is a polynomial of degree $\leqslant -n$. The action $(g\cdot s)(x)=g\cdot s(g^{-1}x)$ is given by
        \[\begin{bmatrix}
            a & b\\
            c & d
        \end{bmatrix}z=(dz-b)^k(-cz+a)^{-n-k}=(dx_0-bx_1)^k(-cx_0+ax_1)^{-n-k}\]
        Plugging $E, F$ and $H$ into the matrix, one deduces that $\Gamma(X, \mathcal O_X(-n))$ is the highest weight module of weight $-n\rho$ and the lowest weight module of weight $n\rho$. On the other hand, $n\leqslant 0$ and $1\star n\rho\in -P^+$, which gives the same result by Borel-Weil-Bott. When $n=1$, that $n\rho$ is singular so $H^i(X, \mathcal O_X(-1))=0$ for all $i$ which coincides with the result in algebraic geometry. If $n>1$, we have
        \[-1\star n\rho=\rho-n\rho+\rho=(2-n)\rho\in -P^+\]
        Then $H^1(X,\mathcal L(n\rho))=H^1(X,\mathcal O_X(-n))=L^-((2-n)\rho)$ which also agrees with standard results in algebraic geometry.
    \end{example}
    \subsection{Centers and invariants}
    Still let $\g$ be the Lie algebra of $G$. We consider the universal enveloping algebra $U(\g)$. Let $\mathfrak Z$ be its center. Since it commutes with $\h$ and $\mathfrak n$, $\mathfrak Z$ acts on any highest weight module $M$ with highest weight $\lambda$ by a character $f_\lambda$. Therefore we have a map $\varphi_z:\h^*\to k$ sending a weight $\lambda$ to $f_\lambda(z)$. Recall the algebra $U(\g)$ has a decomposition
    \[U(\g)=U(\h)\oplus(\mathfrak n^-U(\g)+U(\g)\mathfrak n)\]
    where $\g=\mathfrak n^-\oplus\h\oplus\mathfrak n$ and the previously fixed Borel subalgebra is given by $\b=\h\oplus\mathfrak n$. Let $p$ be the projection of $U(\g)$ to $U(\h)$. Since $\h$ is abelian, $U(\h)=\Sym\h=k[\h^*]$. Indeed, if $z\in\mathfrak Z$, $z\in U(\h)\oplus(\mathfrak n^-U(\g)\cap U(\g)\mathfrak n)$. Notice that every highest weight module is a quotient of the \textbf{Verma module} $M_\lambda$ given by
    \[M_\lambda=U(\g)\otimes_{U(\b)}k_\lambda\]
    where $k_\lambda$ is the vector space spanned by a vector $v$ such that $\h$ acts by $\lambda$ and $\mathfrak n$ acts trivially. As $\mathfrak n$ kills $v$, we see that $\varphi_z(\lambda)=p(z)(\lambda)$ where $p(z)\in k[\h^*]$. Thus, we get an algebra homomorphism $\varphi:\mathfrak Z\to U(\h)$ sending $z$ to $\varphi_z$.
    \begin{definition}
        The \textbf{Harish-Chandra homomorphism} $\gamma:\mathfrak Z\to U(\h)$ is given by $\gamma=\tau\circ\varphi$ where $\tau:h\mapsto h-\rho(h)\cdot 1$ is an automorphism of $U(\h)$.
    \end{definition}
    \begin{theorem}[Harish-Chandra]
        The Harish-Chandra homomorphism satisfies:
        \begin{enumerate}[\normalfont(i)]
            \item The map $\gamma$ is independent of different $\D^+$ and is injective.
            \item For every $w\in W$, we have $\gamma(z)(w\lambda)=\gamma(z)(\lambda)$, i.e., $\gamma(z)\in U(\h)^W$.
            \item The map $\gamma:\mathfrak Z\to U(\h)^W$ is an isomorphism.
        \end{enumerate}
    \end{theorem}
    \begin{proof}
        See 23.2 and 23.3 in \cite{humphreys-lie}.
    \end{proof}
    \begin{definition}
        Define the central character $\chi_\lambda$ associated to $\lambda$ to be $\chi_\lambda(z)=\gamma(z)(\lambda)$.
    \end{definition}
    \begin{proposition}\label{prop-hc-char}
        Any character $\mathfrak Z\to k$ is $\chi_\lambda$ for some $\lambda\in\h^*$. Two central characters $\chi_\lambda$ and $\chi_\mu$ are the same if and only if there is some $w\in W$ such that $w\star\lambda=\mu$.
    \end{proposition}
    \begin{proof}
        The claims follow from the observation that since $\mathfrak Z=\Sym(\h)^W$, the GIT quotient of $\h^*=\Spec\Sym(\h)$ by $W$ is precisely $\Spec \mathfrak Z$; and the quotient map $\pi:\h^*\to \Spec \mathfrak Z$ is induced by the injective homomorphism $\gamma:\mathfrak Z\to \Sym(\h)$, so $\pi$ is surjective. By a theorem of Nagata, $\Sym(\h)^W$ is finitely generated. Therefore, $\Spec \mathfrak Z$ is locally of finite type over $k$, meaning the induced map given by composition with $\pi$ on $k$-points $\h^*(k)\to(\Spec \mathfrak Z)(k)$ is surjective. But on global sections level, the $k$-points of $\Spec \mathfrak Z$ are precisely central characters, and $k$-points of $\h^*$ are just weights of $\g$. Thus, all central characters are of the form $\gamma(z)(\lambda)$. The second claim is immediate from our construction.
    \end{proof}
    It remains to study the algebra $U(\h)^W=\Sym(\h)^W$. Now, for the algebraic group $G$ and any $G$-module $V$, we can define the action of $G$ on $k[V]$ by $g\cdot f=f\circ g^{-1}$. Again, by the theorem of Nagata, if $G$ is reductive (in our case semisimple), $k[V]^G$ is finitely generated over $k$. Let $G$ act on $\g$ by the adjoint representation. Then
    \begin{theorem}[Chevalley's restriction theorem]
        The map $k[\g]\to k[\h]$ restricted to $k[\g]^G$ is an isomorphism $k[\g]^G\xrightarrow{\sim}k[\h]^W$.
    \end{theorem}
    \begin{proof}
        See 10.1.1 in \cite{htt-d-modules}.
    \end{proof}
    Let $\mathscr N$ be the \textbf{nilpotent cone} of $\g$ which we recall to be the set of nilpotent elements in $\g$. The set is an irreducible closed subvariety of $\g$. We may write $\det(t-\ad(x))$ as $t^n+\sum_{i=0}^{n-1}f_i(x)t^i$. By nilpotency, $\mathscr N$ is the closed subvariety defined by $f_1,\dots, f_{n-1}$. But then $\mathscr N=\operatorname{Ad}(G)(\mathfrak b\cap\mathscr N)=\operatorname{Ad}(G)\mathfrak n$ which is irreducible. Since $\mathfrak b$ is a $B$-module under the adjoint representation, we can construct a vector bundle $V_{\mathfrak b}$ as in \cref{prop-vec-bundle}. We can then define morphisms $\rho_0:V_{\mathfrak b}\to\g$ and $\theta:V_{\mathfrak b}\to\h$ by
    \[\rho_0([(g, x)])=\operatorname{Ad}(g)x,\quad\theta([(g, x)])=p(x)\]
    where $p$ is the projection from $\mathfrak b=\h\oplus\mathfrak n$ to $\h$. Note that $\theta^{-1}(0)$ is the cotangent bundle on the flag variety $X$ since the space orthogonal to $\mathfrak b$ is $\mathfrak n$ in $\g$ with respect to the Killing form. Therefore, $\mathfrak n$ can be identified with $(\g/\mathfrak b)^*$ via the Killing form. But the latter is $T^*X$, and the former is $\theta^{-1}(0)$. We quote the following theorems without proof (see 8.2 in \cite{dixmier-uea})
    \begin{theorem}\label{thm-nilpotent-cone}
        Let $Y_+$ be the subspace of $Y(\g)=\Sym(\g)^G$ consisting of sums of homogeneous elements of positive degrees. Then $\mathscr N$ is defined by the ideal $\Sym(\g)\cdot Y_+$ which is prime. Moreover, $\mathscr N$ is a normal variety containing an open dense $G$-orbit.
    \end{theorem}
    \begin{theorem}[Kostant]\label{thm-kostant}
        The morphism $\rho_0:\theta^{-1}(0)=T^*X\to\mathscr N$ is a resolution of singularity.
    \end{theorem}

    
    \subsection{Beilinson-Bernstein}
    In this section we will state the Beilinson-Bernstein theorem. The first goal is to understand the representation theoretic nature of the sheaf of differential operators $\mathcal D_X$ on flag varieties. Let $X$ be any smooth variety with an action of a linear algebraic group $G$ on $X$ and $\g$ be the Lie algebra of $G$. Given a locally free sheaf $\mathcal V$ of finite rank on $X$ with a $G$-equivariant structure given by $\varphi:\s^*\mathcal V\to\pr_2^*\mathcal V$. For any $x\in\g$, define $\partial_x\in\End_k(\mathcal V)$ by
    \[\partial_xs=(i^*\circ\varphi^{-1})((x\boxtimes \id)\cdot\varphi(\s^*s))\]
    where $i:X\hookrightarrow G\times X$ is given by $x\mapsto(1, x)$, and the action of $x$ in the tensor product is regarded as its image under the canonical map $\g\to\Gamma(G, \mathcal D_G)$ which is a invariant $G$-invariant vector field. Recall that $U(\g)$ can be identified with the space of left invariant vector fields on $G$. Define $\tau_{\mathcal V}$ to be the map $x\mapsto\partial_x$.
    \begin{lemma}
        Suppose $\mathcal V$ is a locally free $G$-equivariant sheaf on $X$. Then for any $x\in \g$, $f\in\mathcal O_X$ and $s\in \mathcal V$, we have
        \[\tau_{\mathcal V}(x)(f\cdot s)=f\cdot \tau_{\mathcal V}(x)(s)+\tau_{\mathcal O_X}(f)\cdot s\] 
    \end{lemma}
    \begin{proof}
        This is immediate since the action of $x$ is a left invariant derivation and $\varphi$ is $\mathcal O_{G\times X}$-linear.
    \end{proof}
    We can therefore extend $\tau_{\mathcal V}$ to a ring homomorphism $U(\g)\to \Gamma(X, \mathcal D_X^{\mathcal V})$. To see this, compute that for any $f, g\in \mathcal O_X$, we have $[f, \tau_{\mathcal V}(x)](s)=f\cdot \tau_{\mathcal V}(x)(s)-f\cdot \tau_{\mathcal V}(x)(s)+\tau_{\mathcal O_X}(f)\cdot s=\tau_{\mathcal O_X}(f)\cdot s\in\Gamma(X, F_0\mathcal D_X^{\mathcal V})$. Therefore, $\Gamma(X, \mathcal D_X^{\mathcal V})$ becomes a $U(\g)$-module.

    Still let $G$ be a simply connected, connected semisimple algebraic group over $k$ an algebraically closed field of characteristic zero. Denote by $\D$ the root system, $\D^+$ the set of positive roots and $P$ be the weight lattice of $G$. By $\rho$ we mean the half sum of positive roots. For any $\lambda\in P$, let $\mathcal L(\lambda)$ be the $G$-equivariant invertible sheaf on $X=G/B$. Let $\mathcal D_\lambda$ be the t.d.o. $\smash{\mathcal D_X^{\mathcal L(\lambda+\rho)}}$. Let $\tau_\lambda$ denote the map we defined above associated to $\mathcal L(\lambda+\rho)$, and $\chi_\lambda$ the central character associated to $\lambda$. Let $\mathfrak Z$ be the center of $U(\g)$. We have adjoint functors $\mathcal D_\lambda\otimes_{U(\g)}-$ and $\Gamma(X, -)$ between $\Mod(\mathcal D_\lambda)$ and $\Mod(\g)$ by Hom-tensor product adjunction. The first key theorem in \cite{bb-original} describes $\tau_\lambda$ explicitly.
    \begin{theorem}\label{thm-bb-1}
        Given $\lambda\in P$, for any $z\in \mathfrak Z$, $\tau_\lambda(z)$ is the multiplication by $\chi_\lambda(z)$. Moreover, the homomorphism $\tau_\lambda$ is surjective with kernel $U(\g)\cdot\ker\chi_\lambda$.
    \end{theorem}
    The second theorem shows the $\mathcal D_\lambda$-affinity of $X$ for special $\lambda$.
    \begin{theorem}\label{thm-bb-2}
        For $\lambda\in P$,
        \begin{enumerate}[\normalfont(i)]
            \item If $\alpha^\vee(\lambda)\leqslant 0$ for all $\alpha\in\D^+$,\footnote[2]{Usually people state this theorem using the term ``dominant weight''. But the definition of $\mathcal L(\lambda)$ requires us to use anti-dominant weights, while other authors might define $\mathcal L(\lambda)$ using $-\lambda$. So to avoid confusions, I choose not to use these terms.} then $\Gamma(X, -)$ is exact.
            \item If $\alpha^\vee(\lambda)< 0$ for all roots $\alpha\in\D^+$, then any $\mathcal D_\lambda$-module $\mathcal M$ is generated by global sections.
        \end{enumerate}
        In particular, if $\alpha^\vee(\lambda)<0$ for all $\alpha\in\D^+$, then $X$ is $\mathcal D_\lambda$-affine.
    \end{theorem}
    \begin{remark}
        Please note that we are proving a weaker version of the original theorem in \cite{bb-original}. Our construction of $\mathcal D_\lambda$ relies on the invertible sheaf $\mathcal L(\lambda)$, which can be constructed only if $\lambda\in P$. In \cref{sec-lie-algebroid}, we will discuss the general construction in \cite{bb-original, beilinson-janzten} for all $\lambda\in\h^*$.
    \end{remark}
    Denote by $\Mod(\g_\lambda)$ the category of $U(\g)$-modules with central character $\chi_\lambda$. 
    \begin{corollary}
        For $\lambda\in P$, if $\alpha^\vee(\lambda)< 0$ for all roots $\alpha\in\D^+$, then the functor $\Gamma(X, -)$ induces equivalences
        \[\Mod(\mathcal D_\lambda)\xrightarrow{\sim}\Mod(\g_\lambda),\quad\Mod_c(\mathcal D_\lambda)\xrightarrow{\sim}\Mod_f(\g_\lambda)\]
        with the quasi-inverse $\mathcal D_\lambda\otimes_{U(\g)}-$.
    \end{corollary}
    \begin{proof}
        By \cref{thm-bb-1} $\Gamma(X, \mathcal D_\lambda)$ is a quotient of $U(\g)$, so the functors $\mathcal D_\lambda\otimes_{U(\g)}-$ and $\mathcal D_\lambda\otimes_{\Gamma(X, \mathcal D_\lambda)}-$ are isomorphic, and we are done by \cref{prop-d-affinity} and \cref{lem-coh-fin-equiv}.
    \end{proof}





    \subsection{Proof of the theorems}
    The general idea in the proof of \cref{thm-bb-1} is to first compute the global sections of $\gr^F\mathcal D_\lambda$ using Kostant's theorem, compute the action of the center of $U(\g)$ on $\Gamma(X, \mathcal D_\lambda)$ and then show that the filtration $\Gamma(X, F_\bullet\mathcal D_\lambda)$ is the same as the PBW filtration of $U(\g)/U(\g)\cdot\ker\chi_\lambda$ after passing to their associated graded rings. To prove \cref{thm-bb-2}, we wish to understand what $\mathcal M$ is inside $\mathcal M\otimes_{\mathcal O_X} \mathcal L(\mu)\otimes_kL^+(-\mu)$ for $\mu\in -P^+$, so that we could study maps on cohomologies of $\mathcal D_\lambda$-modules by the induced maps on cohomologies of these tensor products, which are better understood due to the ampleness of $\mathcal L(\mu)$. We start by proving \cref{thm-bb-1}. Indeed, by \cref{lem-gr-iso-is-iso}, we can prove it on the level of associated graded algebras. This is very convenient since we know that $\gr^F\mathcal D_\lambda\cong \Sym\Theta_X$ by definition. On the other hand, in $U(\g)$ we have a PBW filtration, and we showed that $\tau_\lambda(\g)\subseteq \Gamma(X, F_1\mathcal D_\lambda)$. The map $\tau_\lambda:U(\g)\to \Gamma(X, \mathcal D_\lambda)$ respects the filtration. Since $\Gamma(X, -)$ is left exact, $\Gamma(X, F_l\mathcal D_\lambda)/\Gamma(X, F_{l-1}\mathcal D_\lambda)$ sits inside $\Gamma(X, \gr^{\mathit{PBW}}_l\mathcal D_\lambda)$. Therefore, $\tau_\lambda$ induces
    \[\gr\tau_\lambda:\gr^{\mathit{PBW}} U(\g)\to\Gamma(X, \gr^F\mathcal D_\lambda)\]
    Note that $\gr^{\mathit{PBW}} U(\g)\cong\Sym(\g)$. Identifying $\Sym(\g)$ with $\Gamma(\g^*,\mathcal O_{\g^*})$, and $\Gamma(X, \gr^F\mathcal D_\lambda)\cong\Gamma(X, \pi_*\mathcal O_{T^*X})=\Gamma(T^*X,\mathcal O_{T^*X})$. The map $\gr\tau_\lambda:\Gamma(\g^*,\mathcal O_{\g^*})\to \Gamma(T^*X,\mathcal O_{T^*X})$ is in fact the pullback of the moment map $T^*X\xrightarrow{\rho_0}\mathscr N\hookrightarrow\g\xrightarrow{\sim}\g^*$ where $\rho_0$ is constructed before \cref{thm-kostant}. This is not hard to verify since $\tau_\lambda$ is essentially defined by the action of $G$ on itself.
    \begin{proposition}\label{prop-gr-tau}
        Let $Y_+$ be the subspace of $Y(\g)=\Sym(\g)^G$ consisting of sums of homogeneous elements of positive degrees. The homomorphism $\gr\tau_\lambda$ is surjective with kernel $\Sym(\g)\cdot Y_+$.
    \end{proposition}
    \begin{proof}
        Identify $\g$ with $\g^*$ via the Killing form as usual. The map $\gamma$ surjects into the nilpotent cone $\mathscr N$ of $\g$. Write $\gamma=\gamma_1\circ\gamma_2$ where $\gamma_1:T^*X\to\mathscr N$ and $\gamma_2:\mathscr N\hookrightarrow\g^*$. The morphism $\gamma_1:T^*X\to\mathscr N$ is then a resolution of singularity by \cref{thm-kostant}. Therefore, since $\mathscr N$ is normal, and $\gamma_1$ is birational and finite, $\gamma_{1, *}\mathcal O_{T^*X}=\mathcal O_{\mathscr N}$. Now as $\gamma_1$ is surjective, we have an isomorphism $\Gamma(\mathscr N, \mathcal O_{\mathscr N})\cong \Gamma(T^*X, \mathcal O_{T^*X})$. The kernel of the closed immersion $\gamma_2$ is the ideal sheaf $\mathcal I_{\mathscr N}$ of $\mathscr N$, and the induced map $\Gamma(\g^*, \mathcal O_{\g^*})\to \Gamma(\mathscr N,\mathcal O_{\mathscr N})$ is surjective. The kernel of this map is the global section of $\mathcal I_{\mathscr N}$ which by \cref{thm-nilpotent-cone} is the ideal generated by $S_+$.
    \end{proof}
    \begin{proposition}
        For any $z\in\mathfrak Z$, $\tau_\lambda(z)=\chi_{\lambda}(z)\id$.
    \end{proposition}
    \begin{proof}
        Since $\mathfrak Z=U(\g)^G$, $\tau_\lambda(z)$ is also $G$-invariant. There is a dense $G$-orbit in $\mathscr N$, so the $G$-invariant part of $\Gamma(\mathscr N,\mathcal O_{\mathscr N})$ are the constants. In this case, $\Gamma(X, \gr^F \mathcal D_\lambda)^G=k$. The representations $\Gamma(X, F_{l-1}\mathcal D_\lambda)$, $\Gamma(X, F_{l}\mathcal D_\lambda)$ and $\Gamma(X, \gr_l^F\mathcal D_\lambda)$ of $G$ decomposes into direct sum of $G$-invariant parts and their complements. Therefore, taking the $G$-invariant parts we get an exact sequence
        \[0\to\Gamma(X, F_{l-1}\mathcal D_\lambda)^G\to\Gamma(X, F_{l}\mathcal D_\lambda)^G\to\Gamma(X, \gr_l^F\mathcal D_\lambda)^G\]
        Since $X$ is projective, the $G$-invariant global sections of $F_0\mathcal D_\lambda=\mathcal O_X$ is $k$. By our previous argument, $\Gamma(X, \gr^F_0\mathcal D_\lambda)^G=k$ and $\Gamma(X, \gr^F_l\mathcal D_\lambda)^G=0$ for all $l>0$ by directness. Via an inductive argument, one sees that $\Gamma(X, F_{l}\mathcal D_\lambda)^G=k$ for all $l$, and therefore $\Gamma(X, \mathcal D_\lambda)^G=k$.

        Now we evaluate $\tau_\lambda(z)$ on some nonzero section $s\in\mathcal L(\lambda+\rho)$ (here we see $\mathcal D_X^{\mathcal L}$ as $\mathcal D_X(\mathcal L,\mathcal L)$). Let $v$ be a nonzero element in the fiber at $1B\in X$ of the vector bundle associated to $\mathcal L(\lambda+\rho)$. Let $s$ be the section on the open $N^-B/B$ such that $s(uB)=uv$ for all $u\in N^-$. Then by construction, $\tau_\lambda(h)s=(\lambda+\rho)(h)s$ and $\tau_\lambda(a)s=0$ for all $a\in\mathfrak n^-$. Decompose $z\in\mathfrak Z$ as $z=u_1+u_2$ where $u_1\in U(\h)$ and $u_2\in U(\g)\mathfrak n^-$. Then we have
        \[\tau_\lambda(z)s=\tau_\lambda(u_1)s=(\lambda+\rho)(u_1)s\]
        But $(\lambda+\rho)(u_1)$ is precisely $\chi_\lambda(z)$, completing the proof.
    \end{proof}
    \begin{proof}[Proof of \cref{thm-bb-1}]
        We will show that
        \[0\to J_l=U(\g)\cdot\ker\chi_\lambda\cap \mathit{PBW}_lU(\g)\to \mathit{PBW}_lU(\g)\to\Gamma(X, F_l\mathcal D_\lambda)\to 0\]
        is an exact sequence for all $l$ through a diagram chasing. To simplify the notations, I will drop $\mathit{PBW}$ and write instead $F_\bullet U(\g)$ for the filtration on $U(\g)$. I will follow \cite{htt-d-modules} to write $I_l=\ker\chi_\lambda\cap F_lU(\g)$, 
        \[J_l=\sum_{m+n=l}F_mU(\g)I_n\]
        and $K_l$ the $l$th homogeneous component of $\Sym(\g)\cdot Y_+$. The following diagram is commutative because inclusions and $\tau_\lambda$ respect gradings.
        \[
            \begin{tikzcd}
                & 0 \arrow[d]                 & 0 \arrow[d]                                       & 0 \arrow[d]                                                &   \\
            0 \arrow[r] & J_{l-1} \arrow[d] \arrow[r] & F_{l-1}U(\g) \arrow[d] \arrow[r, "\tau_\lambda"]  & {\Gamma(X, F_{l-1}\mathcal D_\lambda)} \arrow[d] \arrow[r] & 0 \\
            0 \arrow[r] & J_l \arrow[d] \arrow[r]     & F_lU(\g) \arrow[d] \arrow[r, "\tau_\lambda"]      & {\Gamma(X, F_{l}\mathcal D_\lambda)} \arrow[d] \arrow[r]   & 0 \\
            0 \arrow[r] & K_l \arrow[d] \arrow[r]     & \Sym(\g)_l \arrow[d] \arrow[r, "\gr\tau_\lambda"] & {\Gamma(X, \gr_l^F\mathcal D_\lambda)} \arrow[r]           & 0 \\
                        & 0                           & 0                                                 &                                                            &  
            \end{tikzcd}   
        \]
        Suppose the first row is exact. The last row is always exact by \cref{prop-gr-tau}. The second column is trivially exact and the third column is exact by $\Gamma(X, -)$. Now if the first column is exact, then by an easy diagram chasing we see that the second row is exact. For $l>0$, degree $\leqslant l-1$ elements of $I_l$ die in $\Sym(\g)_l$ and $I_l\subseteq \ker\chi_\lambda\subseteq \mathfrak Z=U(\g)^G$. Therefore, $J_l$ does map to $K_l$, and the kernel of this map is precisely $J_{l-1}$. To see surjectivity, taking the $G$-invariant part of the exact sequence $F_lU(\g)\to\Sym(\g)_l\to 0$. Thus for any $x\in\Sym(\g)_l^G$ there is a preimage $y_1\in\mathfrak Z\cap F_lU(\g)$. Since $F_lU(\g)^G\to \Sym(\g)_l^G$ kills constants, $x$ also has the preimage $y=y_1-\chi_\lambda(y_1)\in\ker\chi_\lambda$. Thus, $y\in I_l\subseteq J_l$ maps to $x\in K_l$.

        Now for $l=0$, we have $F_0U(\g)=k$, $\Gamma(X, F_0\mathcal D_\lambda)=k$ and $I_0=0$. Therefore, we have an exact sequence
        \[0\to J_0\to F_0U(\g)\to \Gamma(X, F_0\mathcal D_\lambda)\to 0\]
        which completes the proof by the inductive step above.
    \end{proof}
    \begin{example}
        We compute the map $\tau_\lambda$ for $G=SL_2$. As before, $X=G/B=\mathbb P^1$ and let $U_0$ and $U_\infty$ be the standard affine open charts. Then $\mathcal D_{n\rho}|_{U_0}=\mathcal D_{U_0}$ and similarly for $\mathcal D_{n\rho}|_{U_\infty}$ since $\mathcal O(-n-1)$ is trivial on affine opens. The sheaf $\mathcal D_{n\rho}|_{U_0}$ is generated by $z$ and $\partial_z$ with $[\partial_z, z]=1$ and similarly $\mathcal D_{n\rho}|_{U_\infty}$ is generated by $w$ and $\partial_w$ with $[\partial_w, w]=1$. For convenience let $U=U_0\cap U_\infty$. On $\Gamma(U, \mathcal D_X)$, it is easy to compute that $\partial_w=-z^2\partial_w$. Since $\mathcal L((n+1)\rho)=\mathcal O_X(-n-1)$ is trivial on affine opens, we obtain an isomorphism $\Gamma(U, \mathcal O_X)\to\Gamma(U,\mathcal O_X)$ sending $1$ to $z^{-n-1}$ given by transition maps from $U_\infty$ to $U_0$. In the t.d.o., $\partial_w$ is thus identified with $-z^2\partial_z-(n+1)z$. Moreover, we see that
        \[\varphi((e\otimes 1)\cdot\varphi^{-1}(\s^*p))=\s^*(-\partial_zp)\]
        for any section $p$ of $\mathcal O_X(-n-1)$ on $U_0$. Thus, $\tau_{n\rho}$ sends $e\in U(\mathfrak{sl}_2)$ to $-\partial_z$, and one could verify that $\tau_{n\rho}(f)=z^2\partial_z$ and $\tau_{n\rho}(h)=-2z\partial_z$.
    \end{example}
    To prove \cref{thm-bb-2}, we will need some extra maps. For $\mu\in-P^+$, let $p_\mu:\mathcal O_X\otimes_k L^-(\mu)\to\mathcal L(\mu)$ be the surjective morphism of $\mathcal O_X$-modules as $\mathcal L(\mu)$ is generated by its global sections which are $L^-(\mu)$ by \cref{thm-borel-weil-bott}. Since $\mathcal L(\lambda_1+\lambda_2)=\mathcal L(\lambda_1)\otimes\mathcal L(\lambda_2)$ by \cref{rmk-tensor-sheaf}, we see that $\mathcal L(\mu)^\vee=\mathcal L(-\mu)$ and $L^-(\mu)^*\cong L^+(-\mu)$. We thus obtained an injection $i_\mu:\mathcal O_X\to\mathcal L(\mu)\otimes_k L^+(-\mu)$. Tensoring by $\mathcal D_\lambda$-module $\mathcal M$ on the left, we get morphisms $\overline{p_\mu}:\mathcal M\otimes_k L^{-}(\mu)\to\mathcal M\otimes_{\mathcal O_X}\mathcal L(\mu)$ and $\overline{i_\mu}:\mathcal M\to\mathcal M\otimes_{\mathcal O_X}\mathcal L(\mu)\otimes_k L^+(-\mu)$ which are also surjective and injective respectively.
    \begin{lemma}
        The lowest weight module $L^-(\mu)$ has a filtration
        \[0=L^r\subseteq L^{r-1}\subseteq\cdots\subseteq L^2\subseteq L^1=L^-(\mu)\]
        of $B$-submodules such that $L^i/L^{i+1}$ is the one-dimensional $B$-module with character associated to some $\mu_i\in P$, and $\{\mu_1,\dots,\mu_{r-1}\}$ are weights of $L^-(\mu)$ such that if $\mu_i<\mu_j$ then $i<j$ and $\mu_i=\mu$ if and only if $i=1$.
    \end{lemma}
    \begin{proof}
        This is classical representation theory. We can define the filtration using modules spanned by weight vectors, and $\h$-modules in our case give rise to $B$-modules.
    \end{proof}
    \begin{lemma}
        The $\mathcal O_X$-module $\mathcal O_X\otimes_k L^-(\mu)$ has a filtration of $G$-equivariant modules
        \[0=\mathcal V^r\subseteq\cdots\subseteq \mathcal V^2\subseteq \mathcal V^1=\mathcal O_X\otimes_k L^-(\mu)\]
        such that $\mathcal V^i/\mathcal V^{i+1}\cong\mathcal L(\mu_i)$ for $\mu_i$ satisfying the same conditions in the above lemma.
    \end{lemma}
    \begin{proof}
        The trivial bundle $X\times L^-(\mu)$ has a filtration
        \[U^r\subseteq U^{r-1}\subseteq\cdots\subseteq U^2\subseteq U^1=X\times L^-(\mu)\]
        if we let $U^i=\{(gB, x):x\in g(L^i)\}$ for the $L^i$ in the lemma above. Define $\mathcal V^i$ to be the sheaf of sections of $U^i$ which is an $\mathcal O_X$-submodule of $\mathcal O_X\otimes_k L^-(\mu)$, completing the proof.
    \end{proof}
    \begin{proposition}\label{prop-key-lemma}
        For all $\lambda\in P$,
        \begin{enumerate}[\normalfont(i)]
            \item If $\alpha^\vee(\lambda)< 0$ for all positive roots $\alpha$, $\ker \overline{p_\mu}$ is a direct summand of $\mathcal M\otimes_k L^-(\mu)$ as a sheaf of abelian groups.
            \item If $\alpha^\vee(\lambda)\leqslant 0$ for all positive roots $\alpha$, $\im\overline{i_\mu}$ is a direct summand of $\mathcal M\otimes_{\mathcal O_X}\mathcal L(\mu)\otimes_k L^-(\mu)$ as a sheaf of abelian groups.
        \end{enumerate}
    \end{proposition}
    \begin{proof}
        By the previous lemma, $\mathcal M\otimes_k L^-(\mu)$ has a filtration $\{\overline{\mathcal V}^i\}$ such that $\overline{\mathcal V}^i/\overline{\mathcal V}^{i+1}\cong \mathcal M\otimes_{\mathcal O_X}\mathcal L(\mu_i)$ obtained by tensoring with $\mathcal M$. Using a similar fact on the filtration of $L^+(-\mu)$, we may obtain a filtration 
        \[0=\overline{\mathcal W}^1\subseteq\cdots\subseteq\overline{\mathcal W}^{r-1}\subseteq\overline{\mathcal W}^r=\mathcal M\otimes_{\mathcal O_X}\mathcal L(\mu)\otimes_k L^+(-\mu)\]
        of $\mathcal M\otimes_{\mathcal O_X}\mathcal L(\mu)\otimes_k L^+(-\mu)$ such that $\overline{\mathcal W}^{i+1}/\overline{\mathcal W}^i\cong\mathcal M\otimes_{\mathcal O_X}\mathcal L(\mu-\mu_i)$. For any $z\in\mathfrak Z$ and any $\mu\in -P^+$, $(z-\chi_{\lambda+\mu}(z))\mathcal M\otimes_{\mathcal O_X}\mathcal L(\mu)$ vanishes since $\mathcal M\otimes_{\mathcal O_X}\mathcal L(\mu)$ being a $\mathcal D_{\lambda+\mu}$-module is killed by the kernel of $\chi_{\lambda+\mu}$ by \cref{thm-bb-1}. Therefore, the filtrations of $\mathcal M\otimes_k L^-(\mu)$ and $\mathcal M\otimes_{\mathcal O_X}\mathcal L(\mu)\otimes_k L^+(-\mu)$ suggest that they are killed by
        \[\prod_{i=1}^{r-1}(z-\chi_{\lambda+\mu_i}(z))\mbox{ and }\prod_{i=1}^{r-1}(z-\chi_{\lambda+\mu-\mu_i}(z))\]
        respectively. By a linear algebra argument, since $\mathfrak Z$ is commutative, from the above results we can decompose the modules into
        \begin{align*}
            \mathcal M\otimes_k L^-(\mu)&=\bigoplus_{\chi\text{ central}}(\mathcal M\otimes_k L^-(\mu))^\chi\\
            \mathcal M\otimes_{\mathcal O_X}\mathcal L(\mu)\otimes_k L^+(-\mu)&=\bigoplus_{\chi\text{ central}}(\mathcal M\otimes_{\mathcal O_X}\mathcal L(\mu)\otimes_k L^+(-\mu))^\chi
        \end{align*}
        where $(-)^\chi$ is the generalized eigenspace of the action $\mathfrak Z$ with central character $\chi$. Again by the above annihilator results, the central characters of $\mathfrak Z$ on $\mathcal M\otimes_k L^-(\mu)$ are precisely $\chi_{\lambda+\mu_i}$, and on $\mathcal M\otimes_{\mathcal O_X}\mathcal L(\mu)\otimes_k L^+(-\mu)$ are $\chi_{\lambda+\mu-\mu_i}$.

        Now if $\alpha^\vee(\lambda)<0$ for all positive roots $\alpha$ and $\chi_{\lambda+\mu_i}=\chi_{\lambda+\mu}$, then by \cref{prop-hc-char}, there is some $w\in W$ such that $w(\lambda)-\lambda+w(\mu_i)-\mu=0$. Yet $w(\lambda)\geqslant \lambda$ since $\alpha^\vee(\lambda)<0$ for all positive roots. Also, $w(\mu_i)$ is a weight of $L^-(\mu)$ by properties of the Weyl group. By minimality, we must have $w(\mu_i)\geqslant \mu$. In conclusion, $\chi(\lambda)-\lambda=w(\mu_i)-\mu=0$ and thus $w=1$ and $\mu_i=\mu$. If $\alpha^\vee(\lambda)\leqslant 0$ and $\chi_{\lambda+\mu-\mu_i}=\chi_{\lambda}$, then there is some $w$ such that $w(\lambda)-\lambda+\mu_i-\mu=0$. But then $\mu_i=\mu$ by a similar but weaker argument.

        On the other hand, $\overline{p_\mu}$ is the morphism $\overline{\mathcal V_1}\to\overline{\mathcal V_1}/\overline{\mathcal V_2}$, so its kernel is eliminated by a large enough power of $z-\chi_{\lambda+\mu}$. Similarly, $\overline{i_\mu}$ is the map $\overline{\mathcal W_2}\to \overline{\mathcal W_r}$ so its image is killed by a large enough power of $\chi_{\lambda+\mu-\mu}=\chi_\lambda$, completing the proof. 
    \end{proof}
    \begin{proof}[Proof of \cref{thm-bb-2}]
        For the exactness of $\Gamma(X, -)$, assume $\alpha^\vee(\lambda)\leqslant 0$ for all positive roots $\alpha$. Write $\mathcal M=\varinjlim \mathcal N$ where $\mathcal N$ varies among the coherent $\mathcal O_X$-submodules of $\mathcal M$. Then $H^k(X,\mathcal M)=\varinjlim H^k(X,\mathcal N)$. We will show the map $H^k(X,\mathcal N)\to H^k(X,\mathcal M)$ is zero. By \cref{thm-invertible-sheaf-weight}, there is some $\mu\in -P^+$ such that $H^k(X,\mathcal N\otimes\mathcal L(\mu))=0$ (by ampleness and taking large enough tensor product). Then
        \[H^k(X, \mathcal N\otimes_{\mathcal O_X}\mathcal L(\mu)\otimes_k L^+(-\mu))=H^k(X, \mathcal N\otimes_{\mathcal O_X}\mathcal L(\mu))\otimes_k L^+(-\mu)=0\]
        But (ii) in \cref{prop-key-lemma} suggests that $\overline{i_\mu}$ induces an injective map 
        \[H^k(X,\mathcal M)\to H^k(X, \mathcal M\otimes_{\mathcal O_X}\mathcal L(\mu)\otimes_k L^+(-\mu))\]By naturality of $H^k$, we see that $H^k(X,\mathcal N)\to H^k(X, \mathcal M)$ must be zero, so $\Gamma(X, -)$ is exact.
        

        To show that all left $\mathcal D_\lambda$-modules on $X$ are generated by global sections, we define $\mathcal M'$ of $\mathcal M$ to be the submodule generated by global sections. Let $\mathcal M''=\mathcal M/\mathcal M'$. Assume $\mathcal M''\neq 0$. Let $\mathcal N$ be a nonzero coherent $\mathcal O_X$-submodule of $\mathcal M''$, so there is some $\mu\in -P^+$ such that $\Gamma(X, \mathcal N\otimes_{\mathcal O_X}\mathcal L(\mu))\neq 0$. Therefore, $\Gamma(X, \mathcal M''\otimes_{\mathcal O_X}\mathcal L(\mu))$ does not vanish. Yet $\Gamma(X, \mathcal M''\otimes_k L^-(\mu))\to \Gamma(X, \mathcal M''\otimes_{\mathcal O_X}\mathcal L(\mu))$ is surjective by \cref{prop-key-lemma} so $\Gamma(X, \mathcal M'')\neq 0$. But this is impossible since $\mathcal M'$ is generated by global sections of $\mathcal M$ and $\Gamma(X, -)$ is exact.
    \end{proof}







    \subsection{Lie algebroids: a more general approach}
    \label{sec-lie-algebroid}
    Recall I mentioned that there is a general construction of t.d.o. that doesn't rely on $\lambda$ being integral. In this section I will briefly introduce the necessary language used in the general construction of t.d.o. for all $\lambda\in\h^*$. However, some statements in this section will be taken for granted, because we don't have the necessary language such as pullbacks of modules over t.d.o.s and strong $G$-equivariance. The first objects we will study are Lie-Rinehart algebras over fields. They are the algebraic data of Lie algebroids which will be discussed and used later. I will refer to \cite{wadsley-d-modules-I, bekaert-lie-rinehart} as the main sources of this section.
    \renewcommand{\l}{\mathfrak l}
    \begin{definition}
        Let $R$ be a commutative $k$-algebra. A \textbf{Lie-Rinehart algebra} over $R$ is a pair $(\l, \eta)$ where $\l$ is a Lie algebra over $k$ and also an $R$-module, and $\eta:\l\to\Der_k(R)$ a map of both Lie algebras and $R$-modules called the \textbf{anchor map} such that
        \[[x, ry]=r[x, y]+\eta(x)(r)y\]
        for all $x, y\in \l$ and $r\in R$.
    \end{definition}
    \begin{example}
        The $R$-module $\Der_k(R)$ with the identity anchor map is a Lie-Rinehart algebra. Given a Lie algebra $\l$ over $k$, we may consider the free $R$-module generated by $\l$, i.e., $R\otimes_k\l$ on which we have a Lie bracket $[\cdot, \cdot]$ that is an $R$-bilinear extension of the Lie bracket on $\l$, and a trivial anchor map $\eta=0$.
    \end{example}
    \begin{example}\label{exp-twist-lie-rinehart}
        We may also twist the above construction by a Lie algebra map $\tau:\l\to\Der_k(R)$. We still take the vector space $R\otimes_k\l$, but with a Lie bracket defined by
        \[[f\otimes x, g\otimes y]=fg\otimes[x, y]+f\tau(x)(g)\otimes y-g\tau(y)(f)\otimes x\]
        and the anchor map $\eta_{\tau}:f\otimes x\mapsto f\cdot \tau(x)$, we get a new Lie-Rinehart algebra $(R\otimes_k\l, \eta_{\tau})$ called a \textbf{transformation Lie-Rinehart algebra}. The reader may verify it's a Lie-Rinehart algebra.
    \end{example}
    An \textbf{enveloping algebra of $(\l, \eta)$} is an associative $k$-algebra $U$ with a map $i_R:R\to U$ of $k$-algebras and a map $i_\l:\l\to U$ of Lie algebras such that for all $r\in R$ and $x\in \l$,
    \[i_\l(rx)=i_R(r)i_\l(x),\quad [i_\l(x), i_R(r)]=i_R(\eta(x)(r))\]
    Take the enveloping algebra with $i_R$ and $i_\l$ universal among all and denote it by $U_R(\l)$. We can construct it explicitly. The sum $R\oplus \l$ has a Lie algebra structure given by
    \[[f+x, g+y]=(\eta(x)(g)-\eta(y)(f))+[x, y]\]
    that is, it's the semi-direct product of these Lie algebras. There is an inclusion $\iota:R\oplus\l\to U(R\oplus\l)$ given by PBW of Lie algebras. Let $U$ be the $k$-subalgebra generated by its images, and $I$ be the two-sided ideal generated by $\iota(f+0)\cdot \iota(g+x)-\iota(fg+fx)$.
    \begin{proposition}
        The algebra $U_R(\l)=U/I$ is the universal enveloping algebra of $\l$ with $i_R:r\mapsto \iota(r+0)$ and $i_\l:x\mapsto \iota(0+x)$.
    \end{proposition}
    \begin{proof}
        The maps $i_R$ and $i_\l$ clearly satisfy the desired relations. The universal property is not hard to check.
    \end{proof}
    \begin{example}\label{exp-uea-lie-rinehart}
        The universal enveloping algebra of $R\otimes_k \l$ with a given map $\tau:\l\to \Der_k(R)$ in \cref{exp-twist-lie-rinehart} is isomorphic to $R\otimes_k U(\l)$ as $k$-vector spaces.
    \end{example}
    \begin{theorem}[Poincaré-Birkhoff-Witt, {rinehart-lie-rinehart}]\label{thm-pbw-lr-algebra}
        If $\l$ is finite dimensional, then $\Sym(\l)$ is isomorphic to $\gr U_R(\l)$.
    \end{theorem}
    \begin{example}\label{exp-smash-prod}
        \textit{Crucial:} consider the linear algebraic group $G=\Spec R$ with Lie algebra $\g$. We know $R\otimes_k\g\cong \Der_k(R)$ as $k$-vector spaces via the map $r\otimes x\mapsto r\tau(x)$ where $\tau:\g\to \Der_k(R)$ sends $x$ to the left invariant vector field defined by $x$ from the theory of group schemes. Recall $\Der_k(R)$ is Lie-Rinehart with the identity anchor map, and $(\g, \tau)$ naturally becomes Lie-Rinehart. It is immediate that $r\otimes x\mapsto r\tau(x)$ is in fact an isomorphism of Lie-Rinehart algebras over $R$, where the anchor map of $R\otimes_k\g$ is $\eta_\tau:r\otimes x\mapsto r\tau(x)$ as in \cref{exp-twist-lie-rinehart}. But on the other hand, $\gr U_R(\Der_k(R))\cong\Sym\Der_k(R)$ by \cref{thm-pbw-lr-algebra}, where the right hand side is precisely $\gr^F D(R)$ by \cref{prop-d-filtration}. Then by \cref{lem-gr-iso-is-iso}, the algebras $U_R(\Der_k(R))$ and $D(R)$ are isomorphic. Thus by 1.23 in \cite{bekaert-lie-rinehart}, the algebra $D(R)$ is isomorphic to $R\otimes_kU(\g)$ where the algebra structure of the tensor product is given by, in Sweedler notations,
        \[(r\otimes x)(r'\otimes x')=\sum r(x_{(1)}\triangleright r')\otimes x_{(2)}x'\]
        where $\triangleright$ is the Hopf action of $U(\g)$ on $R$ which in our case is the action of differential operators. We denote by $R\# U(\g)$ this algebra and call it the \textbf{smash product} of $R$ and $U(\g)$.
    \end{example}
    \begin{definition}
        On a smooth algebraic variety $X$, a \textbf{Lie algebroid} is a quasicoherent $\mathcal O_X$-module $\mathcal L$ and a morphism of $\mathcal O_X$-modules $\eta:\mathcal L\to\Theta_X$ such that on each affine open $U\subset X$, $(\Gamma(U,\mathcal L),\eta_U)$ is a Lie-Rinehart algebras over $\Gamma(U,\mathcal O_X)$.
    \end{definition}
    \begin{remark}
        In global languages, we have a $k$-linear Lie bracket $[\cdot,\cdot]:\mathcal L\otimes_k\mathcal L\to \mathcal L$ commuting with $\eta$ and for any local sections $x, y\in\mathcal L$, $f\in\mathcal O_X$, we have
        \[[x, fy]=r[x, y]+\eta(x)(f)y\]
        By Serre-Swan, we see that the category of locally free Lie algebroid of finite rank on an affine $X$ is equivalent to the category of Lie-Rinehart algebras projective over $\Gamma(X, \mathcal O_X)$.
    \end{remark}
    \begin{example}\label{exp-universal-lie-algebroid}
        As in \cref{exp-twist-lie-rinehart}, given a Lie algebra $\l$ and a map $\tau:\l\to\Gamma(X, \Theta_X)$, we can define a Lie algebroid $\mathcal O_X\otimes_k\l$ (as usual this denotes the sheaf $U\mapsto \Gamma(U, \mathcal O_X)\otimes_k\l$ on affine opens) with a Lie bracket
        \[[f\otimes x, g\otimes y]=fg\otimes[x, y]+f\tau(x)(g)\otimes y-g\tau(y)(f)\otimes x\]
        after restrictions if necessary for local sections $f, g\in\mathcal O_X$ and $x, y\in\l$. The anchor map of this Lie algebroid is given by $\eta_{\tau}:f\otimes x\mapsto f\cdot\tau(x)$.
    \end{example}
    \begin{definition}
        We define the universal enveloping algebra $U_{\mathcal O_X}(\mathcal L)$ of a Lie algebroid $(\mathcal L, \eta)$ on $X$ to be the sheaf given by $U\mapsto U_{\Gamma(U, \mathcal O_X)}(\Gamma(U, \mathcal L))$ on affine opens.
    \end{definition}

    Recall we've defined a map $\tau_{\mathcal V}:\g\to \Theta_X$ for a given $G$-equivariant locally free sheaf $\mathcal V$ on $X$. Let $\mathcal G$ be the Lie algebroid defined by $\tau_{\mathcal O_X}:\g\to\Gamma(X, \Theta_X)$. We can then consider the universal enveloping algebra $U_{\mathcal O_X}(\mathcal G)$ which in the spirit of \cref{exp-smash-prod} is the sheaf of differential operators. By \cref{exp-uea-lie-rinehart}, the underlying $\mathcal O_X$-module is $\mathcal O_X\otimes_kU(\g)$. On this module, the Lie bracket in \cref{exp-universal-lie-algebroid} is determined by, for any $\xi\in\g$ and $f\in\mathcal O_X$,
    \[[\xi, f]=f[\xi, 1]+\tau_{\mathcal O_X}(\xi)(f)\]
    where I suppressed the tensor symbol. The map $\tau_{\mathcal O_X}$ naturally extends to an $\mathcal O_X$-algebra morphism $\tau_{\mathcal O_X}:U_{\mathcal O_X}(\mathcal G)\to\mathcal D_X$. Define $\tilde\g$ to be the kernel of $\tau_{\mathcal O_X}|_{\mathcal G}$. Then we can consider the subspace
    \[\sum_{a\in\tilde\g}U_{\mathcal O_X}(\mathcal G)(a-\lambda(a))\]
    which turns out to be a two-sided ideal. Set
    \[\mathcal D_X(\lambda)=U_{\mathcal O_X}(\mathcal G)/\sum_{a\in\tilde\g}U_{\mathcal O_X}(\mathcal G)(a-\lambda(a))\]
    We quote the following theorem
    \begin{theorem}
        The above $\mathcal D_X(\lambda)$ are the only strong $G$-equivariant t.d.o.s on $X$.
    \end{theorem}
    \begin{proof}
        See 4.9.2 in \cite{kashiwara-d-modules}. I will leave the strong $G$-equivariance here undefined.
    \end{proof}
    If $\lambda\in P$, given maps $\tau_\lambda:U(\g)\to\Gamma(X, \mathcal D_\lambda)$ and $\mathcal O_X\hookrightarrow\mathcal D_\lambda$, we have a map $U_{\mathcal O_X}(\mathcal G)\to \mathcal D_\lambda$. It can be shown that this map vanishes on $\sum_{a\in\tilde\g}U_{\mathcal O_X}(\mathcal G)(a-(\lambda+\rho)(a))$, so we get a map $\mathcal D_X(\lambda+\rho)\to \mathcal D_\lambda$. Then as the associated graded algebras are both isomorphic to $\Sym\Theta_X$, we have $\mathcal D_X(\lambda+\rho)\cong\mathcal D_{\lambda}$. Therefore, most arguments we employed in the proof of \cref{thm-bb-1} and \cref{thm-bb-2} can be applied to $\mathcal D_X(\lambda+\rho)$ after adjustments. The approach by Lie algebroid is in fact widely used in various generalizations and analogs of Beilinson-Bernstein, say the $p$-adic analytic version in \cite{wadsley-d-modules-I} (which is a very general and powerful construction), the quantum version in \cite{backelin-quantum, tanisaki-quantum}, and the characteristic $p$ version in \cite{bmr-pos-char}.

    \subsection{Quantum Beilinson-Bernstein at generic \texorpdfstring{$q$}{}}
    To further demonstrate the power of Lie algebroids, I will briefly introduce a version of Beilinson-Bernstein for quantum groups. The history of quantum groups started with the integrality of a quantum system, which could be determined by solving for $R$-matrices in the quantum Yang-Baxter equation. Drinfeld, Jimbo, Manin and others discovered and studied comprehensively a special class of Hopf algebras relevant to Yang-Baxter equations in the 1980s. One of the most crucial quantum groups is the following Hopf algebra. Fix an algebraically closed field $k$ of characteristic zero. The directing references for this section are the original papers \cite{MR1694897,backelin-quantum,tanisaki-quantum} of Lunts, Rosenberg, Backelin, Kremnizer and Tanisaki, and the Ph.D. thesis of Nicolas Dupré \cite{dupre-phd}.
    \begin{definition}
        Suppose $\g$ is a semisimple Lie algebra (or a Kac-Moody algebra in general) over $k$ with a fixed Cartan subalgebra $\h$, and $q\neq 0,\pm 1\in k$. Let $\{e_i, f_i, h_i\}_{i\in I}$ be a Cartan-Weyl basis of $\g$. The Drinfeld-Jimbo \textbf{quantum universal enveloping algebra} $U_q(\g)$ for the Cartan datum $(\mathfrak h, \{\alpha_i\}_{i\in I}, \{\alpha_i^\vee\}_{i\in I}, P, P^\vee)$ of $\g$ is the associative algebra over $k$ generated by $\{e_i, f_i, q^{h} : i\in I, h\in P^\vee\}$ such that for $d_i=(\alpha_i,\alpha_i)/2$, $a_{ij}=\alpha_i^\vee(\alpha_j)$:
        \begin{enumerate}[(i)]
            \item For all $h, h'\in P^\vee$, $q^0 = 1, q^hq^{h'}=q^{h+h'}$, $q^he_iq^{-h} = q^{\alpha_i(h)}e_i$, and $q^hf_iq^{-h} = q^{-\alpha_i(h)}f_i$,
            \item For all $i, j\in I$, $e_if_i-f_ie_i = \delta_{ij}\frac{q^{d_ih_i}-q^{-d_ih_i}}{q^{d_i}-q^{-d_i}}$.
            \item And summing over $k$,
        \end{enumerate}
        \begin{align*}
            (-1)^k\left[\begin{smallmatrix} 1-a_{ij} \\ k \end{smallmatrix}\right]_{q^{d_i}}e_i^{1-a_{ij}-k}e_je_i^k=0, \quad (-1)^k\left[\begin{smallmatrix} 1-a_{ij} \\ k \end{smallmatrix}\right]_{q^{d_i}}f_i^{1-a_{ij}-k}f_jf_i^k&=0
        \end{align*}
        where $\left[\begin{smallmatrix} a \\ b \end{smallmatrix}\right]_{q}=\frac{[a]_q!}{[a-b]_q![b]_q!}$ with $[a]_q=(q^a-q^{-a})/(q-q^{-1})$.
    \end{definition}
    There is a Hopf algebra structure on $U_q(\g)$ given by
    \[
        \begin{matrix}
            \D(e_i)=e_i\otimes 1+q^{d_i h_i}\otimes e_i & S(e_i)=-q^{-d_ih_i}e_i & \ve(e_i)=0\\
            \D(f_i)=f_i\otimes q^{-d_i h_i}+1\otimes f_i & S(f_i)=-f_iq^{d_ih_i} & \ve(f_i)=0\\
            \D(q^{\pm h_i})=q^{\pm h_i}\otimes q^{\pm h_i} & S(q^{h_i})=q^{-h_i} & \ve(q^{h_i})=1
        \end{matrix}     
    \]
    When $q\to 1$, the classical limit of $U_q(\g)$ is precisely the Hopf algebra $U(\g)$, although strictly speaking for some authors $U_q(\g)$ is not a deformation algebra of $U(\g)$. This quantum group is particularly interesting because for a specific $q\neq 0,\pm 1$, $U_q(\g)$ actually makes sense, and its representations are rich. 
    
    In this section, we will study $U_q(\g)$ at $q$ \textit{not a root of unity}. From now on, write $U_q=U_q(\g)$ and $K_i=q^{d_ih_i}$ for convenience. Let $U_q(\mathfrak n^+)$, $U_q(\mathfrak n^-)$ and $U_q(\h)$ be the subalgebras of $U_q$ generated by $\{e_i\}$, $\{f_i\}$ and $\{h_i\}$ respectively. Let $U_q(\b^\pm)$ be the subalgebra generated generated by $U_q(\mathfrak n^\pm)$ and $U_q(\h)$. It is not hard to show that $U_q(\b^\pm)$ are Hopf subalgebras using the structures on $U_q$. Similar to the classical case, there is a triangular decomposition $U_q\cong U_q(\mathfrak n^+)\otimes U_q(\h)\otimes U_q(n^-)$ and a Poincaré-Birkhoff-Witt theorem for $U_q$, although we shall not spell out the details. Now, for a root $\alpha\in Q$ (recall we defined pretty early that $Q$ was the root lattice), if $\alpha=\sum_i n_i\alpha_i$ then we write $K_\alpha=\prod_i K_i^{n_i}$. Given a representation $(V, \rho_V)$ of the algebra $U_q$, an additive function $\Lambda:Q\to k$ is a \textbf{weight} of $U_q$ wrt $\rho_V$ if the weight space
    \[V_\Lambda=\{v\in V:\forall \alpha\in Q, \rho_V(K_\alpha)v=\Lambda(\alpha)v\}\]
    is nonzero. If $V$ decomposes into weight spaces, we say it's a \textbf{highest weight module} with highest weight $\Lambda$ if there is a nonzero $e_{\Lambda}$ in $V_\Lambda$ such that $U_q\cdot e_{\Lambda}=V$ and $T(e_i)e_\Lambda=0$ for all $i$. We say $\rho_V$ is a \textbf{representation of type 1} if it is a highest weight module $L^+_q(\Lambda)$ with a highest weight $\Lambda$ of the form $\Lambda(\alpha)=q^{\lambda(\alpha)}$ for some $\lambda\in\h^*$.
    \begin{theorem}
        If $q$ is not a root of unity, then every finite-dimensional representation of $U_q$ is the tensor product of a representation of type 1 and a one-dimensional representation of $U_q$.
    \end{theorem}
    \begin{proof}
        See 7.1.1.5 in \cite{klimyk-uq-rep}.
    \end{proof}
    \begin{theorem}
        Every finite-dimensional representation of $U_q$ is completely reducible.
    \end{theorem}
    \begin{proof}
        See 5.12 in \cite{MR1198203}.
    \end{proof}
    It is therefore tempting to ask if there is an analog of Beilinson-Bernstein for $U_q$ at $q$ not a root of unity, as the representation theory of $U_q$ is almost parallel to the classical case. One would soon realize that the main issues we have are: the absence of a nice quantized geometric object corresponding to the classical flag varieties, and the notion of differential operators in noncommutative settings, so we can't build sheaves or $D$-modules in the classical way. One essential reason why Beilinson-Bernstein is true is the duality between $U(\g)$ and $\mathcal O(G)=\Gamma(G, \mathcal O_G)$. There is a nondegenerate Hopf pairing $\langle\cdot,\cdot\rangle:U(\g)\times\mathcal O(G)\to k$ in the sense that, in Sweedler notations, $\langle uv, x\rangle=\langle u, x_{(1)}\rangle\langle v, x_{(2)}\rangle$, $\langle u, xy\rangle=\langle u_{(1)}, x\rangle\langle u_{(2)}, y\rangle$, $\langle 1, x\rangle=\ve(x)$, $\langle u, 1\rangle=\ve(u)$ and $\langle S(u), x\rangle=\langle u, S^{-1}(x)\rangle$. Here the comultiplication of $U(\g)$ is given by $\D(u)=1\otimes u+u\otimes 1$ and of $\mathcal O(G)$ is given by $\D(x)=x\otimes x$. Of course, the pairing induces/is given by the action of $\g$ we used to define $\tau_\lambda$. In the remarkable work \cite{FADDEEV1988129} of Faddeev, Reshetikhin and Takhtajan, a quantization of $\mathcal O(G)$ is constructed via the so-called FRT construction wrt the standard $R$-matrix in $U_q(\g)$. We denote the resulting Hopf algebra by $\mathcal O_q(G)$. Indeed, we still have Hopf pairings $\langle\cdot,\cdot\rangle_q$ between $U_q(\g)$ and $\mathcal O_q(G)$. So it is reasonable to start with $\mathcal O_q(G)$ which is a subalgebra of the Hopf dual of $U_q(\g)$.
    
    There are two major approaches to resolving this difficulty, and we will briefly describe them. The first one is formulated and proposed in a series of papers \cite{MR1694897, MR1481133} by Lunts and Rosenberg, and later completed by Tanisaki in \cite{tanisaki-quantum}. The guiding principle of their work is the combination of the Gabriel-Rosenberg reconstruction theorem and a theorem of Serre, stating that given a $\Z$-graded ring $A$ generated by its degree one elements, the category of quasicoherent sheaves on $\Proj A$ is equivalent to the localization of the category of graded $A$-modules by the subcategory of torsion graded $A$-modules. Artin and Zhang developed the theory of noncommutative projective schemes in \cite{MR1304753}. Given a $\Z$-graded ring $A$ possibly noncommutative, they defined the \textbf{general projective scheme $\Proj_\Z A$} to be triple $(\text{Gr}_\Z A/\text{Tor}_\Z, \mathscr A, s)$ where (i) $\text{Gr}_\Z A/\text{Tor}_\Z A$ is the localization of left $\Z$-graded $A$-modules by the full subcategory of torsion modules, (ii) $\mathscr A$ together with a module whose endomorphisms recover $A$ and an autoequivalence induced by the shift operator. The flag varieties $X=G/B$ are GIT quotients of the base affine space $\tilde X=G/N$ by the action of the maximal torus $H$ on the right. The base affine space $\tilde X=G/N$ has global sections being the $P$-graded ring $A=\Gamma(\tilde X, \mathcal O_{\tilde X})=\oplus_{\lambda\in P^+}L^+(\lambda)$ (this is a consequence of the fact that $\Gamma(\tilde X, \mathcal O_{\tilde X})=\mathcal O(G)^N$ which decomposes into the desired direct sum by analyzing the Peter-Weyl decomposition of $\mathcal O(G)$). In fact, the category of quasicoherent sheaves on $X$ is then equivalent to the localization $\Proj_PA=\text{Gr}_PA/\text{Tor}_P A$ of $P$-graded left $A$-modules by the subcategory of torsion modules. The key observation in the quantum setting is the decomposition
    \begin{proposition}
        We have
        \[\smash{A_q=\{\varphi\in\mathcal O_q(G):\forall x\in U_q(\mathfrak n^-), \varphi\cdot x=\ve(x)\varphi\}\cong\bigoplus_{\lambda\in P^+}L_q^+(q^{\lambda})}\]
        where $\mathcal O_q(G)$ is regarded as a $U_q$-bimodule via $x\cdot\varphi=\varphi_{(1)}\langle x, \varphi_{(2)}\rangle_q$ and $\varphi\cdot x=\langle x, \varphi_{(1)}\rangle_q\varphi_{(2)}$. In particular, $A_q$ is a $P$-graded $k$-algebra with zero degree elements being $k$.
    \end{proposition}
    \begin{proof}
        The standard argument still relies on the Peter-Weyl decomposition of $\mathcal O_q(G)$, which can be found in most texts on representations of $U_q$.
    \end{proof}
    We define the \textbf{category of quasicoherent sheaves $\Mod(G_q/B_q)$} on the quantized flag variety as $\Proj_P A_q=\text{Gr}_P A_q/\text{Tor}_PA_q$ ($B_q$, $G_q$ are just notations). To define differential operators in $\Mod(G_q/B_q)$, Lunts and Rosenberg developed a nice theory of differential operators on noncommutative rings in \cite{MR1481133}. But Tanisaki realized that their construction resulted in something too large for Beilinson-Bernstein. Instead, in \cite{tanisaki-quantum}, he defined the \textbf{ring of differential operators on the quantized flag variety} to be the subalgebra $D_q$ in $\End_k(A_q)$ generated by $l_\varphi, r_\varphi$, $\partial_x$ and $\s_\lambda$ where $\varphi\in A_q$, $x\in U_q$ and $\lambda\in P$. Here $l_\varphi$ and $r_\varphi$ are left and right multiplications by $\varphi$, $\partial_x$ is the left action of $x$ on $A_q$ and $\s_\lambda$ is the multiplication by $q^{(\lambda, \mu)}$ for homogeneous elements of degree $\mu$ in $A_q$. It is not hard to define a grading of $D_q$. Let $\text{Gr}_{P,\lambda}D_q$ be the full subcategory of $\text{Gr}_P D_q$ of modules $M$ with $\s_\mu$ acting on the $\xi$th homogeneous component of $M$ by $q^{(\mu,\lambda+\xi)}$. We call the localization $\Mod_\lambda(D_q)=\text{Gr}_{P,\lambda}D_q/(\text{Tor}_PD_q\cap\text{Gr}_{P,\lambda}D_q)$ the \textbf{category of $D$-modules on the quantized flag variety}. Tanisaki proved that
    \begin{theorem}[\cite{tanisaki-quantum}]
        Given $\lambda\in P^+$, $\Mod_\lambda(D_q)$ is equivalent to the category of $U_q$-modules with central character $\zeta_\lambda$ sending $x=f\otimes K_\alpha\otimes e$ first to $\ve(e)\ve(f)\alpha$ in $k[P]$ and then applying the morphism $k[P]\to k$ defined by $\mu\mapsto q^{(\lambda,\mu)}$. 
    \end{theorem}

    Before we discuss the second approach by Backelin and Kremnizer, let's recall the results of \cref{prop-B-equivariant} and \cref{prop-b-equiv-module-comodule}. We showed that quasicoherent sheaves on $G/B$ are equivalent to $\mathcal O(G)$-modules that are $\mathcal O(B)$-comodules. We know $\mathcal O(G)$ and $\mathcal O(B)$ have deformations $\mathcal O_q(G)$ and $\mathcal O_q(B)$, so we can try to work with a quantum version of $\Mod_{\mathcal O(B)}(\mathcal O(G))$ in the notations of \cref{prop-b-equiv-module-comodule}. We have a map $\mathcal O_q(G)\to \mathcal O_q(B)$ which makes $\mathcal O_q(G)$ into a $\mathcal O_q(B)$-comodule. Define a \textbf{$B_q$-equivariant sheaf} to be a triple $(M, \alpha,\beta)$ where $M$ is a vector space, $\alpha:\mathcal O_q(G)\otimes M\to M$ is a left $\mathcal O_q(G)$-module structure and $\beta:M\to M\otimes \mathcal O_q(B)$ is a right $\mathcal O_q(B)$-comodule structure such that $\alpha$ is a map of right $\mathcal O_q(B)$-comodules. Denote by $\Mod_{B_q}(G_q)$ the category of such sheaves. Backelin and Kremnizer showed that this category is very nice and is in fact equivalent to $\Mod(G_q/B_q)$ constructed by Lunts and Rosenberg. This construction gives us the benefit of working with deformation of algebras on $G$, which are very easy to formulate and construct and since $G$ is affine. The global section functor in this setting is easily defined by taking the $B_q$-invariants: for $M\in\Mod_{B_q}(G_q)$, $\Gamma(M)=\{m\in M:\beta_M(m)=m\otimes 1\}$ where $\beta(m)$ is the $\mathcal O_q(B)$-comodule structure of $M$.
    
    To define the ring of differential operators, we recall the crucial example \cref{exp-smash-prod}. We showed that $D(\mathcal O(G))=\mathcal O(G)\# U(\g)$. So why not define \textbf{the ring of differential operators on the quantized flag variety} as $D_q=\mathcal O_q(G)\# U_q(\g)$? Let's do this! Well, a problem immediately appears: $D_q$ is not in $\Mod_{B_q}(G_q)$. The Hopf algebra $U_q$ action on $D_q$ are induced by: (i) on $\mathcal O_q(G)$ by $x\cdot r=r_{(1)}\langle x,r_{(2)}\rangle_q$ and (ii) on $U_q$ by the standard adjoint representation $\ad(x)(y)=\sum x_{(1)}yS(x_{(2)})$ (in the classical limit $q=1$ we get $x_{(1)}=x_{(2)}=x$ for all $x\in\g$ so $\ad(x)(y)$ is the classical adjoint representation). Therefore the action is $x\cdot (r\otimes y)=(x_{(1)}\cdot r)\otimes \ad(x_{(2)})(y)$. But a $\mathcal O_q(B)$-coaction on $D_q$ relies on a crucial property of the representation: local finiteness (c.f. I.9.16 in \cite{MR1898492}, or the notion of integrable modules in the classical setting). In classical settings, $U(\g)$ is a locally finite representation of itself via the adjoint action, i.e., $\dim_k \ad(U(\g))x<\infty$ for all $x\in U(\g)$. But this is not true for $U_q(\g)$! However, by \cite{MR1198203}, the finite part $U^{f}_q=\{x\in U_q:\dim_k \ad(U_q)x<\infty\}$ is a subalgebra of $U_q$ and the finite part $D_q^f=\mathcal O_q(G)\# U^f_q$ sits inside $\Mod_{B_q}(G_q)$. We then define \textbf{a $(B_q, \lambda)$-equivariant $D_q$-module} to be a triple $(M,\alpha,\beta)$ where $M$ is a vector space with a left $D_q$-action $\alpha$ and a right $\mathcal O_q(B)$-action $\beta$ which induces an $U_q(\b)$-action such that (i) the $U_q(\b)$-action of $\alpha\otimes\id$ on $M\otimes k_\lambda$ is $\beta\otimes\lambda$ and (ii) the map $\alpha$ is $U_q(\b)$-linear. Denote by $\Mod(D_q,B_q,\lambda)$ the category of all $(B_q, \lambda)$-equivariant $D_q$-modules. Further define $U_{q, \lambda}=U^f_q/J_\lambda$ where $J_\lambda$ is the annihilator of the Verma module $M_\lambda$ in $U_q^f$. Backelin and Kremnizer proved that
    \begin{theorem}[\cite{backelin-quantum}]
        The categories $\Mod(D_q,B_q,\lambda)$ and $\Mod(U^f_{q, \lambda})$ are equivalent.
    \end{theorem}

    How are the two approaches related? Strictly speaking they are not equivalent. But if we take $D_q^f$ to be the subalgebra of $\End_k(A_q)$ generated by $l_\varphi$, $\partial_x$ and $\s_\lambda$ where $\varphi\in A_q$, $x\in U_q$ and $\lambda\in P$ in Tanisaki's approach (no more right translations), then by 9.1 in the preprint \cite{tanisaki-comparison},
    \begin{proposition}\label{prop-tanisaki-equiv}
        The categories $\Mod(D_q,B_q,\lambda)$ and $\Mod_\lambda(D^f_{q})$ are equivalent.
    \end{proposition}
    \begin{remark}
        I will end this paper by a final remark. The quantum analogs of both \cref{thm-bb-1} and \cref{thm-bb-2} were proved in the two different approaches. The proof of quantum \cref{thm-bb-1} in \cite{backelin-quantum} is highly similar to the one we did in the classical setting, contrary to an extremely different one in \cite{tanisaki-quantum}. However, observed by both Tanisaki in \cite{tanisaki-root-of-unity-2} and Dupré in his thesis, the computation of global sections in \cite{backelin-quantum} has gaps. Their argument relies on results in \cite{MR1262429}, in which a filtration of $U^f_q$ is defined and a quantum version of Kostant's theorem is proved. Tanisaki claims that the filtration does not induce the PBW filtration on $U(\g)$ when $q\to 1$. Yet I was unable to follow the paper \cite{MR1262429} and I couldn't even locate the crucial results due to its complexity. So unfortunately I cannot comment on the issue of Backelin and Kremnizer's proof. But luckily, we have \cref{prop-tanisaki-equiv} and the fact that Tanisaki produced a successful proof of the global section theorem in \cite{tanisaki-comparison}.
    \end{remark}
    
    \renewcommand{\section}[2]{\vskip 0.01em}
    \printbibliography
\end{document}